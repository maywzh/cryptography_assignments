\documentclass[11pt,en,authoryear]{elegantpaper}

\title{ AI for Game Testing and Measurement}
\author{Wangzhihui Mei \\ 2019124044 6603385}
\institute{UOW-CCNU JI}

\date{\today}

\begin{document}

\maketitle

% \begin{abstract}
 
% \keywords{ Game AI, Template}
% \end{abstract}


\section{Introduction}
\noindent
Game Artificial Intelligence (GAI) has been developed into real Artificial Intelligence rather then pseudo AI based on predefined automatic program. The achievement of AlphaGo in 2005 indicated that AI techniques have made greate development, while the game industry benefit little from the technology development. There is still a lot of room for exploration. One of these directions is to help Players improve the game experience, such as new game modes between AI players and human players, more realistic NPCs, etc. Another important direction is to help game production improve efficiency, such as reducing the coding difficulty of rules and behavior trees, game testing, game level generation, etc. 
 
\noindent
In this article we focus on how AI improves the effectiveness and efficiency of game testing. 

\section{Problem Statement}
\noindent
The traditional game development and testing workflow has run efficiently for a long time, but still fail to meet the challenges of high-speed iterations in today's gaming industry.
When the scenario that needs to be tested in game contains more strategic content, the traditional testing method seems to be a little powerless. Conventional automated testing methods can only be used to ensure that the game can run normally, and can not reliably and efficiently test the playability of the game. Also, The game interface animation are quite stochastic and complex, which makes it hard for automated test as the testing program agents may become over complex and enormous. Some teams initially use some rule-based test cases or robot-assisted test work, but simple rules often fail to adequately test the game scene or even pass customs. Complex rules can incur huge maintenance costs. As the content of the game is constantly adjusted during development, it is ultimately necessary to rely on a relatively low-cost manual method to complete the test. Not only that, games with more complicated rules, such as card games and war chess games, will generate a huge amount of strategy combinations, and it is extremely difficult to effectively cover all scenes with human resources. The beta test before the game goes online is one of the important links before the game is officially launched. This work is completely dependent on the players involved. Limited by the level of participation in the test of players and a variety of subjective factors, not only the feedback cycle is long, the test benchmarks and results can basically only be judged by experience. In summary, there is a need for better game testing methods and tools. The widely used AI techniques can be applied in these works. More specifically, the following research questions need to be addressed:
\begin{itemize}
    \item What are the general procedures of different types game testing?
    \item How to apply AI techniques to adequately complete the game testing task?
    \item What are the current industry practice in game testing based on AI?
\end{itemize}

\section{Objectives}
\noindent
The rapid development of AI technology in recent years has brought new possibilities to solve the dilemma in testing work. The main objective is to evaluate the effectiveness and advantages of game testing based on AI techniques. This includes evalutation of performance, validity and feasibility of game-testing AI. Also comprehensive reviews and analyses of techniques of game-testing AI will be conducted. This can reveal the details and principles of game-testing AI. Additionally, several instances of practical application of game-testing AI will be evalutated. Particularly, the research has the following sub-objectives:
\begin{itemize}
    \item To provide a comprehensive evalutaion of game-testing AI;
    \item To review the technical details and designs of various game-testing AI;
    \item To currentj industry practices in regards to game-testing AI;
\end{itemize}

\section{Preliminary Literature Review}
The advantages of automated testing are speed and scale, and automated testing with AI technology is no exception. In the following cases, testing can be performed in parallel in a distributed environment. The game itself runs in a container, and the AI ​​agent controls the operation of the game through the program interface, thereby simulating the user's behavior.

\section{Methodology}
Literature review and instance analysis will be the primary research method. Application scenario analysis and description will be the very first step to introduce the work procedure of game testing based on AI agent. The study will first focus on different typical application scenerios of game-testing AI. Based on the introduction, classfication methods. W


\bibliography{ref}

\end{document}
