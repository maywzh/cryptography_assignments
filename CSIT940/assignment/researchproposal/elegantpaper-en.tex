\documentclass[11pt,en,authoryear]{elegantpaper}

\title{ AI for Game Testing and Measurement}
\author{Wangzhihui Mei \\ 2019124044 6603385}
\institute{UOW-CCNU JI}

\date{\today}

\begin{document}

\maketitle

% \begin{abstract}
 
% \keywords{ Game AI, Template}
% \end{abstract}


\section{Introduction}
\noindent
Game Artificial Intelligence (GAI) has been developed into real Artificial Intelligence rather then pseudo AI based on predefined automatic program. The achievement of AlphaGo in 2005 indicated that AI techniques have made greate development, while the game industry benefit little from the technology development. There is still a lot of room for exploration. One of these directions is to help Players improve the game experience, such as new game modes between AI players and human players, more realistic NPCs, etc. Another important direction is to help game production improve efficiency, such as reducing the coding difficulty of rules and behavior trees, game testing, game level generation, etc.

\noindent
In this article we focus on how AI improves the effectiveness and efficiency of game testing. 

\section{Problem  Statement}
The traditional game development and testing workflow has run efficiently for a long time, but still fail to meet the challenges of high-speed iterations in today's gaming industry.

When the scenario that needs to be tested contains more strategic content, the traditional testing method seems to be a little powerless. Conventional automated testing methods can only be used to ensure that the game can run normally, and can not reliably and efficiently test the playability of the game.

Some teams initially use some rule-based test cases or robot-assisted test work, but simple rules often fail to adequately test the game scene or even pass customs; complex rules can incur huge maintenance costs. As the content of the game is constantly adjusted during development, it is ultimately necessary to rely on a relatively low-cost manual method to complete the test.

Not only that, games with more complicated rules, such as card games and war chess games, will generate a huge amount of strategy combinations, and it is extremely difficult to effectively cover all scenes with human resources.

The beta test before the game goes online is one of the important links before the game is officially launched. This work is completely dependent on the players involved. Limited by the level of participation in the test of players and a variety of subjective factors, not only the feedback cycle is long, the test benchmarks and results can basically only be judged by experience.


\section{}
The rapid development of AI technology in recent years has brought new possibilities to solve the dilemma in testing work. The use of deep neural networks, reinforcement learning and other technologies to create AI agents can respond to complex game strategies like high-level players, and assist QA to fully complete complex scene testing.

Through user data, AI can also simulate the real behavior of players at different levels, which can largely replace human players for large-scale testing, fully cover test scenarios, and shorten the feedback adjustment cycle.

Not only that, due to the generalization of the neural network, AI requires only a few adjustments to deal with the changes brought about by the game upgrade, greatly reducing maintenance costs. And AI has features such as copyability, acceleration, and stable test results, which are impossible with traditional test methods.

\section{Donation}
The advantages of automated testing are speed and scale, and automated testing with AI technology is no exception. In the following cases, testing can be performed in parallel in a distributed environment. The game itself runs in a container, and the AI ​​agent controls the operation of the game through the program interface, thereby simulating the user's behavior.

\bibliography{ref}

\end{document}
