% ************************** Thesis Abstract *****************************
% Use `abstract' as an option in the document class to print only the titlepage and the abstract.

%Please list 3-5 keywords, and replace them with "keyword1", "keyword2", "keyword3",...
\begin{abstract}{Graph Neural Network}{Knowledge Point Labeling}{knowledge tracing}{Recommended Exercises}{}
	After 2010, artificial intelligence technology has gradually become a popular research topic in computer technology. In particular, the advent of AlphaGo has aroused great concern in the industry for the prospect of artificial intelligence. This has brought about the industry's explosive development and raised a large number of research topics. In artificial intelligence-related research, various algorithm innovations, theoretical breakthroughs, and model applications are emerging one after another, laying the foundation for various industries' intelligence. The application of artificial intelligence technology in the field of education has also given birth to the emergence of the concept of intelligent education. Among them, adaptive learning is one of the popular application fields in intelligent education~\cite{chen2018recommendation}. Adaptive learning models generally track students' learning status by combining extensive data analysis of massive student group learning data and precise data analysis of target student individual data, targeting personalizing the learning path according to the students' individual characteristics and the proficiency of knowledge mastery~\cite{soltani2019adaptive}. Adaptive learning technology can use automated machine learning algorithms to complete student evaluation and teaching plans that required much manual labor in the past, which can systematically alleviate the current scarcity and uneven distribution of domestic educational resources and reduce the burden on education practitioners and students. It also has excellent development prospects and commercial value. More and more artificial intelligence research teams and intelligent education technology companies on the market focus on developing and applying adaptive learning tools. Some smart educational technology companies have used adaptive learning as the core function or main selling point of their products. Adaptive educational technology can comprehensively analyze students' individual-level learning ability, knowledge proficiency, group-level popular learning resources, error-prone questions, etc. The most suitable learning path and learning resources such as exercises, materials, and knowledge points can be pushed to the student. According to the students ' knowledge status, the system automatically adjusts the knowledge focus of the pushed learning resources to prevent the repetitive practice of already mastered knowledge points or lack of practice of unmastered knowledge points.

	On the one hand, teachers can analyze the whole class's knowledge mastery proficiency based on the system's data or visual charts to create a learning status assessment report for each student and adaptively adjust the overall teaching plan. On the other hand, students can use the system to analyze their knowledge weaknesses and target appropriate exercises. Thus, adaptive learning is one of the potentially feasible solutions to the ``automatic assessment of students'' knowledge mastery status and instructional program generation'' in online education.

	The purpose of this paper is to propose a knowledge-tracing-based model for recommending high school mathematics exercises, using the subject of high school mathematics as the primary research context. In high school mathematics, practice exercises are the principal means for students to improve their learning ability. However, in the current high school mathematics teaching, teachers or students need to find suitable exercises to practice from a vast library of exercises, which are often too large, highly repetitive, and confusingly organized. There are many low-quality unlabeled exercises in the exercise bank, which need to be manually labeled with knowledge points. Some students study through excessive exercises tactic, but this is less efficient and often results in repetition of familiar knowledge and avoidance of unfamiliar knowledge. To improve the effectiveness of students' exercises and thus enhance the proficiency and comprehensiveness of knowledge acquisition, the experienced teaching staff is needed to conduct an analysis of students' knowledge status and select appropriate exercises from the exercise bank for a recommendation. The method is uneconomical and inefficient because of its high manual workload, its reliance on expert a priori knowledge, and its inclusion of a large amount of repetitive work. Besides, the traditional exercise recommendation takes the student group as the minimum granularity. However, it does not perform recommendations based on the knowledge mastery proficiency of specific students, which ignores the problem that different students have different learning abilities, so the recommendation is less fine-grained and ineffective for most students. The method is uneconomical and inefficient because of its high manual workload, its reliance on expert a priori knowledge, and its inclusion of a large amount of repetitive work. Also, the traditional exercise recommendation takes the student group as the minimum granularity while does not recommend for the knowledge mastery proficiency of specific students, which ignores the problem that different students have different learning abilities, so the recommendation is less fine-grained and ineffective for most students. To solve the problems of existing exercise recommendation methods, knowledge tracing techniques can be applied to track students' learning and thus target automated exercise recommendations. This thesis aims to design an exercise recommendation system based on knowledge point labeling, knowledge tracing, and resource recommendation techniques and thus introduce an intelligent adaptive learning solution in terms of exercise recommendation.

	The proposed recommendation system for high school mathematics exercises consists of three modules: the exercise knowledge point labeling module, the knowledge tracing module, and the recommendation module. The exercise knowledge point labeling module's function is to perform knowledge point labeling for exercises without knowledge point labeling, thus replacing the traditional manual knowledge point labeling with an automated form. The knowledge point labeling is the pre-work of the exercise recommendation, and the knowledge labeled exercises can be used as the data source of the exercise recommendation system. The knowledge tracing module calculates students' knowledge proficiency state vector, which represents students' mastery of subject knowledge points, concepts, and skills, by tracking students' exercise records. Knowledge tracing is the core part of the system. In the final exercise recommendation module, there are two stages of matching and sorting; the former is applied to the exercise database to apply a variety of matching strategies to quickly filter the exercises and generate a collection of recommended candidate exercises, and the latter inputs the collection into the knowledge tracing system in the sorting stage for refined recommendation sorting to generate the final recommendation results.
	\begin{itemize}
		\item Chapter 2 proposes a multi-knowledge point labeling method for high school mathematics exercises based on bidirectional LSTM and graph neural network. The exercise knowledge point labeling module contains two sub-modules: exercise text mining and multi-knowledge point label classification. Since most of the corpus exercises contain only unstructured data such as textual information, this paper focuses on knowledge point extraction utilizing exercise text mining. It applies a bidirectional LSTM network with an attention mechanism to perform exercise text mining. The exercises are firstly pre-processed by word separation, cleaning, regularization, and other pre-processing steps to obtain word sequences while filtering out a large amount of interference of irrelevant information. Next, the pre-trained BERT vector generation approach is used as the learning word embedding vector instead of the simple one-hot encoding approach, preventing the dimensional disaster caused by the sparse input word vector matrix. Moreover, the hidden dependencies between word vectors can also be characterized by embedding learning conducive to inter-knowledge point dependencies. After that, text information extraction by a bidirectional LSTM model can better solve the problem of long-range dependency of contextual elements in the text. Also, to capture inter-knowledge point dependencies on the classification model, a multi-label knowledge point labeling model based on graph convolutional network (GCN) is proposed in this paper, where a graph embedding of knowledge points represents each label, and the label graph is mapped into a set of intrinsically dependent knowledge point classifiers after several rounds of iterative learning. Subsequently, the word vector of the exercise text extracted from the previous sub-network is fed into the set of knowledge point classifiers to derive a multi-knowledge point prediction probability vector, thus realizing the multi-knowledge point labeling task. In the experimental phase, the proposed method in this thesis is compared with a series of baseline models by conducting experiments on a self-made high school mathematics exercise dataset, and a series of multi-label classification metrics are used to compare and evaluate the model performance. The experimental results show that the method achieves more superior performance on the more complex sets of exercises with more complex knowledge point relationships.s method has achieved superior performance on the problem sets with more complex knowledge point relationships.
		\item Chapter 3 proposed an improved model for dynamic key-value memory networks (DKVMN).  The model inherits the idea of calculating the relevance of exercises and students' mastery based on knowledge point weights from the original DKVMN model, and has the following improvements compared to the original DKVMN model. The first improvement is an attempt to incorporate student answer features such as answer delays and request hints into the model, thus capturing the impact of students' characteristics on the answers to the exercises. Also, the original DKVMN model does not take into account the interaction of related knowledge points, but treats knowledge points as independent memory units. The second improvement point of the model is to incorporate the information propagation idea of graph neural networks into the model to address this deficiency. In the process of correcting the model for knowledge point mastery proficiency, the propagation mechanism of adjacent nodes of the graph network is used to readjust the proficiency change of related knowledge points, thus adding the ability to characterize the relationship between knowledge points to the model. In the experimental phase, various aspects are compared with the original DKVMN model and some other baseline models on a publicly available dataset. The results show that the model's performance and interpretability are improved relative to the original DKVMN model and other baseline models.
		\item Chapter 4 presents a mathematical exercise recommendation model based on 2 phases of Matching-Ranking. The first stage is the matching model, which is a hybrid model based on multiple matching strategies with two processes: multiplex-matching and merging. In the process of multiplex-matching, multiple matching strategies such as collaborative filtering, popularity, user preferences are used to generate several subsets of exercise recommendation candidates separately. Then, in the process of merging, these candidate subsets are merged by weighted ranking to form a final set of exercise recommendation candidates. The second stage is a knowledge-tracing-based recommendation item ranking model, in which each exercise in the set of exercise candidates obtained in the previous stage is input to the knowledge-tracing model proposed in the previous chapter for correctness prediction, and the one with the most error-prone exercises is used as the recommendation item with the highest priority. The model mainly solves the problem that traditional models often require user-initiated ratings for recommendation model construction and cannot make efficient recommendations based on the user's knowledge mastery proficiency state. After the performance test on the public dataset and the control experiment with the baseline model, the proposed model has a better performance in tracking the students' weak knowledge mastery and can recommend the appropriate exercises based on the mastery proficiency status obtained from the tracking.
	\end{itemize}

	This paper analyzes the system's requirements, rationalizes the entire recommendation system into multiple modules, and designs different neural network models and algorithms for each module to achieve and optimize the above three modules. It has both algorithm design and experimental verification methods—a certain degree of innovation. After experimental verification, it has better performance than similar models.
\end{abstract}
