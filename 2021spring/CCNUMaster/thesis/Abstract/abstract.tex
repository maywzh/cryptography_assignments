% ************************** Thesis Abstract *****************************
% Use `abstract' as an option in the document class to print only the titlepage and the abstract.

%Please list 3-5 keywords, and replace them with "keyword1", "keyword2", "keyword3",...
\begin{abstract}{Graph neural network}{Knowledge point labeling}{Knowledge tracing}{Exercise recommendation}{}
    %高中数学习题推荐系统是一种自适应学习系统,它应用于高中数学学科中,通过跟踪评估学生的知识掌握水平来为学生提供最合适的习题,可以解决人工教育无法解决的个体性知识评估和学习资源推荐。本文立足于这个研究背景,提出一种基于知识追踪的习题推荐系统,它包含三个模块,分别是作为习题库预处理部分的习题知识点标注模块,作为系统核心部分的知识追踪模块,以及最终的习题推荐模块。在习题知识点标注模块中,一种结合图神经网络和基于注意力机制的双向LSTM模块的新型网络架构被提出来用于习题知识点抽取,通过与现有的文本分类模型进行对比,在抽取隐藏知识点和多知识点分类指标上都取得了更强的性能。在知识追踪部分,结合图神经网络的网络传播特性,对DKVMN知识追踪模型进行了改进,并引入学生的个性化答题特征,提出了GKVMN模型。经过与baseline模型对比,提出的模型在多个性能指标上超过了其他模型。在习题推荐模块中,一种基于matching和ranking两阶段的推荐系统模型被提出,它在matching阶段结合多种matching策略来生成候选推荐习题集,在ranking阶段结合知识追踪模块生成的学生知识状态熟练度来为生成的习题排序。通过与传统的推荐算法进行对比,并设计对比指标,结果显示了提出的推荐系统模型具备自适应学习的能力。
    Currently, in high school mathematics education, there are problems such as difficulty in evaluating students' knowledge mastery and difficulty in recommending suitable exercises for practicing. High school mathematics exercise recommendation system is an adaptive learning system which is applied to high school mathematics to recommend students with the most suitable exercises by tracing and assessing their knowledge mastery level.

    Based on this research background, a knowledge-tracing-based exercise recommendation system is designed and proposed, consisting of three modules: the exercise knowledge point labelling module as the pre-processing part of the exercise recommendation set, the knowledge tracing module as the core part of the system to perform real-time tracking of students' knowledge mastery, and the exercise recommendation model as the functional module to generate recommended exercises based on the evaluation of the knowledge proficiency of students.

    In the exercise knowledge point labeling module, a dedicated network combining graph neural network and attention-based bidirectional LSTM layer is proposed for knowledge point labelling of exercise, which achieves better performance in classifying hidden knowledge points and knowledge point with complicated relations in experiments comparing with several baseline models. In the knowledge tracing module, an improved model for the DKVMN knowledge tracing network is proposed, combining the node information propagation mechanism of graph neural networks and the memory mechanism of original DKVMN model. Besides, the model takes new features such as students' answering behaviors as the input of the model, which improves the predictive performance of the model. In the experiment comparing with the baseline model, the proposed model outperforms other baseline models in given metrics. In the exercise recommendation module, a recommendation model based on matching and ranking is proposed. The model uses different screening strategies to generate candidate recommendation sets, and then uses the knowledge mastery proficiency output from the knowledge tracking model to rank recommended exercises. Several experiments are performed to validate the effectiveness of the model.
\end{abstract}
