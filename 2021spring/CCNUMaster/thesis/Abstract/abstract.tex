% ************************** Thesis Abstract *****************************
% Use `abstract' as an option in the document class to print only the titlepage and the abstract.

%Please list 3-5 keywords, and replace them with "keyword1", "keyword2", "keyword3",...
\begin{abstract}{Graph Neural Network}{Knowledge Point Labeling}{knowledge tracing}{Recommended Exercises}{}
    After 2010, artificial intelligence technology has gradually become a research hotspot in the field of computer technology. In particular, the advent of AlphaGo has aroused great concern in the industry for the prospect of artificial intelligence. This has brought about the explosive development of the industry, and also raised a large number of research topics. In artificial intelligence-related research, various algorithm innovations, theoretical breakthroughs and model applications are emerging one after another, laying the foundation for the intelligence of various industries. The application of artificial intelligence technology in the field of education has also given birth to the emergence of the concept of intelligent education. Among them, adaptive learning is one of the popular application fields in intelligent education\cite{ma2017adalearn}. Adaptive learning models generally track the learning status of students by combining big data analysis of massive student group learning data and precise data analysis of target student individual data, targeting on personalizing the learning path according to the individual characteristics of the students and the proficiency of knowledge mastery\cite{soltani2019adaptive}. Adaptive learning technology can use automated machine learning algorithms to complete student evaluation and teaching plans that required a lot of manual labor in the past, which can systematically alleviate the current scarcity and uneven distribution of domestic educational resources as well as reduce the burden on education practitioners and students. It also has great development prospects and commercial value. There are more and more artificial intelligence research teams and intelligent education technology companies on the market focusing on the development and application of adaptive learning tools. Some smart educational technology companies have used adaptive learning as the core function or main selling point of their products. Adaptive educational technology can comprehensively analyze students' individual level learning ability, knowledge proficiency and group level popular learning resources, error-prone questions, etc., so that the most suitable learning path and learning resources such as exercises, materials, and knowledge points can be pushed to student. The system automatically adjusts the knowledge focus of the pushed learning resources according to the students' knowledge status to prevent repetitive practice of already mastered knowledge points or lack of practice of unmastered knowledge points. On the one hand, teachers can analyze the knowledge mastery proficiency of the whole class based on the data or visual charts output by the system to create a learning status assessment report for each student and adjust the overall teaching plan in an adaptive manner. On the other hand, students can use the system to analyze their knowledge weaknesses and thus targeted appropriate exercises. Thus, adaptive learning is one of the potentially feasible solutions to the problem of ``automatic assessment of students'' knowledge mastery status and instructional program generation'' in online education.

    The purpose of this paper is to propose a knowledge-tracing-based model for recommending high school mathematics exercises, using the subject of high school mathematics as the main research context. In the subject of high school mathematics, practice exercises are the main means for students to improve their learning ability. However, in the current high school mathematics teaching, teachers or students need to find suitable exercises to practice from a huge library of exercises, which are often too large, highly repetitive and confusingly organized. There are quite a lot of low-quality unlabeled exercises in the exercise bank, which need to be manually labeled with knowledge points. Some students study through excessive exercises tactic, but this is less efficient and often results in repetition of familiar knowledge and avoidance of unfamiliar knowledge. In order to improve the effectiveness of the exercises performed by students and thus enhance the proficiency and comprehensiveness of knowledge acquisition, experienced teaching staff is needed to conduct analysis of students' knowledge status and to select appropriate exercises from the exercise bank for recommendation. The method is uneconomical and inefficient because of its high manual workload, its reliance on expert a priori knowledge, and its inclusion of a large amount of repetitive work. In addition, the traditional exercise recommendation takes the student group as the minimum granularity, but does not recommend for the knowledge mastery proficiency of specific students, which ignores the problem that different students have different learning abilities, so the recommendation is less fine-grained and ineffective for most students. The method is uneconomical and inefficient because of its high manual workload, its reliance on expert a priori knowledge, and its inclusion of a large amount of repetitive work. In addition, the traditional exercise recommendation takes the student group as the minimum granularity, but does not recommend for the knowledge mastery proficiency of specific students, which ignores the problem that different students have different learning abilities, so the recommendation is less fine-grained and ineffective for most students. In order to improve the problems of traditional exercise recommendation methods, knowledge tracing techniques can be applied to track students' learning and thus target automated exercise recommendations. The goal of this thesis is to design an exercise recommendation system based on knowledge point annotation, knowledge tracing and resource recommendation techniques, and thus introduce an intelligent adaptive learning solution in terms of exercise recommendation.

    The proposed recommendation system for high school mathematics exercises consists of three modules, which are the exercise knowledge point annotation module, the knowledge tracing module and the recommendation module. The function of the exercise knowledge point labeling module is to perform knowledge point labeling for exercises without knowledge point labeling, thus replacing the traditional manual knowledge point labeling with an automated form. The knowledge point labeling is the pre-work of the exercise recommendation, and the knowledge labeled exercises can be used as the data source of the exercise recommendation system. The knowledge tracing module calculates students' knowledge proficiency state vector, which is a representation of students' mastery of subject knowledge points, concepts and skills, by tracking students' exercise records. knowledge tracing is the core part of the system. In the final exercise recommendation module, there are two stages of matching and sorting, the former is applied to the exercise database to apply a variety of matching strategies to quickly filter the exercises and generate a collection of recommended candidate exercises, and the latter inputs the collection into the knowledge tracing system in the sorting stage for refined recommendation sorting to generate the final recommendation results.
    \begin{itemize}
        \item Chapter 2 proposes a multi-knowledge point annotation method for high school mathematics exercises based on bidirectional LSTM and graph neural network. The exercise knowledge point annotation module contains two sub-modules: exercise text mining and multi-knowledge point label classification. Since most of the exercises in the exercise database contain only unstructured data such as textual information, this paper focuses on knowledge point extraction by means of exercise text mining. It applies a bidirectional LSTM network with attention mechanism to perform exercise text mining. The exercises are firstly pre-processed by word separation, cleaning, regularization and other pre-processing steps to obtain word sequences while filtering out a large amount of interference of irrelevant information. Next, the word2vec approach is used as the learning word embedding vector instead of the simple one-hot encoding approach, which can prevent the dimensional disaster caused by the sparse input word vector matrix. Moreover, the hidden dependencies between word vectors can also be characterized by embedding learning, which is conducive to the construction of inter-knowledge point dependencies. After that, text information extraction by a bidirectional LSTM model can better solve the problem of long-range dependency of contextual elements in text. In addition, to capture inter-knowledge point dependencies on the classification model, a multi-label knowledge point labeling model based on graph convolutional network (GCN) is proposed in this paper, where each label is represented by a graph embedding of knowledge points, and the label graph is mapped into a set of intrinsically dependent knowledge point classifiers after several rounds of iterative learning. Subsequently, the word vector of the exercise text extracted from the previous sub-network is fed into the set of knowledge point classifiers to derive a multi-knowledge point prediction probability vector, thus realizing the multi-knowledge point labeling labeling task. In the experimental phase, the proposed method in this thesis is compared with a series of baseline models by conducting experiments on a self-made high school mathematics exercise dataset, and a series of multi-label classification metrics are used to compare and evaluate the model performance. The experimental results show that the method achieves more superior performance on the more complex sets of exercises with more complex knowledge point relationships.s method has achieved superior performance on the problem sets with more complex knowledge point relationships.
        \item Chapter 3 proposes a knowledge tracing model based on graph attention network (GAT) and Transformer architecture. The model learns the complex correlations between exercises at the knowledge point level through graph attention networks, and optimizes the traditional knowledge tracing model for the shortcoming of insufficient characterization of complex correlations between knowledge points between exercises. It solves the following problems, (1) The traditional model models knowledge points as mutually independent relationships or simplified probabilistic dependencies, while ignoring the complex graph-like relationships among knowledge points, thus performing poorly for data with complex knowledge point dependencies. (2) The traditional model cannot output the students' knowledge state, but only the probability of correct answer for the next exercise, which makes it difficult to combine with the recommendation model to recommend exercises. The proposed model combines the powerful representational learning capability of graph attention networks for data and graphs in non-Euclidean space and the remote-dependent attention mechanism modeling capability of Transformer model for serialized exercise data, which has better performance in handling longer exercise records and complex exercise datasets with knowledge relationships. In the experimental phase, the performance of the model proposed in this thesis is compared with the baseline model by conducting experiments on several publicly available datasets in the knowledge tracing domain. The experimental results show that the present model achieves better or comparable results in terms of evaluation parameters relative to most models on the publicly available datasets.
        \item Chapter 4 presents a mathematical exercise recommendation model based on 2 phases of Matching-Ranking. The first stage is the matching model, which is a hybrid model based on multiple matching strategies with two processes: multiplex-matching and merging. In the process of multiplex-matching, multiple matching strategies such as collaborative filtering, popularity, user preferences are used to generate the several subset of exercise recommendation candidates separately. Then, in the process of merging, these candidate subsets are merged by weighted ranking to form a final set of exercise recommendation candidates. The second stage is a knowledge-tracing-based recommendation item ranking model, in which each exercise in the set of exercise candidates obtained in the previous stage is input to the knowledge-tracing model proposed in the previous chapter for correctness prediction, and the one with the most error-prone exercises is used as the recommendation item with the highest priority. The model mainly solves the problem that traditional models often require user-initiated ratings for recommendation model construction, and cannot make efficient recommendations based on the user's knowledge mastery proficiency state. After the performance test on the public dataset and the control experiment with the baseline model, the proposed model has a better performance in tracking the students' weak knowledge mastery and can recommend the appropriate exercises based on the mastery proficiency status obtained from the tracking.
    \end{itemize}

    This paper analyzes the requirements of the system, rationalizes the entire recommendation system into multiple modules, and designs different neural network models and algorithms for each module to achieve and optimize the above three modules. It has both algorithm design and experimental verification methods. A certain degree of innovation. After experimental verification, it has better performance than similar models.
\end{abstract}
