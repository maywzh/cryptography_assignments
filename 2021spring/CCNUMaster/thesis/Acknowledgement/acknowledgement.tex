% ************************** Thesis Acknowledgements **************************

\begin{acknowledgments}
    %经过两年的硕士研究生阶段的学习,在论文完成之际,我真诚地向曾经帮助过我的老师、学院领导、辅导员、同学以及朋友们表示诚挚的谢意。

    %首先,我要特别感谢我的导师王志锋教授,他在我完成平常的科研任务的过程中,给予了耐心细致的指导和极大的帮助。在本文的选题、构思、实验仿真和论文撰写过程中,也提出了相当多有益而启发性的建议。最终使我的论文得到了不断的完善。这都多亏了王老师的辛勤指导和教诲。

    %然后,我要感谢学院所有的领导和老师,他们在我的硕士阶段的工作和学习过程中,提出了相当多宝贵的建议,让我的内心得到了锤炼,对此我充满了感激。这对于我今后的职业生涯,无疑是不可多得的宝贵财富。

    %另外还要感谢三年来和我一起走过研究学习生涯的全体同学和朋友们,在学习中我们相互帮助,互相激励和关心。感谢他们的友情和支持。

    After two years of study at the master's level, I sincerely express my sincere gratitude to my teachers, faculty leaders, counselors, classmates, and friends who have helped me on the occasion of completing my dissertation.

    First of all, I would like to give special thanks to my supervisor, Prof.\ Zhifeng Wang, who has given me patient and meticulous guidance and a great help in the process of completing my usual scientific tasks. He also provided many valuable and enlightening suggestions while selecting the topic, conception, experimental simulation, and thesis writing of this paper. Finally, my thesis has been improved continuously. This is all thanks to Mr.\ Wang's diligent guidance and teachings.

    Then, I would like to thank all the college leaders and teachers for their considerable and valuable advice during my work and study at the master's level, which has refined my heart and for which I am full of gratitude. This is undoubtedly an invaluable asset for my future career.

    I would also like to thank all my classmates and friends who have walked with me through my research study career in the past three years, and we have helped each other in our study, motivated and cared for each other. Thanks for their friendship and support!

    This work was supported by the National Natural Science Foundation of China (No.61901165, 61501199), Science and Technology Research Project of Hubei Education Department (No. Q20191406), Hubei Natural Science Foundation (No. 2017CFB683), Hubei Research Center for Educational Informationization Open Funding (No. HRCEI2020F0102), and Self-determined Research Funds of CCNU from the Colleges' Basic Research and Operation of MOE (No. CCNU20ZT010).

\end{acknowledgments}
