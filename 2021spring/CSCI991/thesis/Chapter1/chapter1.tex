%*******************************************************************************
%*********************************** First Chapter *****************************
%*******************************************************************************

\chapter{Introduction}  %Title of the First Chapter

\ifpdf
	\graphicspath{{Chapter1/Figs/Raster/}{Chapter1/Figs/PDF/}{Chapter1/Figs/}}
\else
	\graphicspath{{Chapter1/Figs/Vector/}{Chapter1/Figs/}}
\fi


%********************************** %First Section  **************************************
\section{Research Background and Significance} %Section - 1.1
%
%人工智能的技术研究和商业化应用部署在近年来产生了加速发展的趋势。各种基于人工智能和大数据分析相关算法的部署上线在加速企业、组织数字化,改善产业链结构和提高信息利用效率方面发挥了积极作用。教育行业作为传统的服务行业,也存在相当大的智能化改进的空间。在传统教育模式中,学生往往以班级为基础教育集体,因而在一个班级中,所有的教学活动的粒度也以班级作为基本单位,导致学生所学习的内容与其需求不完全匹配,并且已经掌握的知识点过度练习和未掌握的知识点缺乏练习的情况时有发生。这造成了学生出现厌学、学习焦虑等状况,对学习效率和学习效果造成了较大影响。另一方面,对于教师而言,普遍性的个体知识状态精细监测和评估需要巨大的工作量,因此教师往往只关注一部分学生,导致多数学生的学习情况被忽略。这对于学生的学习积极性和学习状况具有较大的影响。

%而在近年来,我国发布了一系列促进人工智能在教育领域的应用的政策。通过人工智能技术来进行学生学习情况监测和追踪,可以大大改善教育质量和效率,从而促成智慧教育的实现。在智慧教育中,自适应学习是一个已经经过大量实践和验证过的商业模式,相当多的在线教育平台部署了自适应学习系统服务。它运用数据分析、深度学习等技术,分析学生的学习行为数据,自动学生进行知识状态的评估和追踪,并结合学生的潜能、优势等等个性化信息向学生提供学习路径规划和学习资源推荐服务。它能够帮助学生提升学习的有效性和效率,在降低教师教学压力的同时提升教师的教学质量。其中,习题推荐系统是一个自适应学习的模式。习题推荐系统包括学习知识掌握情况建模和基于知识掌握情况的习题推荐两个部分。知识掌握建模往往利用学习者的学习交互记录例如做题记录、测验记录等等来对于学生的内在学习特征进行刻画,实现对于学习者知识掌握状态的动态追踪过程,以此构建针对性的知识强化练习。习题推荐部分则通过分析学生的知识掌握情况,推荐对于学生的知识掌握较弱的部分提高最合适的习题,实现因材施教,提高学习效果和效率。

%目前,高中数学学科存在知识点联系紧密,习题库庞大而混乱,学生不知道如何针对性地提高知识掌握程度等问题。因而本文以高中数学学科为研究背景,目的是提出一种实时追踪学生数学知识掌握情况,并依照学生的知识状态进行习题推荐的系统。该系统分为三个部分,第一个部分对习题进行知识点标注,作为推荐的依据,第二部分是核心部分,即基于学生的做题记录进行学生的知识掌握状态追踪,第三部分结合习题的知识点和学生的知识掌握情况进行习题推荐。该系统可以有效地实现自适应学习的目的。

The technical research and commercial application deployment of artificial intelligence have produced a trend of accelerated development in recent years. The deployment of various algorithms based on artificial intelligence and big data analysis has played an active role in accelerating the digitization of enterprises and organizations, improving the structure of the industrial chain, and increasing the efficiency of information utilization. As a traditional service industry, the education industry also has considerable room for intelligent improvement. In the traditional education model, students often take the class as the basic education collective. Therefore, in a class, the granularity of all teaching activities is also based on the class as the basic unit, which leads to the fact that the content of the students' learning does not completely match their needs, and what they have already mastered Excessive practice of knowledge points and lack of practice of unmastered knowledge points occur from time to time. This has caused students to be tired of learning, learning anxiety, etc., which has a greater impact on learning efficiency and learning effects. On the other hand, for teachers, generalized detailed monitoring and evaluation of individual knowledge status requires a huge workload, so teachers often only pay attention to a part of the students, leading to the neglect of the learning situation of most students. This has a greater impact on students' learning enthusiasm and learning conditions.

In recent years, China has issued a series of policies to promote the application of artificial intelligence in education. Using artificial intelligence technology to monitor and track student learning can greatly improve the quality and efficiency of education, thereby contributing to the realization of smart education. In smart education, adaptive learning is a business model that has undergone a lot of practice and verification. Quite a number of online education platforms have deployed adaptive learning system services. It uses data analysis, deep learning and other technologies to analyze students' learning behavior data, automatically evaluate and track students' knowledge status, and combine students' potential, advantages and other personalized information to provide students with learning path planning and learning resource recommendation services . It can help students improve the effectiveness and efficiency of learning, and improve the teaching quality of teachers while reducing teachers' teaching pressure. Among them, the exercise recommendation system is an adaptive learning model. The exercise recommendation system includes two parts: modeling of learning knowledge mastery and exercise recommendation based on knowledge mastery. Knowledge mastering modeling often uses learners' interactive learning records, such as question records, test records, etc., to characterize the students' internal learning characteristics, and to realize the dynamic tracking process of the learner's knowledge mastery status, so as to build targeted knowledge Intensive exercises. The exercise recommendation part analyzes the students' knowledge mastery, recommends the most suitable exercises for the parts with weaker knowledge mastery of the students, so as to realize teaching students in accordance with their aptitude and improve the learning effect and efficiency.

At present, high school mathematics subjects are closely connected with knowledge points, the question bank is huge and chaotic, and students do not know how to improve their knowledge mastery in a targeted manner. Therefore, this article takes the high school mathematics subject as the research background, and the purpose is to propose a real-time tracking student's mathematics knowledge mastery, and according to the student's state of knowledge for exercise recommendation system. The system is divided into three parts. The first part is to mark the knowledge points of the exercises as a basis for recommendation. The second part is the core part, which is to track the students' knowledge mastery status based on the students' record of the exercises. The third part combines The knowledge points of the exercises and the students' knowledge mastery are recommended for exercises. The system can effectively achieve the purpose of adaptive learning.

%********************************** %Second Section  *************************************
\section{Research Status}
%本文的研究课题为高中数学习题推荐系统,第一个部分为基于图神经网络和自然语言处理的习题知识点标注,第二个部分为基于图注意力网络与Transformer的知识追踪模型,第三个部分为基于协同过滤和神经网络的习题推荐模块, 的知识跟踪与推荐系统等。因而本文涉及的技术包括高中数学图神经网络,自然语言处理,多标签分类,推荐系统等研究课题, 接下来,本节将回顾这些技术的研究现状。

The research topic of this article is a high school math learning question recommendation system. The first part is the labeling of exercise knowledge points based on graph neural network and natural language processing. The second part is the knowledge tracing model based on graph attention network and Transformer. The third part is the knowledge tracing model based on graph attention network and Transformer. Part of it is the exercise recommendation module based on collaborative filtering and neural network, the knowledge tracing and recommendation system, etc. Therefore, the technologies involved in this article include high school mathematical graph neural networks, natural language processing, multi-label classification, recommendation systems and other research topics. Next, this section will review the current research status of these technologies.

\subsection{High school Mathematics}
%无论在基础教育还是科学研究中,数学都是一门必要的的学科。数学是一门专门研究数量与空间形式之间关系的科学,其符号系统更加完整,公式结构清晰独特,文字和图像等语言表达方式也更加生动直观。在数学学科中,知识结构和认知结构的建立对于该学科的学习起着相当重要的作用。``认知结构''表示在人大脑当中陈述性知识的组织形式,而经过学习内化后的认知结构通过网络结构或图形展示出来的就是知识结构\cite{tanujaya2017relationship}。学习者所需要学习的知识内容大多来源于前人在实践活动中的经验总结,其学习的过程就是对这些归纳好的知识进行认知学习,并不断对知识的结构进行消化、调整、重组,从而构建更为完善和适合自己的知识结构的过程,也是一个与创新思维结合的过程。在数学学科中,知识的掌握往往来自于实践,例如习题练习、证明推导等等。数学学科作为中学阶段的主要科目,具有高度抽象性、严密逻辑性与广泛应用性等特点。该学科知识体系是利用众多的抽象性知识概念建立起来的,并借助对这些概念知识进行学习与思维拓展,形成新的抽象性概念知识。此外,数学逻辑性十分严密,因为数学学科当中得出的任何结论,都需要经过严密的逻辑推理与严格的证明才能被认为是合理的。它是我们参与社会实践活动或科学研究的重要手段和工具,在各行各业以及社会的各个领域当中都离不开对数学的学习。数学知识点之间也具有内在关联性,因而它们可以在具体学习过程中根据特定的逻辑顺序进行安排和学习。这些内在关联关系可以分为同义关系、前驱关系、后继关系、包含关系、兄弟关系、对立关系等几种。通过对知识点进行分析,可以建立知识点关联网络,便于进行后续的知识状态追踪。

Mathematics is a vital subject both in basic education and in scientific research, and a science specializing in the study of the relationship between quantity and spatial form. The mathematical notation system is more complete, the formula structure is clear and unique, and the language expression methods such as text and images are more vivid and intuitive. In mathematics, the establishment of knowledge structure and cognitive structure plays a very important role in the study of the subject. "Cognitive structure" means the organization of declarative knowledge in the human brain, and the cognitive structure after learning is internalized through network structure or graphics is the knowledge structure\cite{tanujaya2017relationship}. Most of the knowledge content that learners need to learn comes from the experience summaries of predecessors in practical activities. The learning process is to conduct cognitive learning on these summarized knowledge, and continue to digest, adjust, and reorganize the structure of knowledge. Thus, the process of constructing a more complete and suitable knowledge structure is also a process of combining with innovative thinking. In mathematics, the mastery of knowledge often comes from practice, such as exercises, proof derivation and so on. As the main subject in the middle school stage, mathematics has the characteristics of high abstraction, rigorous logic and wide application. The subject knowledge system is established by using many abstract knowledge concepts, and with the help of learning and thinking expansion of these conceptual knowledge, new abstract conceptual knowledge is formed. In addition, the logic of mathematics is very rigorous, because any conclusion drawn in mathematics must undergo rigorous logical reasoning and rigorous proofs before it can be considered reasonable. It is an important means and tool for us to participate in social practice activities or scientific research. It is inseparable from the study of mathematics in all walks of life and all fields of society. Mathematical knowledge points are also inherently related, so they can be arranged and learned according to a specific logical sequence in the specific learning process. These internal relations can be divided into synonymous relations, predecessor relations, successor relations, containment relations, brother relations, antagonistic relations and so on\cite{yang2019knowledgerelation}. Through the analysis of knowledge points, a knowledge point association network can be established, which is convenient for the following knowledge status tracking.

\subsection{Graph Neural Network}
%近年来,神经网络模型的高速发展推动了机器学习相关的研究,各种针对各类应用环境和任务的神经网络范式被设计出来,例如广泛应用于图像识别领域的卷积神经网络范式(CNNs)以及应用于自然语言处理任务的循环神经网络范式(RNNs)。传统神经网络对于欧几里得空间中的数据例如语言、序列、图片等具有较好的计算和处理能力,但对于非欧几里得空间例如知识网络则具有一定的局限性。本文中的研究对象为学科知识,知识点之间形成一个网状的关联关系,因此用传统的神经网络模型处理无法完全表征知识点间的复杂关系,以图来学习知识点网络则更加匹配。每个知识点或者习题可以作为图的节点,节点之间的边则代表知识点或习题之间的关联关系。因此本文引入了图神经网络来捕捉知识点和习题之间的关联性,可以在更高的维度对数据进行合理抽象,取得较好的效果。

%目前对于图结构,可以进行的机器学习任务可以粗略分为以下几种:
%(1)图节点分类任务:对于一个每个节点都有对应的特征的图,且部分节点的类别已知,可以设计分类任务针对未知节点进行分类。(2)图边结构预测任务:对于一个部分节点之间的边的关系已知的图,根据已有的信息来挖掘位置的边结构和关系,这类任务就是对边的预测任务,也就是对节点和节点之间关系的预测。(3)图的分类:对于整个图来说,我们也可以对图分类,图分类又称为图的同构问题,这往往通过聚合图的节点特征然后进行分类来实现。

In recent years, the rapid development of neural network models has promoted research related to machine learning. Various neural network paradigms for various application environments and tasks have been designed, such as convolutional neural network paradigms (CNNs), which are widely used in the field of image recognition. And recurrent neural network paradigms (RNNs) applied to natural language processing tasks. Traditional neural networks have good calculation and processing capabilities for data in Euclidean space, such as languages, sequences, pictures, etc., but have certain limitations for non-Euclidean spaces such as knowledge networks. The research object in this article is subject knowledge, and the knowledge points form a net-like relationship. Therefore, the traditional neural network model can not fully characterize the complex relationship between the knowledge points, and the knowledge point network is more matched by learning the knowledge point network with a graph. Each knowledge point or exercise can be used as a node in the graph, and the edges between nodes represent the relationship between the knowledge points or exercises. Therefore, this paper introduces a graph neural network to capture the correlation between knowledge points and exercises, which can reasonably abstract data in a higher dimension and achieve better results.

At present, for the graph structure, the machine learning tasks that can be performed can be roughly divided into the following types:
\begin{enumerate}
	\item Graph node classification task: For a graph in which each node has a corresponding feature, and the categories of some nodes are known, a classification task can be designed to classify unknown nodes.
	\item Graph edge structure prediction task: For a graph where the edge relationship between some nodes is known, the edge structure and relationship of the location are mined based on the existing information. This type of task is the edge prediction task, that is, Prediction of the relationship between nodes and nodes.
	\item Graph classification: For the entire graph, we can also classify the graph. Graph classification is also called the graph isomorphism problem. This is often achieved by aggregating the node characteristics of the graph and then classifying it.
\end{enumerate}

%图神经网络在2009年由Franco等人提出,该模型基于不动点理论,目的是获得每个节点的隐藏状态。但当堆叠次数过多时,该状态收敛容易过于平滑而导致难以学习图的特征信息。图神经网络一般以迭代的方式来对节点进行数据传递计算,以此学习目标节点表示。图神经网络的概念和应用经历了不断的发展,接连有新的图神经网络模型被提出来。近年来随着GPU等并行计算设备算力的提高,使得许多过于由于计算能力限制而无法应用的图神经网络模型有了实践应用的可能性。在本文中,主要用到了图卷积神经网络(GCN)和图注意力神经网络(GAT)等。GCN是Kipf于2016年提出的用于半监督分类的模型\cite{kipf2016semi},GCN的计算是基于层级结构,每一层都是上一层的特征抽取的结果,从节点层面,节点之间互相传播隐藏状态。最终GCN输出经过多层抽象的结果,在对节点特征信息学习和图结构信息的学习方面,在大多数公开节点分类或边预测相关的数据集上GCN几乎均取得了State of the Art(SOTA)的结果。GAT是由Veličković等人在2018年提出\cite{veličković2018graph},该网络在传播过程引入自注意力机制,每个节点的隐藏状态通过注意其邻居节点来计算。该网络采用了局部网络的设计结构,因此在计算过程中只计算相邻节点,降低了计算负载。

The graph neural network was proposed by Franco et al. in 2009. The model is based on the fixed point theory and aims to obtain the hidden state of each node. However, when the number of stacking times is too large, the state will easily converge too smoothly, which makes it difficult to learn the feature information of the graph. Graph neural networks generally perform data transfer calculations on nodes in an iterative manner to represent the learning target node. The concept and application of graph neural network have undergone continuous development, and new graph neural network models have been proposed successively. In recent years, with the increase in computing power of parallel computing devices such as GPUs, many graph neural network models that cannot be applied due to computing power limitations have become practically applicable. In this article, the graph convolutional neural network (GCN) and graph attention neural network (GAT) are mainly used. GCN is a semi-supervised classification model proposed by Kipf in 2016\cite{kipf2016semi}. The calculation of GCN is based on a hierarchical structure. Each layer is the result of feature extraction from the previous layer. From the node level, between nodes Propagate the hidden state to each other. The final GCN output is the result of multi-layer abstraction. In terms of learning node feature information and graph structure information, GCN has almost achieved State of the Art (SOTA) on most public node classification or edge prediction related data sets. the result of. GAT is proposed by Veličković et al. in 2018\cite{veli2018graph}. The network introduces a self-attention mechanism in the propagation process, and the hidden state of each node is calculated by paying attention to its neighbor nodes. The network adopts a local network design structure, so only adjacent nodes are calculated in the calculation process, which reduces the calculation load.

\subsection{Knowledge tracing algorithms}
%知识跟踪(KT)是一项可根据学生过去的答题记录与结果对学生的知识掌握程度进行建模的技术,它是模拟学习者知识掌握情况的一个典型模型,目前已经发展成为智能辅导系统中对学习者知识掌握情况建模的主流方法。知识追踪的主要任务是根据学习者的的历史学习记录来分析学习者的知识掌握情况,从而自动跟踪学生的知识水平随时间的变化,以便能够准确地预测学生在未来学习中的表现并提供适当的学习辅导。 在此过程中,知识空间用于描述学生知识获取的水平。 在知识追踪的过程中,知识空间被建模为概念的集合,学生对概念集合的一部分的掌握构成了学生对知识的掌握。 一些教育研究人员认为,学生对一组特定的相关知识点的掌握会影响他们在练习中的表现,即学生已掌握的知识集与他们在练习上的外部表现密切相关。教师可以通过评估学生的知识状态来更好地了解学生知识掌握的薄弱之处,针对性地对教学方案进行调整。

%在知识追踪模型中,传统的方式是通过认知诊断的方式来实现,它是一种对学生进行知识点熟练度建模的算法。其中,Item response theory(IRT)\cite{embretson2013item}基于一维连续模型对学习者进行建模,而DINA模型则基于一系列表示用户对习题相关知识掌握的向量来对学生进行知识状态建模\cite{de2009dina}。除此之外,贝叶斯知识追踪(BKT)是一种应用较为广泛的基于概率图模型的知识追踪模型\cite{yudelson2013individualized},它将学习者的知识状态建模为一组二进制变量,代表是否掌握知识点,该算法利用隐马尔可夫模型(HMM)来维护知识熟练度变量。但BKT的缺陷在于忽略了学生对于知识的遗忘特性和知识点之间的关联性。在2015年,深度知识追踪(DKT)被提出来\cite{piech2015deep},它是首个将RNN模型应用于知识追踪任务的算法,该模型利用LSTM来追踪学生的知识熟练度随时间动态变化的过程,并学习出学生的知识掌握向量,利用该向量预测学生的做题表现。而2017年提出的Dynamic Key-Value Memory Networks for Knowledge Tracing(DKVMN)模型借用了记忆增强网络的思想,可以利用基础概念之间的关系,用一个键矩阵来存储学生对于知识概念的掌握水平,因此该模型可以显式地输出学生对于知识点的掌握成程度。该模型细化了习题与知识点之间的关系,在公开数据集上也取得了比DKT和BKT更优秀的表现。但是,这些模型都未将知识点的复杂关联性进行合适的建模,因此对于知识点关联复杂的习题,会出现预测性能下降的情况。而图神经网络作为一种适配非欧氏空间建模的模型,可以作为对知识点关系网建模的一种解决方案。在ICLR2019提出的Graph-based Knowledge tracing模型,利用图节点来对学生的回答的习题建模,考虑了做题对于知识状态以及相似习题的影响,在公开数据集的测试结果表名该模型的表现超过了DKVMN。

Knowledge Tracing (KT) is a technology that can model students' knowledge mastery based on their past answer records and results. It is a typical model that simulates learners' knowledge mastery. It has been developed into an intelligent tutoring system. The mainstream method of modeling the knowledge mastery of learners. The main task of knowledge tracing is to analyze the learner's knowledge mastery based on the learner's historical learning records, so as to automatically track the changes in the student's knowledge level over time, so as to accurately predict the student's performance in future learning and provide appropriate information. Study counseling. In this process, the knowledge space is used to describe the level of students' knowledge acquisition. In the process of knowledge tracing, the knowledge space is modeled as a collection of concepts, and students' mastery of a part of the concept collection constitutes the students' mastery of knowledge. Some education researchers believe that students' mastery of a specific set of relevant knowledge points will affect their performance in practice, that is, the knowledge set that students have mastered is closely related to their external performance in practice. Teachers can better understand the weaknesses of students' knowledge by assessing students' knowledge status, and make targeted adjustments to teaching plans

Referring to the knowledge tracing model, the traditional way is realized through cognitive diagnosis, which is an algorithm for modeling the proficiency of students' knowledge points. Among them, Item response theory (IRT)\cite{embretson2013item} is based on a one-dimensional continuous model to model learners, while the DINA model is based on a series of vectors representing the user's knowledge of exercises to model students' knowledge status\cite{de2009dina}. In addition, Bayesian Knowledge Tracing (BKT) is a widely used knowledge tracing model based on a probabilistic graph model \cite{yudelson2013individualized}, which models the learner's knowledge state as a set of binary variables, representing Whether to master knowledge points, the algorithm uses Hidden Markov Model (HMM) to maintain knowledge proficiency variables. But the defect of BKT is that it ignores the connection between students' forgetfulness of knowledge and knowledge points. In 2015, Deep Knowledge Tracing (DKT) was proposed as \cite{piech2015deep}. It is the first algorithm to apply the RNN model to knowledge tracing tasks. The model uses LSTM to track the dynamic changes of students' knowledge proficiency over time Process, and learn the vector of the students' knowledge mastery, and use the vector to predict the performance of the students. The Dynamic Key-Value Memory Networks for Knowledge Tracing (DKVMN)\cite{zhang2017dynamic} model proposed in 2017 borrows the idea of Memory Augmented Neural Network(MANN)\cite{santoro2016meta}. It can use the relationship between basic concepts and use a key matrix to store students' mastery of knowledge concepts. Therefore, The model can explicitly output the degree of students' mastery of knowledge points. The model refines the relationship between exercises and knowledge points, and has also achieved better performance than DKT and BKT on public data sets. However, none of these models properly model the complex relevance of knowledge points. Therefore, for exercises with complex relevance of knowledge points, the prediction performance will decrease. The graph neural network, as a model adapted to non-Euclidean space modeling, can be used as a solution for modeling the knowledge point relationship network. The Graph-based Knowledge tracing model (GKT)\cite{nakagawa2019graphbased} proposed in ICLR2019 uses graph nodes to model students' answers to the exercises, and considers the impact of the exercises on the knowledge state and similar exercises. The test results of the public data set name the performance of the model Exceeded DKVMN.


\subsection{Recommendation System}
%对于广大的学生来说,现有的习题库过于庞大,因此学生在进行课外练习的过程中往往会产生信息迷航的现象。采取题海战术的学习效率低,成效慢。因此一个基于学生的知识状态进行自适应习题推荐的系统是必要和重要的。目前推荐系统已经在互联网上得到了广泛的应用,无论给服务提供方还是给用户都带来了巨大的收益。目前推荐系统算法

For the majority of students, the existing exercise database is too large, so students often experience information tragedy in the process of extracurricular exercises. The learning efficiency of adopting the sea tactics is low and the results are slow. Therefore, a system for adaptive exercise recommendation based on students' knowledge status is necessary and important. At present, the recommendation system has been widely used on the Internet, and it has brought huge benefits to both service providers and users.

%********************************** % Third Section  *************************************
\section{Research Objectives and Content}  %Section - 1.3
%本研究的目的是提出一种基于知识追踪的的高中习题推荐系统,该系统分为三个部分,第一个部分为习题知识点标注部分,该部分的目的是为众多未标注的习题进行知识点标签分类,该模型为一个多知识点标签分类问题,在本节需要对中文数学习题进行文本挖掘,建立知识点标注模型,并对模型进行性能测试。第二个部分,将标注知识点的习题用作知识追踪模型的输入,该部分通过图神经网络进行习题知识图嵌入表示学习,设计一个基于Encoder-Decoder的知识追踪模型来对学生的知识状态进行跟踪并预测下一道习题的正确概率,同时该模型输出隐藏知识状态向量,用在后面的推荐部分作为下一级输入。最后一个部分,结合协同过滤和神经网络模型,设计基于召回-排序两个阶段的推荐模型,按照优先级排序输出一个习题的推荐列表。

The purpose of this research is to propose a high school exercise recommendation system based on knowledge tracking. The system is divided into three parts. The first part is the labeling part of exercise knowledge points. The purpose of this part is to provide knowledge for many unlabeled exercises. Point label classification. This model is a multi-knowledge point label classification problem. In this section, it is necessary to conduct text mining on Chinese mathematics learning questions, establish a knowledge point labeling model, and test the performance of the model. In the second part, the exercises with annotated knowledge points are used as the input of the knowledge tracking model. This part uses the graph neural network to carry out the knowledge graph embedding representation learning of the exercises, and design a knowledge tracking model based on Encoder-Decoder to carry out the knowledge state of the students Track and predict the correct probability of the next exercise. At the same time, the model outputs the hidden knowledge state vector, which is used as the next-level input in the subsequent recommendation part. The last part combines collaborative filtering and neural network models to design a recommendation model based on the two stages of recall-ranking, and output a recommended list of exercises according to priority.

\section{Thesis Organization and Structure}
%本文的第一章是导论。 介绍了研究的研究背景,当前与行业相关的研究进展和研究重点。然后得出本文的三个核心点:习题知识点标注,知识跟踪和习题推荐。本文的第2章着重于习题知识点标注,在习题资源推荐系统中,需要对解析出习题的知识点,然后根据学生当前的知识掌握情况,针对性地推荐学生掌握不足的知识点相关习题。本章节的算法模型分为习题文本信息抽取和标签标注两个部分,在实验部分通过与若干传统模型进行对比来验证模型的有效性。本文的第3章提出了一个基于图神经网络的知识追踪模型,该模型分为习题-知识点关系嵌入学习、知识状态编码和答题预测解码三个部分。文中先对设计思路和相关技术进行理论介绍,然后在实验部分通过在公开数据集上与基准模型进行对比,验证模型的有效性并评估模型性能。本文的第4章提出了一种基于资源召回和资源排序两个阶段的推荐系统模型,该模型的召回阶段,通过协同过滤算法过滤出学习状况相似学生的相关习题,然后基于习题知识点标注模块所标注的习题知识点向量和知识追踪模块的知识状态隐含表征向量,设计一个神经网络模型来进行习题优先级排序。本文的第5章提出了结论和对于模型各个部分的未来的改进方向。


Chapter 1 of this paper is the introduction. Introduces the research background of the research, the research progress and research focus related to the algorithm used in this article. By analyzing the requirements of the exercise recommendation system, three core points of this article are drawn: learning resource representation, knowledge tracing and resource recommendation.

Chapter 2 of this article focuses on the labeling of exercise knowledge points. In the exercise resource recommendation system, the knowledge points of the exercises need to be parsed, and then based on the students' current knowledge mastery, it is recommended that the students have insufficient knowledge points and related exercises. The algorithm model in this chapter is divided into two parts: text information extraction of exercises and labeling. In the experimental part, the effectiveness of the model is verified by comparing with several traditional models.

Chapter 3 of this article proposes a knowledge tracing model based on graph neural network. The model is divided into three parts: exercise-knowledge point relationship embedding learning, knowledge state encoding and answer prediction decoding. The article first introduces the design ideas and related technologies theoretically, and then compares the benchmark model with the public data set in the experimental part to verify the effectiveness of the model and evaluate the performance of the model.

Chapter 4 of this article proposes a recommendation system model based on the two stages of resource recall and resource ranking. In the recall stage of the model, the relevant exercises of students with similar learning conditions are filtered out through collaborative filtering algorithms, and then the module is labeled based on exercise knowledge points The labeled exercise knowledge point vector and the knowledge state implicit representation vector of the knowledge tracing module are designed to design a neural network model to prioritize exercises.

Chapter 5 of this article puts forward conclusions and directions for future improvements in each part of the model.

% \nomenclature[z-DEM]{DEM}{Discrete Element Method}
% \nomenclature[z-FEM]{FEM}{Finite Element Method}
% \nomenclature[z-PFEM]{PFEM}{Particle Finite Element Method}
% \nomenclature[z-FVM]{FVM}{Finite Volume Method}
% \nomenclature[z-BEM]{BEM}{Boundary Element Method}
% \nomenclature[z-MPM]{MPM}{Material Point Method}
% \nomenclature[z-LBM]{LBM}{Lattice Boltzmann Method}
% \nomenclature[z-MRT]{MRT}{Multi-Relaxation
% 	Time}
% \nomenclature[z-RVE]{RVE}{Representative Elemental Volume}
% \nomenclature[z-GPU]{GPU}{Graphics Processing Unit}
% \nomenclature[z-SH]{SH}{Savage Hutter}
% \nomenclature[z-CFD]{CFD}{Computational Fluid Dynamics}
% \nomenclature[z-LES]{LES}{Large Eddy Simulation}
% \nomenclature[z-FLOP]{FLOP}{Floating Point Operations}
% \nomenclature[z-ALU]{ALU}{Arithmetic Logic Unit}
% \nomenclature[z-FPU]{FPU}{Floating Point Unit}
% \nomenclature[z-SM]{SM}{Streaming Multiprocessors}
% \nomenclature[z-PCI]{PCI}{Peripheral Component Interconnect}
% \nomenclature[z-CK]{CK}{Carman - Kozeny}
% \nomenclature[z-CD]{CD}{Contact Dynamics}
% \nomenclature[z-DNS]{DNS}{Direct Numerical Simulation}
% \nomenclature[z-EFG]{EFG}{Element-Free Galerkin}
% \nomenclature[z-PIC]{PIC}{Particle-in-cell}
% \nomenclature[z-USF]{USF}{Update Stress First}
% \nomenclature[z-USL]{USL}{Update Stress Last}
% \nomenclature[s-crit]{crit}{Critical state}
% \nomenclature[z-DKT]{DKT}{Draft Kiss Tumble}
% \nomenclature[z-PPC]{PPC}{Particles per cell}
