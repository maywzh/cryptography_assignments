% ************************** Thesis Abstract *****************************
% Use `abstract' as an option in the document class to print only the titlepage and the abstract.

%Please list 3-5 keywords, and replace them with "keyword1", "keyword2", "keyword3",...
\begin{abstract}{Graph Neural Network}{Knowledge Point Labeling}{Knowledge Tracking}{Recommended Exercises}{}
    After 2010, artificial intelligence technology has become a research hotspot in computer technology, causing explosive development in related industries. Various algorithm innovations, theoretical breakthroughs and model applications have emerged one after another and have also caused intelligent innovations in various industries. Among them, the application of artificial intelligence technology in the field of education has led to the emergence of the concept of intelligent education. Among them, adaptive learning is one of the more popular application fields in intelligent education\cite{ma2017adalearn}. Adaptive learning tracks students' learning status by combining big data analysis based on massive student group learning data and learning data based on individual target students. It develops a personalized learning plan based on students' characteristics and learning goals\cite{soltani2019adaptive}. This technology can systematically alleviate the current contradiction between the scarcity and uneven distribution of domestic educational resources. It can also reduce the burden on education practitioners and students and has good development prospects and commercial value. More and more artificial intelligence research teams and intelligent education technology companies in the market start to focus on developing and applying adaptive learning tools. Simultaneously, many education companies have begun to use adaptive learning as the main core function or main selling point of their products. Adaptive educational technology can combine and analyze students' knowledge mastery, gaps and individual learning abilities so as to have the most suitable learning path and learning resources such as exercises, materials, and knowledge points to students. The system will automatically adjust the difficulty of pushing learning resources according to the students' knowledge status to prevent the diminished interest in learning caused by over-distress or over-easiness. Combined with manual tutoring, teachers can also analyze each student's knowledge gap according to the learning status evaluation report tracked and calculated by the system and adjust the education and learning plan adaptively to realize the personalized teaching provision program assisted by the system. Therefore, adaptive learning is one of the potential practical solutions to "teaching students based on their mastery and individual characteristics" in online education. This article aims to design an exercise recommendation system that combines knowledge point labelling, knowledge tracking, and resource recommendation technology to propose an adaptive learning program in exercise recommendation.

    This article takes high school mathematics as the primary research object. In high school mathematics, exercises are the primary means to improve learning ability. However, current high school mathematics has problems such as too large exercise database, high repetition and disordered organization, among which there are quite a lot of similar or repeated exercises. In traditional high school mathematics education, some students learn through problem tactics. However, they tend to concentrate on the more proficient knowledge points for much practice, resulting in a little effect. Therefore, experienced teaching staff are required to recommend exercises. This method is often done through manual recommendation, and the teacher pushes them to the students in a class through exercises classification and exercises summary. This method requires much manual work, relies on experts' prior knowledge, and does not make recommendations for students' individualized knowledge mastery. The recommendation effect is plough to improve this kind of situation. The application of knowledge tracking technology can recommend exercises based on the students' knowledge mastery. This article aims to design an exercise recommendation system that combines knowledge point labelling, knowledge tracking, and resource recommendation technology to form an intelligent adaptive learning solution in exercise recommendation.

    The high school math learning question recommendation system proposed in this paper includes three modules: the exercise knowledge point labelling module, the knowledge tracking module, and the recommendation module. The exercise knowledge point mining module's function is to label the knowledge points for the exercises that have not been marked so as to replace the traditional manual labelling with machine automation. Knowledge point labelling is the pre-work of exercise recommendation, and the exercises that have been labelled with knowledge can be used as the exercises of the knowledge tracking system to be embedded in the learning data source. The knowledge tracking module obtains and calculates the student's knowledge state's representation by tracking the student's question record. The knowledge state vector represents the student's knowledge and skill mastery and is the core part of the entire system. In the final exercise recommendation module, input the knowledge vector and exercise embedding representation obtained in the previous period, and combine collaborative filtering and related recommendation models to perform preliminary filtering and fine recommendation of exercises, thereby realizing exercise recommendation based on knowledge status.
    \begin{itemize}
        \item In the knowledge point annotation module, a graph convolutional neural network is used to train the label annotation classifier. Use Bi-LSTM to mine the exercise text information, embed the exercise text as the input of the label classifier group, and output a set of label classification results.
        \item In the knowledge tracking module, a graph attention network is used to learn the knowledge relationship of exercises. The embedding vector is output through the Transformer-based Encoder model to output a knowledge state vector, and then the exercise embedding vector and the knowledge state vector are used as the Transformer-based Decoder The input and output of the model predict the answer to the next exercise.
        \item In the exercise recommendation module, a recall model based on collaborative filtering and a ranking model based on MLP are used. In the recall phase, collaborative filtering is used to find the problem sets of other students with similar knowledge status. In the sorting phase, the knowledge status vector obtained from the knowledge tracking module and the label vector of the exercises are input into the final ranking model, and a priority sequence of exercise recommendation is input.
    \end{itemize}

    This paper implements the above three modules by designing different neural network models and algorithms. It is innovative in algorithm design and experimental verification methods. After experimental verification, it has better performance than similar models.
\end{abstract}
