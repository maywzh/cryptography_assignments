% ************************** Thesis Abstract *****************************
% Use `abstract' as an option in the document class to print only the titlepage and the abstract.

%Please list 3-5 keywords, and replace them with "keyword1", "keyword2", "keyword3",...
\begin{abstractC}{图神经网络}{知识点标注}{知识追踪}{习题推荐}{}
    2010年后,人工智能技术成为了计算机技术领域的研究热点,引起了相关行业的爆发式发展,各种算法创新、理论突破和模型应用层出不穷,也引起了各个行业的智能化革新。其中,人工智能技术在教育领域的应用促成了智能教育概念的出现。其中,自适应学习是智能教育中的较为热门的应用领域之一\cite{ma2017adalearn}。自适应学习通过结合基于海量学生群体学习数据的大数据分析和基于目标学生个体的学习数据来追踪学生的学习状态,并基于学生的个体特征和学习目标来制定个性化的学习计划\cite{soltani2019adaptive}。该技术可以系统性缓解目前国内教育资源稀缺与分配不均的矛盾,也可以减轻教育从业者和学生的负担,具有较好的发展前景和商业价值。市场上也有越来越多的人工智能研究团队和智能化教育技术公司专注于自适应学习工具的研发和应用。同时,许多教育公司已开始将自适应学习用作其产品的主要核心功能或主要卖点。自适应教育技术可以结合分析学生的知识掌握度、差距和个体学习能力,从而具备了将最适合的学习路径和学习资源例如习题、资料、知识点推送给学生的能力。系统会根据学生的知识状态自动调整推送学习资源的难度,防止过难或过易引起的学习兴趣减退。结合人工辅导,教师也可以根据系统跟踪和计算出的学习状态评估报告分析每个学生的知识差距,适应性地调整教育和学习计划,从而实现在系统辅助下的个性化教学提供方案。因此,适应性学习是在线教育中``针对学生掌握情况和个性化特征向学生教学''问题潜在实际的解决方案之一。

    本文的以高中数学学科为主要研究对象。在高中数学学科中,习题练习为主要的学习能力提高手段。但是目前高中数学存在习题库过于庞大、重复度高和组织混乱等问题,其中存在相当多相似或者重复的习题。在传统的高中数学教育中,部分学生通过题海战术来进行学习,但实际情况往往是学生集中于较为熟练的知识点进行大量练习,导致练习收效甚微。为了提高知识点掌握的全面性和提高学习和练习效果,需要经验丰富的教学人员进行习题推荐,该方式往往通过全阶段人工的方式来完成。首先由专业人员完成习题的分类和知识点标注,然后进行习题的编纂、汇总,最后由教师分发给学生进行练习和测验。该方法人工工作量大,依赖专家先验知识,且包含大量的重复性工作,因此存在低效和效果不理想的问题。此外,传统习题推荐以学生群体为单位,没有针对特定学生的知识掌握情况进行推荐,也没有考虑不同学生的学习能力不同的问题,因此导致推荐的效果精细度较差。为了改善传统习题推荐方法存在的问题,可以通过应用知识追踪技术基于学生的知识掌握情况进行习题推荐。而本文的目标在于设计一个基于知识点标注、知识追踪和资源推荐技术的习题推荐系统,从而推出一个在习题推荐方面的智能自适应学习的解决方案。

    本文提出的高中数学习题推荐系统包括三个模块,分别为习题知识点标注模块、知识追踪模块和推荐模块。习题知识点挖掘模块的作用是为未标注知识点的习题进行知识点标注,从而将传统的人工标注以机器自动化的形式代替。知识点标注是习题推荐的前置工作,经过知识标注的习题可以作为知识追踪系统的习题嵌入学习数据源。知识追踪模块通过追踪学生的做题记录,获取和计算学生的知识状态表示,该知识状态向量是学生知识和技能掌握度的表征,是整个系统的核心部分。在最后的习题推荐模块,输入前期获取的知识向量和习题嵌入表示,并结合协同过滤和相关的推荐模型来进行习题初步过滤和精细推荐,从而实现基于知识状态的习题推荐。
    \begin{itemize}
        \item 第二章提出了一种基于双向LSTM与图神经网络的高中数学习题多知识点标注方法,习题知识点标注模块包含习题文本挖掘和知识点标注两个子模块。本文主要通过习题文本信息挖掘的方式来进行知识点提取,因此应用了基于注意力机制的双向LSTM网络来进行习题文本挖掘,习题首先经过分词、清洗、正则化等预处理步骤,得到词序列的同时过滤掉大量的无关信息的干扰。然后是计算从而降低矩阵稀疏带来的影响。通过词嵌入的方式而非简单的one-hot编码方式来作为词嵌入,可以防止输入词向量矩阵过于稀疏带来的维数灾难等问题。另外通过嵌入学习的方式,也可以表征词向量间的隐藏依赖关系,有利于构建知识点间依赖关系。接下来,通过双向LSTM模型进行文本信息抽取,能够更好地利用上下文信息进行文本分类。另外,为了捕捉解决知识点间依赖关系,本文提出了一个基于图卷积网络(GCN)的多标签知识点标注模型,每个标签由知识点的嵌入表示,经过多轮迭代学习,将标签图映射为一组内在依赖的知识点分类器。这些分类器随后应用于前一个子网络提取的习题文本描述,从而实现多知识点标签标注任务。实验阶段,通过在实际的高中数学习题数据集上进行实验,将本论文提出的方法与一系列基准模型以及SOTA模型进行对比,并采用一系列多标签分类指标参数来进行模型性能比较和评估。实验结果显示该方法在知识点关系较为复杂的习题集上取得了更加优越的性能。
        \item 第三章提出了基于图注意力网络和Transformer架构的知识追踪模型。该模型通过图注意力网络来学习习题间的知识点关联关系,针对传统的知识追踪模型对于习题间知识点复杂关联关系表征不足的缺陷进行了优化。它解决了如下问题,(1)传统模型无法输出学生的知识状态,而只能输出对于下一道习题的回答正确概率,从而难以结合推荐模型进行习题推荐。(2)传统模型将知识点建模为相互独立的关系或者层次关系,而忽略了知识点间复杂的图状关系,从而对于知识点依赖关系复杂的数据表现不佳。提出的模型结合了图神经网络的对于非欧式空间的数据的强大表征和学习能力,以及Transformer模型对于序列化习题练习数据的信息编码和解码能力,能够对于具有复杂知识依赖的习题集上的知识追踪任务表现良好。实验阶段,通过在实际的知识追踪公开数据集上进行实验,将本论文提出的模型与基准模型和近期提出的SOTA模型进行性能对比。实验结果显示,在公开数据集上,本模型相对于大多数模型在评估参数上都取得了较优或者相近的结果。
        \item 在习题推荐模块中,采用了基于协同过滤的召回模型与基于MLP的排序模型。召回阶段通过协同过滤找到知识状态相似的其他学生的习题集合,在排序阶段,将知识追踪模块上获取的知识状态向量与习题的标签向量输入到最终的排序模型,输入一个习题推荐优先级序列。
    \end{itemize}

    本文通过设计不同的神经网络模型和算法来实现上述三个模块,在算法设计和实验验证方法方面都具有一定的创新性,经过实验验证,具备相对于同类模型更好的性能。
\end{abstractC}
