% ************************** Thesis Abstract *****************************
% Use `abstract' as an option in the document class to print only the titlepage and the abstract.

%Please list 3-5 keywords, and replace them with "keyword1", "keyword2", "keyword3",...
\begin{abstractC}{图神经网络}{知识点标注}{知识追踪}{习题推荐}{}
    近年来,人工智能行业发展迅速,人工智能技术被应用于各行各业,教育行业也不例外。人工智能也已开始应用于教育行业,并且出现了智能教育的概念。在AI技术在教育中的应用类型中,AI自适应学习是在学习的各个方面中使用最广泛的。此外,由于中国人口众多,教育资源短缺,对教育的重视以及其他有利因素,智能自适应学习系统预计将在以后出现。近年来,家庭适应性学习已开始进入许多参与教育培训和教育投资的人们的思想。市场上有越来越多的教育技术公司专注于自适应学习工具。同时,许多教育公司已开始将自适应学习用作其产品的主要核心功能或主要卖点。适应性教育的最大优势在于,它可以找到每个学生的知识差距。自适应学习平台将引导学生学习适合他的下一个最合适的学习内容和活动。当学生遇到学习过程中太难或太低的课程时,他们可以自动调整课程的难度。教师还可以根据系统提供的学习状态评估报告分析每个学生的知识差距,实时调整学习进度,并为每个学生提供个性化的教学。因此,从理论上讲,适应性学习是在线教育中“根据学生的能力向学生教学”问题的潜在可行解决方案之一。学习资源推荐系统是自适应学习系统的一种最普遍的表现形式。而本文通过设计一个基于知识追踪的习题推荐系统来实现自适应学习的目标。

    本文的自适应学习系统包括三个模块,分别为习题知识点挖掘模块、知识追踪模块和习题推荐模块。习题知识点挖掘模块可以未标注知识点的习题进行知识点标注,并将标注知识点的习题作为知识追踪的输入数据。知识追踪模块追踪学生的做题记录,从而跟踪学生的知识状态。习题推荐模块则结合学生的知识状态和习题标签信息,进行针对性的习题推荐。

    在知识点标注模块中,采用了图卷积神经网络来训练标签标注分类器。通过Bi-LSTM来挖掘习题文本信息,将习题文本嵌入作为标签分类器组的输入,输出一组标签的分类结果。

    在知识追踪模块中,采用了图注意力网络来进行习题知识关系嵌入学习,将嵌入向量通过基于Transformer的Encoder模型来输出一个知识状态向量,再将习题嵌入向量与知识状态向量作为基于Transformer的Decoder模型的输入,输出对于下一道习题的答题情况预测。

    在习题推荐模块中,采用了基于协同过滤的召回模型与基于MLP的排序模型。召回阶段通过协同过滤找到知识状态相似的其他学生的习题集合,在排序阶段,将知识追踪模块上获取的知识状态向量与习题的标签向量输入到最终的排序模型,输入一个习题推荐优先级序列。

    本文通过设计不同的神经网络模型和算法来实现上述三个模块,在算法设计和实验验证方法方面都具有一定的创新性,经过实验验证,具备相对于同类模型更好的性能。
\end{abstractC}
