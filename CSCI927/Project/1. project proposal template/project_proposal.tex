\documentclass[runningheads]{llncs}
\usepackage{graphicx}
\usepackage[text={150mm,220mm},centering,nohead]{geometry}
\pagestyle{empty} 
\begin{document}
\title{\large{CSCI927 Service-Oriented Software Engineering (Project Proposal)}}
\author{}
\institute{}
\maketitle
\vspace{-1cm}
%-----------Please Do NOT change the content above.-----------------

%---------------------------------------------------------------------------------------------------------------------------------

%-----------Please write the project information here.---------------

\begin{center}
\Large{\textbf{Online study lounge system based on SOA}} \\ % Please write your project tile in here
\vspace{0.2cm}
\large{\emph{Group Members (Group number): Liting Lyu (6603324), Yueyue  He (6603671), Muzhe Peng (6603646), Wangzhihui Mei (6603385)}} \\%Please write names of your group members as well as the group number in here
\vspace{0.3cm}
\end{center}

%-----------Please write the content of your research proposal from here.---------------
\section*{Situation}
When students study alone, they encountered many problems, such as distraction, puzzles and lack of sense of achievement. Tradiationally, They use Pomodoro Techniques, GTD software and other tools to help them study and work efficiently. However, a software integrating all these functionalities is required. 

\section*{Design:}
Our goal is to build a software system integrating GTD, Q\&A, record and virtual study room modules, which aims to assist students in their study. 

\noindent We generally divide our system into four modules --- Personal, Study, Social and Q\&A. Each module contains functional services, which are based on some basic service. We list some must-have services and good-to-have services.

\subsection*{Must-have services}
\begin{itemize}
    \item User profile service: user can edit profiles for display. They can also fetch learning data and feed flow.
    \item Calendar service: user can punch to add learning record in electrical card with additional learning time and dynamics. Monthly learning report is available.
    \item Task service: User can add/delete/modify learning task and invoke push service. Periodical report generated by system and task progress are available. 
    \item Network disk service: manage learning materials with online disk.
    \item Self-study room service: User create online study room with permission manage system. Study room can interact with calendar services and Task service.
    \item 
\end{itemize}
\subsection*{Good-to-have services}
\begin{itemize}
    \item 
\end{itemize}

\section*{Techniques}
BPMN CMMN xml 

\clearpage
\begin{flushleft}
    \huge{\textbf{Appendix}}
    \end{flushleft}
    \begin{center}
    \Large{\textbf{Project Title:  XXX }} \\*[0.1cm]%Please write names of the project title in here
    \large{\emph{Group Members (Group number): Liting Lyu (6603324), Yueyue  He (6603671), Muzhe Peng (6603646), Wangzhihui Mei (6603385)}} %Please write names of your group members as well as the group number in here
    \end{center}
    %----------------------------------------------------------------------------------------------------------------------------
    
    
    %-----------Please write the content of your appendix (diagrams, figures, tables, etc) from here.---------------
    \noindent This project...


\end{document}

