\documentclass[12pt]{article}%
\usepackage{amsfonts}
\usepackage{fancyhdr}
\usepackage{comment}
\usepackage[a4paper, top=2.5cm, bottom=2.5cm, left=2.2cm, right=2.2cm]%
{geometry}
\usepackage{times}
\usepackage{amsmath}
\usepackage{changepage}
\usepackage{amssymb}
\usepackage{graphicx}
\usepackage{diagbox}
%
\setcounter{MaxMatrixCols}{30}
\newtheorem{theorem}{Theorem}
\newtheorem{acknowledgement}[theorem]{Acknowledgement}
\newtheorem{algorithm}[theorem]{Algorithm}
\newtheorem{axiom}{Axiom}
\newtheorem{case}[theorem]{Case}
\newtheorem{claim}[theorem]{Claim}
\newtheorem{conclusion}[theorem]{Conclusion}
\newtheorem{condition}[theorem]{Condition}
\newtheorem{conjecture}[theorem]{Conjecture}
\newtheorem{corollary}[theorem]{Corollary}
\newtheorem{criterion}[theorem]{Criterion}
\newtheorem{definition}[theorem]{Definition}
\newtheorem{example}[theorem]{Example}
\newtheorem{exercise}[theorem]{Exercise}
\newtheorem{lemma}[theorem]{Lemma}
\newtheorem{notation}[theorem]{Notation}
\newtheorem{problem}[theorem]{Problem}
\newtheorem{proposition}[theorem]{Proposition}
\newtheorem{remark}[theorem]{Remark}
\newtheorem{solution}[theorem]{Solution}
\newtheorem{summary}[theorem]{Summary}
\newenvironment{proof}[1][Proof]{\textbf{#1.} }{\ \rule{0.5em}{0.5em}}

\newcommand{\Q}{\mathbb{Q}}
\newcommand{\R}{\mathbb{R}}
\newcommand{\C}{\mathbb{C}}
\newcommand{\Z}{\mathbb{Z}}

\begin{document}

\title{Solution 3}
\author{Wangzhihui Mei \\ 2019124044 6603385}
\date{}
\maketitle

\section*{Task 3}
The isolation level of the procedure INSERT\_ORDER Should be set as \textbf{SERIALIZABLE}.
\begin{table}[h]
    \begin{tabular}{l|l}
    T1                                                                                                                                & T2                                                                                                                                \\ \hline
    INSERT INTO ORDERS VALUES(63, 1023, 78, ...);                                                                                     &                                                                                                                                   \\ \hline
                                                                                                                                      & INSERT INTO ORDERS VALUES(64, 1023, 83, ...);                                                                                     \\ \hline
    \begin{tabular}[c]{@{}l@{}}SELECT TOTAL\_ORDERS INTO total \\ FROM CUSTOMER \\ WHERE CUSTOMER.CUSTOMER\_CODE = 1023;\end{tabular} &                                                                                                                                   \\ \hline
    15                                                                                                                                &                                                                                                                                   \\ \hline
                                                                                                                                      & \begin{tabular}[c]{@{}l@{}}SELECT TOTAL\_ORDERS INTO total \\ FROM CUSTOMER \\ WHERE CUSTOMER.CUSTOMER\_CODE = 1023;\end{tabular} \\ \hline
                                                                                                                                      & 15                                                                                                                                \\ \hline
    total = 16;                                                                                                                       &                                                                                                                                   \\ \hline
                                                                                                                                      & total=16;                                                                                                                         \\ \hline
    \begin{tabular}[c]{@{}l@{}}UPDATE CUSTOMER \\ SET TOTAL\_ORDERS = total \\ WHERE CUSTOMER.CUSTOMER\_CODE = 16;\end{tabular}       &                                                                                                                                   \\ \hline
                                                                                                                                      & \begin{tabular}[c]{@{}l@{}}UPDATE CUSTOMER \\ SET TOTAL\_ORDERS = total \\ WHERE CUSTOMER.CUSTOMER\_CODE = 16;\end{tabular}       \\ \hline
    COMMIT;                                                                                                                           &                                                                                                                                   \\ \hline
    \end{tabular}
    \end{table}

We can see from the concurrent transactions $T_1$ and $T_2$, The total orders maybe set as 16 which should be 17. The Serializable level prevent this by commit transactions serializablely, so the next transaction can read the latest to

\end{document}