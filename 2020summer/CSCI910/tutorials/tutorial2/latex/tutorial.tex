\documentclass{article}

\usepackage{fancyhdr}
\usepackage{extramarks}
\usepackage{amsmath}
\usepackage{amsthm}
\usepackage{amsfonts}
\usepackage{tikz}
\usepackage[plain]{algorithm}
\usepackage{algpseudocode}
\usepackage{listings} 
\usetikzlibrary{automata,positioning,positioning,calc}
\tikzset{global scale/.style={
    scale=#1,
    every node/.append style={scale=#1}
  }
}


\usepackage{color}


% Basic Document Settings
%

\topmargin=-0.45in
\evensidemargin=0in
\oddsidemargin=0in
\textwidth=6.5in
\textheight=9.0in
\headsep=0.25in

\linespread{1.1}

\pagestyle{fancy}
\lhead{\hmwkAuthorName}
\chead{\hmwkClass\: \hmwkTitle}
\rhead{\firstxmark}
\lfoot{\lastxmark}
\cfoot{\thepage}

\renewcommand\headrulewidth{0.4pt}
\renewcommand\footrulewidth{0.4pt}

\setlength\parindent{0pt}

%
% Create Problem Sections
%

\newcommand{\enterProblemHeader}[1]{
    \nobreak\extramarks{}{Problem \arabic{#1} continued on next page\ldots}\nobreak{}
    \nobreak\extramarks{Problem \arabic{#1} (continued)}{Problem \arabic{#1} continued on next page\ldots}\nobreak{}
}

\newcommand{\exitProblemHeader}[1]{
    \nobreak\extramarks{Problem \arabic{#1} (continued)}{Problem \arabic{#1} continued on next page\ldots}\nobreak{}
    \stepcounter{#1}
    \nobreak\extramarks{Problem \arabic{#1}}{}\nobreak{}
}

\setcounter{secnumdepth}{0}
\newcounter{partCounter}
\newcounter{homeworkProblemCounter}
\setcounter{homeworkProblemCounter}{1}
\nobreak\extramarks{Problem \arabic{homeworkProblemCounter}}{}\nobreak{}

%
% Homework Problem Environment
%
% This environment takes an optional argument. When given, it will adjust the
% problem counter. This is useful for when the problems given for your
% assignment aren't sequential. See the last 3 problems of this template for an
% example.
%
\newenvironment{homeworkProblem}[1][-1]{
    \ifnum#1>0
        \setcounter{homeworkProblemCounter}{#1}
    \fi
    \section{Problem \arabic{homeworkProblemCounter}}
    \setcounter{partCounter}{1}
    \enterProblemHeader{homeworkProblemCounter}
}{
    \exitProblemHeader{homeworkProblemCounter}
}

%
% Homework Details
%   - Title
%   - Due date
%   - Class
%   - Section/Time
%   - Instructor
%   - Author
%

\newcommand{\hmwkTitle}{Tutorial\ \#2}
\newcommand{\hmwkDueDate}{March 19, 2020}
\newcommand{\hmwkClass}{CSCI910 Software Requirements}
\newcommand{\hmwkClassTime}{}
\newcommand{\hmwkClassInstructor}{Chen Jiageng}
\newcommand{\hmwkAuthorName}{\textbf{Mei Wangzhihui}}
\newcommand{\hmwkAuthorNum}{\textbf{2019124044}}
%
% Title Page
%

\title{
    \vspace{2in}
    \textmd{\textbf{\hmwkClass:\ \hmwkTitle}}\\
    % \normalsize\vspace{0.1in}\small{Due\ on\ \hmwkDueDate\ at 3:10pm}\\
    % \vspace{0.1in}\large{\textit{\hmwkClassInstructor\ \hmwkClassTime}}
    \vspace{3in}
}

\author{\hmwkAuthorName\ \hmwkAuthorNum}
\date{}

\renewcommand{\part}[1]{\textbf{\large Part \Alph{partCounter}}\stepcounter{partCounter}\\}

%
% Various Helper Commands
%

% Useful for algorithms
\newcommand{\alg}[1]{\textsc{\bfseries \footnotesize #1}}

% For derivatives
\newcommand{\deriv}[1]{\frac{\mathrm{d}}{\mathrm{d}x} (#1)}

% For partial derivatives
\newcommand{\pderiv}[2]{\frac{\partial}{\partial #1} (#2)}

% Integral dx
\newcommand{\dx}{\mathrm{d}x}

% Alias for the Solution section header
\newcommand{\solution}{\textbf{\large Solution}}

% Probability commands: Expectation, Variance, Covariance, Bias
\newcommand{\E}{\mathrm{E}}
\newcommand{\Var}{\mathrm{Var}}
\newcommand{\Cov}{\mathrm{Cov}}
\newcommand{\Bias}{\mathrm{Bias}}

\begin{document}

\maketitle

\pagebreak

% Discover ambiguities or omissions in the following statement of requirements for part of a ticket-issuing system:

%An automated ticket machine sells rail tickets. Users select their destination and input a credit card and a personal identification number. The rail ticket is issued and their credit card account charged. When the user presses the start button, a menu display of potential destinations is activated, along with a message to the user to select a destination and the type of ticket required. Once a destination has been selected, the ticket price is displayed and customers are asked to input their credit card. Its validity is checked and the user is then asked to input their personal identifier (PIN). When the credit transaction has been validated, the ticket is issued.

\begin{homeworkProblem}
    \begin{itemize}
        
        \item What will happen if the payment is cancelled or failed?
        \item How can customer cancel their ticket and re-order new ticket?
        \item How did the system treat with wrong information user entered?
        \item How can user order multiple tickets of different destinations?
        \item Can the machine sell the ticket which start different with where it located?
    \end{itemize}
\end{homeworkProblem}

% Suggest how an engineer responsible for drawing up a system requirements specification might keep track of the relationships between functional and non-functional requirements

\begin{homeworkProblem}
    The engineer needs build link between the functional to non-functional requirements, and assign the importance level to them. The functional requirements implement the non-functional requirements that will be followed. Keep tracking the relationships between functional and non-functional requirements.
\end{homeworkProblem}

%Write a set of non-functional requirements for the ticket-issuing system,setting out its expected reliability and response time.
\begin{homeworkProblem}
\begin{enumerate}
    \item The system should response no more than 5000 milliseconds.
    \item The system should provide fail recovery mechanism enabling the refund and order cancellation.
    \item The total down time of the system should be less than 1 hour per year.
    \item The user interface should be friendly and easy to use.
    \item The system should provide convenience for people with disabilities
\end{enumerate}
\end{homeworkProblem}
\end{document}
