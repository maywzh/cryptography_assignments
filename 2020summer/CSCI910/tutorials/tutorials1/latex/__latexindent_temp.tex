\documentclass{article}

\usepackage{fancyhdr}
\usepackage{extramarks}
\usepackage{amsmath}
\usepackage{amsthm}
\usepackage{amsfonts}
\usepackage{tikz}
\usepackage[plain]{algorithm}
\usepackage{algpseudocode}
\usepackage{listings} 
\usetikzlibrary{automata,positioning,positioning,calc}
\tikzset{global scale/.style={
    scale=#1,
    every node/.append style={scale=#1}
  }
}


\usepackage{color}

\definecolor{dkgreen}{rgb}{0,0.6,0}
\definecolor{gray}{rgb}{0.5,0.5,0.5}
\definecolor{mauve}{rgb}{0.58,0,0.82}

\lstset{language=Go,
  basicstyle=\ttfamily\scriptsize,
  keywordstyle=\color{blue}\ttfamily,
  stringstyle=\color{red}\ttfamily,
  commentstyle=\color{green}\ttfamily,
  breaklines,%自动换行
  }
%
% Basic Document Settings
%

\topmargin=-0.45in
\evensidemargin=0in
\oddsidemargin=0in
\textwidth=6.5in
\textheight=9.0in
\headsep=0.25in

\linespread{1.1}

\pagestyle{fancy}
\lhead{\hmwkAuthorName}
\chead{\hmwkClass\: \hmwkTitle}
\rhead{\firstxmark}
\lfoot{\lastxmark}
\cfoot{\thepage}

\renewcommand\headrulewidth{0.4pt}
\renewcommand\footrulewidth{0.4pt}

\setlength\parindent{0pt}

%
% Create Problem Sections
%

\newcommand{\enterProblemHeader}[1]{
    \nobreak\extramarks{}{Problem \arabic{#1} continued on next page\ldots}\nobreak{}
    \nobreak\extramarks{Problem \arabic{#1} (continued)}{Problem \arabic{#1} continued on next page\ldots}\nobreak{}
}

\newcommand{\exitProblemHeader}[1]{
    \nobreak\extramarks{Problem \arabic{#1} (continued)}{Problem \arabic{#1} continued on next page\ldots}\nobreak{}
    \stepcounter{#1}
    \nobreak\extramarks{Problem \arabic{#1}}{}\nobreak{}
}

\setcounter{secnumdepth}{0}
\newcounter{partCounter}
\newcounter{homeworkProblemCounter}
\setcounter{homeworkProblemCounter}{1}
\nobreak\extramarks{Problem \arabic{homeworkProblemCounter}}{}\nobreak{}

%
% Homework Problem Environment
%
% This environment takes an optional argument. When given, it will adjust the
% problem counter. This is useful for when the problems given for your
% assignment aren't sequential. See the last 3 problems of this template for an
% example.
%
\newenvironment{homeworkProblem}[1][-1]{
    \ifnum#1>0
        \setcounter{homeworkProblemCounter}{#1}
    \fi
    \section{Problem \arabic{homeworkProblemCounter}}
    \setcounter{partCounter}{1}
    \enterProblemHeader{homeworkProblemCounter}
}{
    \exitProblemHeader{homeworkProblemCounter}
}

%
% Homework Details
%   - Title
%   - Due date
%   - Class
%   - Section/Time
%   - Instructor
%   - Author
%

\newcommand{\hmwkTitle}{Tutorial\ \#1}
\newcommand{\hmwkDueDate}{March 19, 2020}
\newcommand{\hmwkClass}{CSCI910 Software Requirements}
\newcommand{\hmwkClassTime}{}
\newcommand{\hmwkClassInstructor}{Chen Jiageng}
\newcommand{\hmwkAuthorName}{\textbf{Mei Wangzhihui}}
\newcommand{\hmwkAuthorNum}{\textbf{2019124044}}
%
% Title Page
%

\title{
    \vspace{2in}
    \textmd{\textbf{\hmwkClass:\ \hmwkTitle}}\\
    % \normalsize\vspace{0.1in}\small{Due\ on\ \hmwkDueDate\ at 3:10pm}\\
    % \vspace{0.1in}\large{\textit{\hmwkClassInstructor\ \hmwkClassTime}}
    \vspace{3in}
}

\author{\hmwkAuthorName\ \hmwkAuthorNum}
\date{}

\renewcommand{\part}[1]{\textbf{\large Part \Alph{partCounter}}\stepcounter{partCounter}\\}

%
% Various Helper Commands
%

% Useful for algorithms
\newcommand{\alg}[1]{\textsc{\bfseries \footnotesize #1}}

% For derivatives
\newcommand{\deriv}[1]{\frac{\mathrm{d}}{\mathrm{d}x} (#1)}

% For partial derivatives
\newcommand{\pderiv}[2]{\frac{\partial}{\partial #1} (#2)}

% Integral dx
\newcommand{\dx}{\mathrm{d}x}

% Alias for the Solution section header
\newcommand{\solution}{\textbf{\large Solution}}

% Probability commands: Expectation, Variance, Covariance, Bias
\newcommand{\E}{\mathrm{E}}
\newcommand{\Var}{\mathrm{Var}}
\newcommand{\Cov}{\mathrm{Cov}}
\newcommand{\Bias}{\mathrm{Bias}}

\begin{document}

\maketitle

\pagebreak

\begin{homeworkProblem}
%Based on your own knowledge, discuss why different application types require specialized software engineering techniques to support their design and development.
    \begin{itemize}
        \item The technology of development varies with different software, some use advanced technologies and some use conventional technologies, the management of these development team are different 
        \item The cost of different application are different, some applications have higher budget, so they are more likely to take steady development method, while other applications have lower budget, so they are more likely to take efficient development method.
        \item The delivery of the software are different, some softwares are delivered as a steady version with long-time support (such as Windows) while some other softwares have short lifetime and are updated frequently (such as web apps)
    \end{itemize}
\end{homeworkProblem}

\begin{homeworkProblem}
%Consider the integration and configuration process model shown in Figure “Reuse-oriented software engineering”. Explain why it is essential to repeat the requirements engineering activity in the process
Both user requirements and system requirements are needed in the Requirements specification. The reusable components can be used to correct the requirements modification. Lack of control of the reusable
\end{homeworkProblem}
\begin{homeworkProblem}
%Suggest why it is important to make a distinction between developing the user requirements and developing system requirements in the requirements engineering process.
The user requirements is written for user to understand the requirements so they are usually written by natural plain language rather than rigorous definition. Users are required to understand the application scenario.

The system requirements is precise document of the project defining the detail of the delivery and the indexes of the system. They specify the project in a more precise way than user requirements.
\end{homeworkProblem}
\end{document}
