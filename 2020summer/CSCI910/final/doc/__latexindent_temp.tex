%Copyright 2014 Jean-Philippe Eisenbarth
%This program is free software: you can 
%redistribute it and/or modify it under the terms of the GNU General Public 
%License as published by the Free Software Foundation, either version 3 of the 
%License, or (at your option) any later version.
%This program is distributed in the hope that it will be useful,but WITHOUT ANY 
%WARRANTY; without even the implied warranty of MERCHANTABILITY or FITNESS FOR A 
%PARTICULAR PURPOSE. See the GNU General Public License for more details.
%You should have received a copy of the GNU General Public License along with 
%this program.  If not, see <http://www.gnu.org/licenses/>.

%Based on the code of Yiannis Lazarides
%http://tex.stackexchange.com/questions/42602/software-requirements-specification-with-latex
%http://tex.stackexchange.com/users/963/yiannis-lazarides
%Also based on the template of Karl E. Wiegers
%http://www.se.rit.edu/~emad/teaching/slides/srs_template_sep14.pdf
%http://karlwiegers.com
\documentclass{scrreprt}
\usepackage{listings}
\usepackage{underscore}
\usepackage[bookmarks=true]{hyperref}
\usepackage[utf8]{inputenc}
\usepackage[english]{babel}
\hypersetup{
    bookmarks=false,    % show bookmarks bar?
    pdftitle={Software Requirement Specification},    % title
    pdfauthor={Jean-Philippe Eisenbarth},                     % author
    pdfsubject={TeX and LaTeX},                        % subject of the document
    pdfkeywords={TeX, LaTeX, graphics, images}, % list of keywords
    colorlinks=true,       % false: boxed links; true: colored links
    linkcolor=blue,       % color of internal links
    citecolor=black,       % color of links to bibliography
    filecolor=black,        % color of file links
    urlcolor=purple,        % color of external links
    linktoc=page            % only page is linked
}%
\def\myversion{1.0 }
\date{}
%\title
\usepackage{hyperref}
\begin{document}

\begin{flushright}
    \rule{16cm}{5pt}\vskip1cm
    \begin{bfseries}
        \Huge{SOFTWARE REQUIREMENTS\\ SPECIFICATION}\\
        \vspace{1.9cm}
        for\\
        \vspace{1.9cm}
        Internet banking system\\
        \vspace{1.9cm}
        \LARGE{Version \myversion approved}\\
        \vspace{1.9cm}
        Prepared by Wangzhihui Mei\\
        \vspace{1.9cm}
        CCNU-UOW JI\\
        \vspace{1.9cm}
        \today\\
    \end{bfseries}
\end{flushright}

\tableofcontents


% \chapter*{Revision History}

% \begin{center}
%     \begin{tabular}{|c|c|c|c|}
%         \hline
%         Name & Date & Reason For Changes & Version \\
%         \hline
%         21   & 22   & 23                 & 24      \\
%         \hline
%         31   & 32   & 33                 & 34      \\
%         \hline
%     \end{tabular}
% \end{center}

\chapter{Product Constraints}

\section{The purpose of the product}

The purpose of the Internet Banking System is to ease bank account management and to create a convenient and easy-to-use application for users, trying to perform money transferring. The system is an online system for bank business providing feasible and comfortable user experience for user.

\section{The client, customer and other stakeholder}
The customers are the existing offline bank customer who have bank accounts, either saving account or credit account. They can access their bank accounts via a browser or a mobile App using this system.
Other stakeholders include bank service provider, the local law support and public security department. The bank set up servers 
\section{Users of the product}

\section{Requirements constraints}


\section{Naming conventions and definitions}


\section{Relevant facts}

\section{Assumptions}

\section{Intended Audience and Reading Suggestions}
% $<$Describe the different types of reader that the document is intended for,such as developers, project managers, marketing staff, users, testers, and documentation writers. Describe what the rest of this SRS contains and how it is organized. Suggest a sequence for reading the document, beginning with the overview sections and proceeding through the sections that are most pertinent to each reader type.$>$
The document focus on requirements of the Internet Banking System. Developers and testers should focus on section section 3,4,5 for overall understand of the product requirements. User may focus on system features on section 4, which can make them more feasible using the software.

\section{Project Scope}
% $<$Provide a short description of the software being specified and its purpose,including relevant benefits, objectives, and goals. Relate the software to corporate goals or business strategies. If a separate vision and scope document is available, refer to it rather than duplicating its contents here.$>$
The purpose of the Internet Banking System is to ease bank account management and to create a convenient and easy-to-use application for users, trying to make money transferring. The system is based on client-server architecture. There will be some backend server supporting all business functions and user can access their bank accounts via a browser or a mobile App. Above all, the product can provide a comfortable user experience for users.

% The Internet Banking System offer the feature of accessing bank accounts. In this project, the main focus is the money transferring function. There are several phase of the function: Login and logout of the online bank account, checking account balances, transferring money. There are some constraints of the products, including 



% \section{References}
% $<$List any other documents or Web addresses to which this SRS refers. These may
% include user interface style guides, contracts, standards, system requirements
% specifications, use case documents, or a vision and scope document. Provide
% enough information so that the reader could access a copy of each reference,
% including title, author, version number, date, and source or location.$>$

\section{Product Perspective}
% $<$Describe the context and origin of the product being specified in this SRS.
% For example, state whether this product is a follow-on member of a product
% family, a replacement for certain existing systems, or a new, self-contained
% product. If the SRS defines a component of a larger system, relate the
% requirements of the larger system to the functionality of this software and
% identify interfaces between the two. A simple diagram that shows the major
% components of the overall system, subsystem interconnections, and external
% interfaces can be helpful.$>$
The product is an assistant system for bank service system. The system is base on existing business process and rules. 

It store several following informations:
\begin{itemize}
    \item account number
    \item account password
    \item account balance 
    \item account types
    \item other relative infomation
\end{itemize}

\chapter{Functional requirements}

\section{The score of the product}

\section{Functional and data requirements}

\chapter{Non-functional requirements}
\section{Look and feel requirements}

\section{Usability requirements}

\section{Performance requirements}

\section{Operational requirements}
\section{Maintainability and portability requirements}
\section{Security requirements}
\section{Cultural and political requirements}

\section{Legal requirements}


\chapter{Project issues}
\section{Open issues}
\section{Off-the-shelf solutions}
\section{New problems}
\section{Tasks}
\section{Cutover}
\section{Risk}
\section{Costs}
\section{User documentation}
\section{Waiting room}
\end{document}
