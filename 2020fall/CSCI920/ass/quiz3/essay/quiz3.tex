\documentclass[a4paper,11pt]{article}
\usepackage{harvard}
\usepackage{setspace}
\usepackage{float}
\usepackage[margin=3cm]{geometry}
\usepackage{fontspec}
\usepackage{graphicx}
\setmainfont{Times-Roman}
\onehalfspacing


\begin{document}


\title{The discussion on the construction of Base Transceiver Station}
\author{Wangzhihui Mei \\ 2019124044 \ 6603385
\\ \\
Central China Normal University \& 
University of Wollongong Joint-Institude}


\date{}

\maketitle
\thispagestyle{empty}
\clearpage
A base transceiver station (BTS) is a piece of equipment that facilitates wireless communication between user equipment (UE) and a network. The construction of Base Transceiver Station, which can improve the signal quality, is considered to be built in Noda's neighborhood. However, The Base Transceiver Station may introduce air pollution and do harm to people's health. This comes to a dilemma that people will suffer either Inconvenience of weak signal for the missing of the BTS or the potential harm to their health. Noda, who was in charge of the construction, issued a report stating that his home was also in the community, so his construction proposal was eventually allowed by the state and the BTS study and construction recommendation were implemented. This raises potential ethical issues. The reason for this is that Noda does not represent the rest of the community in making decisions about the construction of the BTS, and in this case he is effectively withholding information from the public, depriving the public of their right to know and their right to propose the construction of facilities for the community. Here, Noda has taken it upon himself to make irresponsible decisions for everyone in the Neighborhood, and his behavior is a very reckless one. This action, in effect, tacitly assumes that everyone's assessment of BTS is that the pros outweigh the cons, which is far from the case. A democratic vote is necessary here, which would reflect the will of the majority of residents.

Noda's proposal is likely to be biased because, by doing so, Noda, as the person in charge of a project, has set an example by voluntarily assuming the health risks associated with the project. By doing so, he is seen by outsiders as the equivalent of making himself the risk taker. In the general perception, as a risk taker, one tends not to act against one's own interests, and therefore tends to consider the hazards of a construction project that has a personal impact on one's own interests in a conservative manner. If the actual risk takers decide to go ahead with the project, decision makers are more likely to think that the stakeholders will agree with the decision to go ahead with the project. However, this is often not the case, and it is possible that the potential health impacts of BTS may be acceptable to Noda, for example, because Noda may not live in the neighborhood or because Noda's home has an advanced air filtration system to eliminate the effects of polluted air. The public's advice is hidden here because, as stakeholders, different people will weigh the risks and benefits in different ways, with some believing that stronger signals are important and others that keeping the air clean and healthy is more important. Leading by example, such an old and effective way, poses a great risk here. Just because Noda is willing to take the risk doesn't mean that others are willing to take it.

The State did act appropriately after Noda's disclosure. The reason of it is that the State did not have general and overall evaluation toward the potential threat and harm of the construction of BTS. The State made the decision that the services can be retained without appropriate survey and investigation. Noda's report should not be the only factor to judge the construction of BTS. It would be reasonable, first, for the state government to scrutinize Noda himself to ensure that his conclusions are neutral and not self-serving. During the construction of the BTS, Noda should also assume the necessary responsibility to comply with the law in all his activities. The state government should further evaluate Noda's service to ensure that it minimizes the impact of potential problems, such as air pollution, and to ensure that it solves signal problems in the area. In addition, the state should make the construction of the BTS public and actively collect feedback from residents to ensure that the construction of the BTS does not run counter to the interests of the majority of the population.

In terms of morals, Noda should do everything except publicly. For him personally, in this process, he was actually pressured by the state government, and his reasonable demands were not responded to by the state government. Instead, the state government went against his wishes to promote the construction of the BTS project. Noda himself was also forced to conduct research on the project and recommend it against his will. Considering that Noda is the person in charge of the project, Noda can actively announce the impact of the project to the public so that residents can fully understand the potential problems and benefits of the project. If the project is opposed by most people because of its risks, then Residents can jointly oppose the construction of the project, and the state government will re-evaluate the project instead of pressure Noda. Or alternatively, Noda can sue the state government to make the incident known to the public in a legal way. For the public, the state government deprived the public of their right to know and allowed the project to start construction in a dark box operation, which harmed the interests of many people and exposed the public's health to risks. This behavior of the state government is actually illegal. Noda can contact the media to expose the incident and use the public to fight it. In another non-extreme way, Noda can submit a more detailed report to make government decision makers aware of the health hazards of BTS to residents. Noda should ask the government to conduct fieldwork in order to have a scientific and detailed assessment of BTS.
\newpage
\setcounter{page}{1} %Counting from this page



\clearpage
%\bibliographystyle{agsm}
%\bibliography{wpref}
\end{document}