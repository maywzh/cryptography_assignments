\documentclass[a4paper,11pt]{article}
\usepackage{harvard}
\usepackage{setspace}
\usepackage{float}
\usepackage[margin=3cm]{geometry}
\usepackage{fontspec}
\usepackage{graphicx}
\setmainfont{Times-Roman}
\onehalfspacing


\begin{document}


\title{}
\author{Wangzhihui Mei \\ 2019124044 \ 6603385
\\ \\
Central China Normal University \& 
University of Wollongong Joint-Institude}


\date{}

\maketitle
\thispagestyle{empty}
\clearpage


\newpage
\setcounter{page}{1} %Counting from this page


\section{Grand Challenges}
Educational robotics is considered as a branch of education science, which suffer from some problems well known. In the latter sections, some grand challenges and critical problems will be identified and discussed.

\subsection{School refuse the application of advanced educational robots}
There are some legislative bodies doing some researchs that the market of  educational purpose robots grow tremendous, whose trend is still continuing\cite{blikstein2013digital}. Although some experts are optimistic about the development of educational robots, the current development of educational robots is still not optimistic. Some educational institutions still lack understanding or refuse to understand educational robotics. In some countries, ministries of education have issued policies to promote the development of educational robots\cite{benitti2012exploring}, and some schools are fleshing out the use of educational robots, but in some regions, educational robots have not yet been deployed in the classroom, and most experimental and hands-on classes on educational robots have not yet been linked to the classroom; they tend to appear in after-school programs or summer camps. Robotics education has become a subject curriculum, especially for primary and secondary schools teachers, equipment, space and activity funding, teaching experience, etc. are very challenging. The reason is that the technical threshold of robotics also implies an industry sales threshold. Because it is an emerging category, consumer awareness is often low, and even channel employees are not aware of it, which requires effort to train the channel and brings new costs. On the other hand, educational robotics product lines often have limited space in offline retail stores, and if there are only 50 spots, the channel often chooses the easiest to sell and most profitable products. Low user awareness, high sales thresholds, and low repurchase rates all hinder the channel's choice. The reason for the cold sales, in addition to the early stage of the market, the lack of user awareness, more expensive and other factors, there is an important reason is the lack of product performance and educational attributes. most of the products in the C-end market have more toy attributes, even if DJI products are excellent, but how to integrate the entertainment and competitive attributes and teaching attributes, how to improve the teaching system is also a problem.


\subsection{Technologies in schools are out-dated}
The promotion of educational technology has been adopted as a long-term development strategy in many countries. However, the use of educational technology in schools is still outdated in many regions. Modern education demands logical deduction, critical thinking, problem solving, creativity and teamwork and communication skills from students\cite{alimisis2012robotics}. Today's students are growing up in a very different environment than previous generations, and to get a foothold in the modern world, students must learn how to think creatively, plan systematically, analyze critically, and work collaboratively to accomplish tasks. However, current edutech applications, including educational robots, in many schools do not support this educational philosophy. In many cases,  is used only as an adjunct to traditional teaching methods. Typical current edutech is not well suited to promote critical thinking, problem-solving skills, creativity, and teamwork and communication skills because they are still very mechanistic and prescriptive. In accordance with the prementioned discussion for the so-called 21st century skills, current societal developments call for a shift in educational technology from technical (or computer) skills towards technological and computational fluency or literacy. For the field of educational robotics it dictates a move from just using it to offer vocational skills for future science, technology, engineering and mathematics workers towards fluency or literacy with robotic technology making its intellectual and manual advantages available for every future citizen. Robotics should be consistent with leadism, providing an immersive training atmosphere and environment for the skills students need to enter society in order to create an educational approach that is more in line with modern needs.

\subsection{Robotics needs wider application field}
Despite all the mentioned applications of educational technology, the reality is that the application area for educational robots is still very narrow\cite{demo2012and}. Educational robots today are focused on supporting the teaching of other subjects that are often more relevant to educational robots, such as robot programming, robot mechanics, and so on. If we want to attract more participants in the educational robotics approach to education, it needs a broader range of applications. There are many possible robotics applications that have the potential to engage students with a wide range of interests. We need to create creative ways to increase the appeal and learning feedback of educational robots. Educational robots can engage more students, and there are many young people who were not interested in educational robots who become interested in them after making effective connections with them (e.g., playing and talking together). Another way is to allow students to customize their robots in their own way, for example by programming them to create their own unique robots, which has been found to greatly enhance the effectiveness of robotics education. Another way is to allow students to customize their robots in their own way, for example by programming them to create their own unique robots, which has been found to greatly enhance the effectiveness of robotics education. We can also engage students in hands-on robotics systems, for example by imitating the robot's movements. This is a very motivating way to educate students and can greatly increase their interest.


\subsection{Shifting from "black box" to "white box"}
Today's educational robots are often black boxes, which means they are often pre-programmed or pre-assembled. Their internal mechanisms are often closed to the outside world, so it is difficult to develop extended applications, which makes educational robots often passive tools\cite{alimisis2013educational}. Behind the black box nature of robotics is the reason that creating and defining robots requires a high level of competence from students. However, perceived difficulties of robotics tasks have been found to be due to deficient design rather than learners' cognitive deficiencies. Whatever is the underlying misconception, the "black box" metaphor is compatible with the traditional educational paradigm of the teacher or of the curriculum book revealing and explaining ready-made ratified and thus unquestioned information. Very differently from this approach, constructivism methodologies require the transition to the design of transparent ("white-box") robots where users can construct and deconstruct objects, can program robots from scratch and have a deep structural access to the artifacts themselves rather than just consume ready-made technological products. However, students often fall onto "plateaus", unable to progress beyond a certain point and find that they cannot construct something very interesting when starting from scratch every time. So, compromises to transparency in the design of robotics kits for learning have been suggested resulting in the so-called "black-and-white-box" perspectives, so that children can engage in meaningful, interesting and challenging constructivist activity through the control of robots and/or their environment.



\subsection{The calls for validation of the impact of robotics}
The current adoption of educational robots is still in a growth phase, but the potential impact of educational robots still needs to be evaluated and proven. This needs to be validated by more experiments\cite{bredenfeld2010robotics}. Without scientific validation and sound evaluation, the popularity of educational robots will likely stop there. Systematic evaluation and validation of educational robots is still lacking. Research on the usefulness of educational robots for each subject and the achievement of learning goals needs to be validated. The goal is to see if the students become more interested in the subjects they are studying or if they grow in their cognitive and social skills.

\clearpage
\bibliographystyle{agsm}
\bibliography{wpref}
\end{document}