\documentclass[a4paper,11pt]{article}
\usepackage{harvard}
\usepackage{setspace}
\usepackage{float}
\usepackage[margin=3cm]{geometry}
\usepackage{fontspec}
\usepackage{graphicx}
\usepackage{indentfirst} 

\setmainfont{Times-Roman}
\onehalfspacing


\begin{document}


\title{Discussion about Information Communication Technologies}
\author{Wangzhihui Mei \\ 2019124044 \ 6603385
\\ \\
Central China Normal University \& 
University of Wollongong Joint-Institude}


\date{}

\maketitle
\thispagestyle{empty}
\clearpage


\newpage
\setcounter{page}{1} %Counting from this page
\setlength{\parindent}{2em} %2em代表首行缩进两个字符
\section{Introduction}

\section{Information Communication Technologies}
\subsection{What is ICT}

Some sources define Information and communication technologies (ICT) as "electronic means of capturing, processing, storing and communicating information" \cite{heeks1999information}. It is based on binary digital information based on computer hardware and software and network. ICT, which is a new concept and a new technical field formed by the convergence of information technology and communication technology. In the past, communication technology and information technology were two completely different categories: communication technology focused on the transmission technology of message dissemination, while information technology focused on the encoding or decoding of information and the way it was transmitted on the communication carrier. As technology evolved, the two technologies slowly became inextricably linked, thus gradually merging into one category.

In the Okinawa Charter for a Global Information Society, published by the Group of Eight in Okinawa at the beginning of the 21st century, it is believed that "ICT is one of the most powerful drivers of social development in the 21st century and will rapidly become an important engine of world economic growth."

Information and communication technology (ICT) has been the most influential technology in human society in the second half of the 20th century, especially since the 1990s. During this period, not only the development, manufacturing and services of ICT have developed rapidly, but also the application of ICT has become more and more widespread. ICT has penetrated into all areas of human economic and social life, promoting economic and social development and the improvement of people's living standards. At the same time, the economic and social development and the improvement of people's living standards have created more demand for ICT, which further promotes the development of ICT.

ICT is used for economic, social and interpersonal communication and interaction. ICT has dramatically changed the way people work, communicate, learn and live. In addition, ICT continues to revolutionize all parts of the human experience with the first computers, and now robots once handled many human tasks. For example, once a computer answered a phone call and directed the call to the appropriate individual response; now robots can not only answer the phone, but also handle the caller's service request more quickly and efficiently.

The importance of ICT for economic development and business growth is so great, in fact, that it has introduced what many have classified as the fourth industrial revolution.

ICT also supports a broad shift in society as individuals are moving from personal, face-to-face interactions to the digital space. This new era is often referred to as the digital age .

However, for all its revolutionary aspects, ICT does not function evenly\cite{ratheeswari2018information}. Simply put, wealthier countries and wealthier individuals enjoy greater access and are thus better positioned to seize the advantages and opportunities that ICT offers. Economic advantages are found in the ICT marketplace as well as in the broader sphere of business and society as a whole.

ICT services are integration services, outsourcing services, disaster recovery services, platform application business services such as information interoperability, application interaction and collaborative processing that are not based on the integrated service platform for enterprise information, and the organic combination of knowledge services and other IT services and applications, including system professional services, knowledge services and other IT services and application business services, provided for government and enterprise customers.



\subsection{Professionalism of ICT practitioner}

ICT is of fundamental importance to many other industries and economies, and also has considerable professional status, which requires professional bodies to develop appropriate standards of knowledge and codes of conduct. ICT's relevant regulatory boards set standards of knowledge and experience for professionals, and develop codes of ethics, conduct and professional practice, and provide professional certification.

In practice, professional boards or certification bodies expect their certified members to be professional and recognized and accepted as such\cite{weckert2013professionalism}. This requires practitioners to have a considerable level of skill and knowledge to provide a high level of service while adhering to professional standards and ethics. However, membership in a professional body is not a prerequisite for ICT practitioners to practice. Therefore some non-certified practitioners lack understanding of their professional obligations, use their level of knowledge indiscriminately, and have insufficient understanding of the code of ethics and professional practice. There is a fairly broad spectrum of ICT practitioners, ranging from those with relevant university degrees and minimum experience through industry certification to those who have acquired skills from on-the-job training and experience. The role of professional bodies is to establish standards that distinguish practitioners from professionals.

In contrast to other basic service industries such as bus drivers, ICT professionals often require a higher level of intellectual expertise\cite{gleason2002ict}, a deeper understanding of intellectual ethics, and the pursuit of ethical norms. They need to be educated to a higher level and to refine their experience and knowledge on the job.

\section{functions of codes of conduct of professional societies}


\section{Conclusion}
\clearpage
\bibliographystyle{agsm}
\bibliography{wpref}
\end{document}