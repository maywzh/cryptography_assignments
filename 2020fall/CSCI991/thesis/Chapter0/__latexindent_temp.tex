%*******************************************************************************
%*********************************** First Chapter *****************************
%*******************************************************************************

\chapter{Introduction}  %Title of the First Chapter

\ifpdf
    \graphicspath{{Chapter0/Figs/Raster/}{Chapter0/Figs/PDF/}{Chapter0/Figs/}}
\else
    \graphicspath{{Chapter0/Figs/Vector/}{Chapter0/Figs/}}
\fi


%********************************** %First Section  **************************************
\section{Research Background and Significance} %Section - 1.1
Artificial intelligence industry has developed rapidly in recent years, it is being commercialized in all aspects, triggering profound changes in various industries, and the future development of artificial intelligence will be the combination of key technologies and industries.\cite{cui2018performance} At present, AI technology has been implemented in many fields such as finance, medical and security, and the application scenarios are becoming more and more abundant. The commercialization of AI has played a positive role in accelerating the digitization of enterprises, improving the structure of the industrial chain, and increasing the efficiency of information utilization. The traditional education industry also tries to use AI technology to help the development of the industry. Every development of AI is accompanied by breakthroughs in research methods, and deep learning is one of the important representatives of the breakthroughs in machine learning technology in recent years. With the continuous extension of human AI research and application fields, AI will usher in more kinds of technology combination applications in the future. Artificial intelligence has also begun to be applied to the education industry, and the concept of intelligent education has emerged. Among the types of applications of AI technology in education, AI adaptive learning is the most widely used in all aspects of learning. In addition, due to China's large population base, the shortage of educational resources, the importance attached to education and other favorable factors intelligent adaptive learning system is expected to come later. 

In recent years, domestic adaptive learning has begun to enter the minds of many people involved in education training and education investment. There are more and more education technology companies in the market that focus on adaptive learning tools. At the same time, many education companies have started to use adaptive learning as the main core function or main selling point of their products. The biggest advantage of adaptive education is that it can locate the knowledge gaps of each student. The adaptive learning platform will guide the student to the next most suitable learning content and activities for him. When students encounter a course that is too difficult or too low in the learning process, they can automatically adjust the difficulty of the course. Teachers can also analyze the knowledge gaps of each student based on the learning status evaluation report provided by the system, adjust the learning progress in real time, and provide personalized teaching for each student. So theoretically, adaptive learning is one of the potentially feasible solutions to the problem of "teaching to students according to their abilities" in online education. To make a practical adaptive learning system, I plan to use knowledge tracing to track students' learning status and use the factorization machine algorithm to calculate the relevance of topics to students to build such a test recommendation system.The current personalized learning resource recommendation system is one way of implementing adaptive learning, which is the subject of this paper.

In this paper, the study focuses on the recommendation of learning resources for the subject of high school mathematics. In this system, there are two aspects in general: on the one hand, scientific and targeted acquisition and tracing of students' knowledge state, and on the other hand, recommendation of personalized learning resources based on students' knowledge mastery state. We use the knowledge tracing algorithm of graph neural network to acquire and track students' knowledge states, and the factorization agent to try to combine the output of graph neural network with prior knowledge for resource recommendation.


%********************************** %Second Section  *************************************
\section{Research Status} %Section - 1.2

The dominant content of the research is knowledge tracing and recommendation system. Some advanced graph neural network algorithm is applied to finish the task. There have been some research advances and related applications in the area of knowledge tracing and factorization. We surveyed some existing knowledge tracing algorithms and applications, and some applications of factorization machine.

\subsection{Property of high school Math}
Disciplines and knowledge are closely related to each other, so that disciplinary knowledge denotes the specific knowledge contained in a particular field of study. Disciplines are referred to in this study only for specific subjects in the field of education, such as mathematics, language, chemistry and so on. The first step is to learn how to make the best use of the knowledge that is available. The knowledge is obtained from practice, so after learning it, it can also be applied to social practice. Scientific knowledge is declarative because it can be expressed in a series of symbols, words and diagrams; it is also procedural because it can be arranged and learned according to a specific logical order in the process of concrete learning.

Mathematics is a science specializing in the study of the relationship between quantities and spatial forms, its symbolic system is more complete, the formula structure is clear and unique, text and images and other expressions of language is also more vivid and intuitive.

The knowledge that learners need to learn mostly comes from the summaries of the experiences of their predecessors in practical activities. The learning process is a process of cognitive learning of the summarized knowledge and continuous digestion, adjustment and reorganization of the knowledge structure, so as to build a more perfect and suitable knowledge structure, as well as a process of integration with innovative thinking. Thus a good cognitive structure can promote the formation of knowledge structure, and a good knowledge structure can enrich the organization form of cognitive structure. Since the disciplinary knowledge structure consists of two parts: knowledge composition and knowledge dependency, we will analyze the disciplinary knowledge structure from these two aspects, knowledge structure and composition.

The knowledge that learners need to learn mostly comes from the summaries of the experiences of their predecessors in practical activities. The learning process is a process of cognitive learning of the summarized knowledge and continuous digestion, adjustment and reorganization of the knowledge structure, so as to build a more perfect and suitable knowledge structure, as well as a process of integration with innovative thinking. So a good cognitive structure can promote the formation of knowledge structure, and a good knowledge structure can enrich the organization form of cognitive structure. Since the disciplinary knowledge structure consists of two parts: knowledge composition and knowledge dependency, we will analyze the disciplinary knowledge structure from these two aspects.

Knowledge composition refers to the organization of knowledge within a subject area, which mainly includes knowledge points, knowledge blocks and knowledge systems. 
\begin{itemize}
	\item knowledge point: A point of knowledge is the smallest constituent unit of the knowledge structure of a discipline and is used to represent specific concepts. 
	\item knowledge block: A knowledge block is a collection of one or more sets of knowledge points, also known as knowledge modules, in which knowledge blocks and knowledge blocks can be combined to form new knowledge modules, and a subset of knowledge blocks is called a knowledge sub-module.
	\item knowledge body: a body of knowledge is a structured system that is a combination of all the pieces of knowledge in a particular subject area.
\end{itemize}


Mathematics is a science that specializes in the relationship between quantity and spatial form. Its symbol system is more complete, the formula structure is clear and unique, and the language of expression such as words and images is more vivid and intuitive. The knowledge structure of senior secondary mathematics is a more logical and systematic knowledge system organized on the basis of the knowledge structure of junior secondary mathematics. This is because learning for any discipline needs to be based on the existing cognitive structure in order to progressively effective learning and skills training, so in the process of learning high school mathematics, you need to have a solid foundation of junior high school mathematics discipline knowledge. In the past few years, there have been a number of cases in which the students have been able to learn from each other.

\begin{itemize}
	\item Highly abstract: Mathematics has a high degree of abstraction, because the discipline's knowledge system is built using many abstract knowledge concepts, and with the help of these concepts and knowledge to learn and expand thinking, forming new abstract conceptual knowledge. The abstraction of mathematics is reflected in the object is not concerned with the introduction of specific content, only the number of relationships between the spatial form. Therefore, abstraction in mathematics is different from abstraction in other disciplines in terms of both object and degree. There are also some differences between mathematics and the natural sciences, because in mathematics the accuracy of calculations, proofs, and inferences can only be verified using rigorous logical methods and cannot be tested by repetitive experiments, whereas in the natural sciences the verification is the opposite.
	\item Strict logic: The discipline of mathematics is very logical because any conclusion reached in mathematics requires rigorous logical reasoning and rigorous proof in order to be considered reasonable. However, mathematics is not the only discipline that possesses rigorous logic; other natural science studies of reasoning and proof must also possess a certain degree of logic. In mathematics, not all conclusions reached after reasoning and proof can be applied in practice, because many mathematical models are developed and mathematical conclusions drawn under ideal circumstances.
	\item Broad applicability: Mathematics is an important means and tool for us to participate in practical social activities or scientific research, and the study of mathematics is indispensable in all walks of life and in all areas of society. Therefore, mathematics has a wide range of applications and has become an important basis for the development of modern science.
\end{itemize}

\subsection{Knowledge relation}
Knowledge relations represent the connections between knowledge points (or between knowledge blocks and knowledge chunks) in the discipline knowledge structure. It is through these connections that different knowledge points can be formed into knowledge blocks, and different knowledge blocks can be combined to form the whole disciplinary network knowledge structure system. There are many different kinds of knowledge relationships, so that different definitions of knowledge relationships lead to different knowledge structures. Therefore, in order to unify the definition of knowledge relations, we divided them into general relations and special relations based on the general and special characteristics of discipline knowledge structure. The special relationships represent the unique knowledge relationships of a particular discipline, while the universal relationships represent the general relationships of any discipline. Secondly, according to the demands of knowledge graphping research, we divide universal relations into six kinds of knowledge relations: synonymous, fraternal, antecedent, consequent, inclusive and antagonistic; and special relations into four kinds of knowledge relations: detailed, transformative, causal and correlative.
\begin{itemize}
	\item tautology: Expresses the relationship between two points of knowledge that have the same meaning as what is being described, e.g. regular and equilateral triangles.
	\item fraternity: Expresses the relationship between two knowledge points that have the same parent class.
	\item predecessor: It means that you need to finish learning knowledge point A before learning knowledge point B, that is, $A\rightarrow B$ is a precursor relationship.
	\item successor: denotes the inverse of the antecedent relationship, i.e., $B\rightarrow A$ is the successor relationship.
	\item containment: Indicates that knowledge point B is included in the definition of knowledge point A, i.e., $A\rightarrow B$ is an inclusion relationship.
	\item antagonism: From a certain point of view, knowledge point A is incompatible with knowledge point B, i.e. $A\leftrightarrow B$ is an antagonistic relationship.
	\item refinement: A grammatical analysis of the definition of knowledge point A leads to knowledge point B, where $A\leftrightarrow B$ is a detailed relationship
	\item transformation: denotes that knowledge point A and knowledge point B can be transformed to each other under certain conditions, i.e., $A\leftrightarrow B$ is a transformation relationship.
	\item causation: denotes that knowledge point A can be deduced from knowledge point B as a known condition, i.e., $A\leftrightarrow B$ is a causal relationship. 
	\item relation: Indicates that there is a relationship between the definitions of Knowledge Point A and Knowledge Point B, but the relationship is not explicitly specified, i.e., $A\leftrightarrow B$ are correlated.
\end{itemize}

In the process of constructing the discipline knowledge structure, firstly, we need to analyze the current discipline knowledge content, teaching objectives, teaching objects, teaching strategies and discipline characteristics in detail; secondly, we divide the whole discipline knowledge system into several knowledge modules, and then we divide each knowledge module into several knowledge points; finally, with reference to the above ten kinds of knowledge relationships and the knowledge relationships extracted from data sources, we can determine and establish the relationships between knowledge modules and knowledge modules, between knowledge modules and knowledge points, and between knowledge points and knowledge points, so as to form a complete discipline knowledge system structure.



\subsection{Knowledge tracing algorithms}
Knowledge Tracing is a technique that models students' knowledge acquisition based on their past answers to obtain a representation of their current knowledge state. The task is to automatically track the change of students' knowledge level over time based on their historical learning trajectory, in order to be able to accurately predict the students' performance in future learning and to provide appropriate learning tutoring. In this process, the knowledge space is used to describe the level of student knowledge acquisition. A knowledge space is a collection of concepts, and a student's mastery of a part of a collection of concepts constitutes the student's mastery of knowledge. Some educational researchers argue that students' mastery of a particular set of related knowledge points will affect their performance on the exercise, i.e., the set of knowledge that students have mastered is closely related to their external performance on the exercise.

The task of knowledge tracing is to model the student's knowledge mastery state based on the student's answer record, which is usually a time series, and in some business scenarios is time-independent, so that we can accurately predict their future answers and make reference for future intelligent questioning based on this to avoid giving students too difficult or too easy questions. Specifically, suppose a student's answer record is $x_0,x_1,...,x_t$, and we are going to predict the next interaction $x_{t+1}$, usually one interaction $x_t=(q_t,a_t)$, $q_t$ represents the right or wrong situation of that student's answer to the question $a_t$.

There are several kinds of knowledge tracing algorithms: 
\begin{itemize}
	\item Bayesian knowledge tracing(BKT): Bayesian knowledge tracing is an early and commonly used knowledge tracing model, BKT uses user interaction modeling with real-time feedback to model a learner's potential knowledge state as a set of binary variables, each representing whether or not a knowledge point is understood, and there are dynamic changes in mastery of knowledge points as students continue to practice, BKT maintains binary variables of knowledge point proficiency by using Hidden Markov Models (HMM), the original BKT model does not take into account students' knowledge forgetting, and related studies address students' guesses, personal vivid knowledge mastery and problem difficulty factors on BKT\cite{yudelson2013individualized}. 
	\item Deep Knowledge Tracing(DKT): The DKT model applies neural networks to the knowledge tracing task for the first time\cite{piech2015deep}, using an LSTM model to track the dynamics of student knowledge proficiency over time, and to learn the potential vector representation of student knowledge proficiency directly from the data. The advantage of DKT is that it can record knowledge over a longer period of time based on students' recent answers. In addition, it can update the knowledge state based on each answer, only the last implicit state needs to be saved, no double counting is required, and it is suitable for online deployment. It does not require domain knowledge, works with any user answer dataset and automatically captures associations between similar questions. The disadvantage of DKT is that the model output fluctuates greatly when the answer sequence is disrupted, i.e., the same questions and the same responses yield different knowledge states when the answer sequence is inconsistent. Due to the above-mentioned problems and the fact that students do not necessarily have continuous consistency in their k
	nowledge during the answer process, it leads to bias in the prediction of students' knowledge states influenced by the sequence. There is also the black box problem, which sometimes leads to the strange situation that the first correct answer leads to a high prediction probability for all subsequent ones, while the first wrong answer leads to a low prediction probability for all subsequent ones.
	\item Dynamic Key-Value Memory Networks for Knowledge Tracing(DKVMN): Dynamic Key-Value Memory Networks for Knowledge Tracing (DKVMN) was proposed in 2017 by Jian Jian of the Chinese University of Hong Kong\cite{zhang2017dynamic}. Based on the strengths and weaknesses of BKT and DKT and using the memory augmentation neural network approach, the Dynamic Key-Value Memory Networks (DKVMN) is proposed. It borrows ideas from memory-enhanced neural networks and combines the advantages of BKT and DKT. DKVMN stores all knowledge points with a static matrix key and a dynamic matrix value to store and update the student's knowledge state. In the DKVMN paper, they compare DKVMN with DKT and a sophisticated version of BKT, BKT+. They found that DKVMN achieves excellent performance and is the most advanced model in the KT domain. In addition to improved performance, it has several other advantages over LSTM, including prevention of overfitting, a smaller number of parameters, and automatic discovery of similar practice questions by underlying concepts. In addition, Chaudhry R\cite{chaudhry2018modeling} improves the performance of DKVMN by jointly training request cue prediction with knowledge tracing through multi-task learning.
\end{itemize}

\subsection{Factorization Machine}
The factorization machine model is a factorization model that can be used in large scale sparse data scenarios\cite{rendle2010factorization}. The solution of this model is linear in time complexity and he can solve it directly using raw data without relying on support vectors like SVM. In addition, FM is a general model that can be used on any real data and can do tasks such as classification and regression and even sorting. The idea is that the idea is to add a linear combination of two features to linear regression, and the way to solve the linear combination is to use a matrix decomposition. FFM is an improvement on FM by adding the concept of Field\cite{juan2016field}, that is, the class to which each feature belongs. Suppose there are f Fields. Then each feature has to have f hidden vectors. When two features are crossed, the dot product of each feature and the vector corresponding to the other Field is used. In this way, it is ensured that the same Field does the same thing for the same feature, and the features of different Fields do different things for the same feature. In addition, there is also a deep learning version of the factorization machine algorithm\cite{guo2017deepfm}, and this model is improved based on wide and deep. First, the model includes FM and DNN parts, which is a parallel structure, and FM and DNN share the same input (embedding). The mapping vector from the field to the embedding layer is exactly the vector learned by the FM layer. It has the advantage that it does not require pre-training and can learn the intersection of low and high dimensional features.

%********************************** % Third Section  *************************************
\section{Research Objectives and Content}  %Section - 1.3
The purpose of this study is to build a high school mathematics learning resource recommendation system based on knowledge tracing and factorization machine algorithm. We use knowledge tracing to model students' knowledge states, which outputs a graphical knowledge state vector, which we use as the next-level input, considering students' individualized differences and knowledge forgetting process, and apply the factorization machine algorithm to the resource recommendation system. For knowledge tracing, we build a graph neural network-based knowledge tracing model, which can well characterize the intrinsic connections of knowledge points in mathematics subjects considering that the knowledge points are a graph-like structure, and output a graph knowledge vector matrix, which can also effectively characterize the connections between problems and knowledge points. The output of the knowledge tracing model is then passed through a factorization machine algorithm to obtain the recommendation degree of the learning resources and output a vector of recommendation weights for different learning resources.

\section{Thesis Organization and Structure}
Chapter 1 of this paper is an introduction. It introduces the research background of the study, current industry-related research progress and the focus of the study. Then it leads to the three core points of this paper: learning resource representation, knowledge tracing and resource recommendation.

Chapter 2 of this paper concentrates on learning resource representation, which addresses storing learning resources through knowledge graphs. This paper explores some concepts of knowledge graphs, related studies, and then gives the process of knowledge graph building. It is also demonstrated that knowledge graph building can effectively characterize the a priori intrinsic features of subject knowledge.

Chapter 3 of this paper gives a knowledge tracing model of pre-trained graph neural network, which better characterizes the graph-like properties of knowledge. It is able to transform the knowledge tracing task into a time-series node-level classification problem in GNNs. Since the knowledge graph structure is not explicitly provided in most cases, we present various implementations of the graph structure. Empirical tests on two open datasets show that the method improves the prediction of student performance without any additional information and shows more interpretable predictions. The inclusion of pre-training is also attempted during the experiments, which can greatly improve the training efficiency and performance.

Chapter 4 of this paper proposes the application of a deep factorization machine algorithm with knowledge tracing model data as input to build a learning resource recommendation system. The output is a weight vector of top n, characterizing the recommended resources of top n.

Chapter 5 of this paper presents the conclusion.

\nomenclature[z-DEM]{DEM}{Discrete Element Method}
\nomenclature[z-FEM]{FEM}{Finite Element Method}
\nomenclature[z-PFEM]{PFEM}{Particle Finite Element Method}
\nomenclature[z-FVM]{FVM}{Finite Volume Method}
\nomenclature[z-BEM]{BEM}{Boundary Element Method}
\nomenclature[z-MPM]{MPM}{Material Point Method}
\nomenclature[z-LBM]{LBM}{Lattice Boltzmann Method}
\nomenclature[z-MRT]{MRT}{Multi-Relaxation
Time}
\nomenclature[z-RVE]{RVE}{Representative Elemental Volume}
\nomenclature[z-GPU]{GPU}{Graphics Processing Unit}
\nomenclature[z-SH]{SH}{Savage Hutter}
\nomenclature[z-CFD]{CFD}{Computational Fluid Dynamics}
\nomenclature[z-LES]{LES}{Large Eddy Simulation}
\nomenclature[z-FLOP]{FLOP}{Floating Point Operations}
\nomenclature[z-ALU]{ALU}{Arithmetic Logic Unit}
\nomenclature[z-FPU]{FPU}{Floating Point Unit}
\nomenclature[z-SM]{SM}{Streaming Multiprocessors}
\nomenclature[z-PCI]{PCI}{Peripheral Component Interconnect}
\nomenclature[z-CK]{CK}{Carman - Kozeny}
\nomenclature[z-CD]{CD}{Contact Dynamics}
\nomenclature[z-DNS]{DNS}{Direct Numerical Simulation}
\nomenclature[z-EFG]{EFG}{Element-Free Galerkin}
\nomenclature[z-PIC]{PIC}{Particle-in-cell}
\nomenclature[z-USF]{USF}{Update Stress First}
\nomenclature[z-USL]{USL}{Update Stress Last}
\nomenclature[s-crit]{crit}{Critical state}
\nomenclature[z-DKT]{DKT}{Draft Kiss Tumble}
\nomenclature[z-PPC]{PPC}{Particles per cell}
