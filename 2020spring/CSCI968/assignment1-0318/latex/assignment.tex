\documentclass{article}

\usepackage{fancyhdr}
\usepackage{extramarks}
\usepackage{amsmath}
\usepackage{amsthm}
\usepackage{amsfonts}
\usepackage{tikz}
\usepackage[plain]{algorithm}
\usepackage{algpseudocode}
\usepackage{listings} 
\usetikzlibrary{automata,positioning,positioning,calc}

\usepackage{color}

\definecolor{dkgreen}{rgb}{0,0.6,0}
\definecolor{gray}{rgb}{0.5,0.5,0.5}
\definecolor{mauve}{rgb}{0.58,0,0.82}

\lstset{language=Go,
  basicstyle=\ttfamily\scriptsize,
  keywordstyle=\color{blue}\ttfamily,
  stringstyle=\color{red}\ttfamily,
  commentstyle=\color{green}\ttfamily,
  breaklines,%自动换行
  }
%
% Basic Document Settings
%

\topmargin=-0.45in
\evensidemargin=0in
\oddsidemargin=0in
\textwidth=6.5in
\textheight=9.0in
\headsep=0.25in

\linespread{1.1}

\pagestyle{fancy}
\lhead{\hmwkAuthorName}
\chead{\hmwkClass\: \hmwkTitle}
\rhead{\firstxmark}
\lfoot{\lastxmark}
\cfoot{\thepage}

\renewcommand\headrulewidth{0.4pt}
\renewcommand\footrulewidth{0.4pt}

\setlength\parindent{0pt}

%
% Create Problem Sections
%

\newcommand{\enterProblemHeader}[1]{
    \nobreak\extramarks{}{Problem \arabic{#1} continued on next page\ldots}\nobreak{}
    \nobreak\extramarks{Problem \arabic{#1} (continued)}{Problem \arabic{#1} continued on next page\ldots}\nobreak{}
}

\newcommand{\exitProblemHeader}[1]{
    \nobreak\extramarks{Problem \arabic{#1} (continued)}{Problem \arabic{#1} continued on next page\ldots}\nobreak{}
    \stepcounter{#1}
    \nobreak\extramarks{Problem \arabic{#1}}{}\nobreak{}
}

\setcounter{secnumdepth}{0}
\newcounter{partCounter}
\newcounter{homeworkProblemCounter}
\setcounter{homeworkProblemCounter}{1}
\nobreak\extramarks{Problem \arabic{homeworkProblemCounter}}{}\nobreak{}

%
% Homework Problem Environment
%
% This environment takes an optional argument. When given, it will adjust the
% problem counter. This is useful for when the problems given for your
% assignment aren't sequential. See the last 3 problems of this template for an
% example.
%
\newenvironment{homeworkProblem}[1][-1]{
    \ifnum#1>0
        \setcounter{homeworkProblemCounter}{#1}
    \fi
    \section{Problem \arabic{homeworkProblemCounter}}
    \setcounter{partCounter}{1}
    \enterProblemHeader{homeworkProblemCounter}
}{
    \exitProblemHeader{homeworkProblemCounter}
}

%
% Homework Details
%   - Title
%   - Due date
%   - Class
%   - Section/Time
%   - Instructor
%   - Author
%

\newcommand{\hmwkTitle}{Assignment\ \#1}
\newcommand{\hmwkDueDate}{March 19, 2020}
\newcommand{\hmwkClass}{CSCI968 Advance Network Security}
\newcommand{\hmwkClassTime}{Assignment 1}
\newcommand{\hmwkClassInstructor}{Chen Jiageng}
\newcommand{\hmwkAuthorName}{\textbf{Mei Wangzhihui}}
\newcommand{\hmwkAuthorNum}{\textbf{2019124044}}
%
% Title Page
%

\title{
    \vspace{2in}
    \textmd{\textbf{\hmwkClass:\ \hmwkTitle}}\\
    % \normalsize\vspace{0.1in}\small{Due\ on\ \hmwkDueDate\ at 3:10pm}\\
    % \vspace{0.1in}\large{\textit{\hmwkClassInstructor\ \hmwkClassTime}}
    \vspace{3in}
}

\author{\hmwkAuthorName\ \hmwkAuthorNum}
\date{}

\renewcommand{\part}[1]{\textbf{\large Part \Alph{partCounter}}\stepcounter{partCounter}\\}

%
% Various Helper Commands
%

% Useful for algorithms
\newcommand{\alg}[1]{\textsc{\bfseries \footnotesize #1}}

% For derivatives
\newcommand{\deriv}[1]{\frac{\mathrm{d}}{\mathrm{d}x} (#1)}

% For partial derivatives
\newcommand{\pderiv}[2]{\frac{\partial}{\partial #1} (#2)}

% Integral dx
\newcommand{\dx}{\mathrm{d}x}

% Alias for the Solution section header
\newcommand{\solution}{\textbf{\large Solution}}

% Probability commands: Expectation, Variance, Covariance, Bias
\newcommand{\E}{\mathrm{E}}
\newcommand{\Var}{\mathrm{Var}}
\newcommand{\Cov}{\mathrm{Cov}}
\newcommand{\Bias}{\mathrm{Bias}}

\begin{document}

\maketitle

\pagebreak

\begin{homeworkProblem}
    \textbf{One-time password SecureID system}

    Google Authenticator. User enter a initializing code to generator a one-time password changing each period. It use the AES-128 algorithm.
    
    \textbf{S/Key System}
    The authentication to Unix-like operating system replacing long-term password. A user's real password is combined in an offline device with a short set of characters and a decrementing counter to form a single-use password. Because each password is only used once, they are useless to password sniffers. After password generation, the user has a sheet of paper with n passwords on it. It use the Random Oracle.

    \textbf{Challenge response protocol}
    Challenge–response authentication can help solve the problem of exchanging session keys for encryption. Using a key derivation function, the challenge value and the secret may be combined to generate an unpredictable encryption key for the session. This is particularly effective against a man-in-the-middle attack, because the attacker will not be able to derive the session key from the challenge without knowing the secret, and therefore will not be able to decrypt the data stream.


\end{homeworkProblem}

\begin{homeworkProblem}
    \begin{lstlisting}
        package main

        import (
            "crypto/elliptic"
            "crypto/rand"
            "crypto/sha256"
            "fmt"
            "io"
            "math/big"
        )
        
        type PublicKey struct {
            elliptic.Curve
            X, Y *big.Int
        }
        
        type PrivateKey struct {
            PublicKey
            D *big.Int 
        }
        
        var one = new(big.Int).SetInt64(1)
        
        func randFieldElement(c elliptic.Curve, rand io.Reader) (k *big.Int, err error) {
            params := c.Params()
            b := make([]byte, params.BitSize/8+8)
            _, err = io.ReadFull(rand, b)
            if err != nil {
                return
            }
        
            k = new(big.Int).SetBytes(b)
            n := new(big.Int).Sub(params.N, one)
            k.Mod(k, n)
            k.Add(k, one)
            return
        }
        
        func GenerateKey(c elliptic.Curve, rand io.Reader) (*PrivateKey, error) {
            k, err := randFieldElement(c, rand) 
            if err != nil {
                return nil, err
            }
            priv := new(PrivateKey)
            priv.PublicKey.Curve = c
            priv.D = k
            priv.PublicKey.X, priv.PublicKey.Y = c.ScalarBaseMult(k.Bytes())
            return priv, nil
        }
        
        func hashIt(x, y *big.Int, message string) *big.Int {
            tempHash := sha256.Sum256([]byte(message))
            hash := tempHash[:]
            tempInput := make([]byte, len(hash))
            tempInput = append(tempInput, hash...)
            tempInput = append(tempInput, x.Bytes()...)
            tempInput = append(tempInput, y.Bytes()...)
            temp := sha256.Sum256(tempInput)
            cTemp := temp[:]
            return new(big.Int).SetBytes(cTemp)
        }
        
        func Sign(rand io.Reader, priv *PrivateKey, message string) (msg string, x, y, a *big.Int, err error) {
        
            var k *big.Int
        
            for {
        
                k, err = randFieldElement(priv.PublicKey.Curve, rand)
                if err == nil {
                    break
                }
            }
        
            x, y = priv.Curve.ScalarBaseMult(k.Bytes())
            cV := hashIt(x, y, message)
            temp := new(big.Int).Mul(priv.D, cV)
            a = new(big.Int).Add(k, temp)
            msg = message
            return
        }
        
        func Verify(pub *PublicKey, message string, x, y, a *big.Int) bool {
        
            cV := hashIt(x, y, message)
            x1, y1 := pub.Curve.ScalarMult(pub.X, pub.Y, cV.Bytes())
            x, y = pub.Curve.Add(x1, y1, x, y)
            x2, y2 := pub.Curve.ScalarBaseMult(a.Bytes())
            return (x.Cmp(x2) == 0 && y.Cmp(y2) == 0)
        }
        
        func Sign_opt(rand io.Reader, priv *PrivateKey, message string) (msg string, cv, a *big.Int, err error) {
        
            var k *big.Int
            for {
        
                k, err = randFieldElement(priv.PublicKey.Curve, rand)
                if err == nil {
                    break
                }
            }
        
            x, y := priv.Curve.ScalarBaseMult(k.Bytes())
            cv = hashIt(x, y, message)
            temp := new(big.Int).Mul(priv.D, cv)
            a = new(big.Int).Add(k, temp)
            msg = message
            return
        }
        
        func Verify_opt(pub *PublicKey, message string, cV, a *big.Int) bool {
            x, y := pub.Curve.ScalarBaseMult(a.Bytes())
            x1, y1 := pub.Curve.ScalarMult(pub.X, pub.Y, cV.Bytes())
            negY1 := new(big.Int).Neg(y1)
            x2, y2 := pub.Curve.Add(x, y, x1, negY1)
            x3, y3 := pub.Curve.Add(x2, y2, x1, y1)
            return (x3.Cmp(x) == 0 && y3.Cmp(y) == 0)
        }
        
        func main() {
            privateKey, err := GenerateKey(elliptic.P256(), rand.Reader)
            if err != nil {
                panic(err)
            }
            msg := "CSCI468/968AdvancedNetworkSecurity,Spring2020"
        
            message, x, y, a, err := Sign(rand.Reader, privateKey, msg)
            if err != nil {
                panic(err)
            }
        
            fmt.Println("message:", message)
            fmt.Println("signature: u_t:", x, y)
            fmt.Println("signature: a_z:", a)
        
            valid := Verify(&privateKey.PublicKey, message, x, y, a)
            fmt.Println("signature verified:", valid)
        
            fmt.Println(" optimized versions:")
        
            message1, cv1, a1, err1 := Sign_opt(rand.Reader, privateKey, msg)
            if err1 != nil {
                panic(err)
            }
        
            fmt.Println("message:", message1)
            fmt.Println("signature: c:", cv1)
            fmt.Println("signature: a_z:", a1)
            valid1 := Verify_opt(&privateKey.PublicKey, message1, cv1, a1)
            fmt.Println("signature verified:", valid1)
        }
        
    \end{lstlisting}

    \textbf{Output}
    \begin{lstlisting}
        message: CSCI468/968AdvancedNetworkSecurity,Spring2020
        signature: u_t: 43068305453503561280361870469765841078369704546103943374318459436367629179869 12359168923261615701252974857854824377642988124531973618366557065204705787927
        signature: a_z: 118113457633065229502460821696906849747556761002336714124847642078416851115
        48021326205279399598472135606629328989136237936188537873579005908404438188124092
        signature verified: true
         optimized versions:
        message: CSCI468/968AdvancedNetworkSecurity,Spring2020
        signature: c: 83530092752711663050226295389097351853301230250199467408878628416381598386762
        signature: a_z: 8780426964957236590780147515699576311543968677026630689745462357275074160
        176151219015926555566164400321326536792780260274081321137737074842644458188839833
        signature verified: true
    \end{lstlisting}
    

\end{homeworkProblem}

\begin{homeworkProblem}
    \begin{tikzpicture}
    
        % Public parameter:
        \node[draw=none,fill=none,align=center] (public) at (0,1) {Schnorr signature scheme \\ $\alpha\xleftarrow{R}\mathbb{Z}_q$, $u\leftarrow g^\alpha$};
    
        % Signer
        \node[draw] (Signer) at (-2,0) {Signer};
        \draw[thick] (Signer) -- ++(0, -6);
    
        % Calculations of Signer
        \node[draw=none,fill=none,anchor=east] at ($(Signer) + (0,-1)$) {$sk\leftarrow \alpha$};
        \node[draw=none,fill=none,anchor=east] at ($(Signer) + (0,-2)$) {$\alpha_{ti}\xleftarrow{R} \mathbb{Z}_q , u_{ti}\leftarrow g^{\alpha_{ti}}$};
        \node[draw=none,fill=none,anchor=east] at ($(Signer) + (0,-3)$) {$c_i\leftarrow H(m_i,u_{ti}),\alpha_{zi}\leftarrow \alpha_{ti} +\alpha c_i$};
    
        % Verifier
        \node[draw] (Verifier) at (2,0) {Verifier};
        \draw[thick] (Verifier) -- ++(0, -6);
    
        % Calculations of Verifier
        \node[draw=none,fill=none,anchor=west] at ($(Verifier) + (0,-1)$) {$pk\leftarrow u $};
        \node[draw=none,fill=none,anchor=west] at ($(Verifier) + (0,-4)$) {$c_i\leftarrow H(m_i,u_{ti})$ \\ accept if $g^{\alpha_{zi}}=u_{ti}\cdot u^{c_i}=g^{\alpha_{ti}+\alpha c_i}$};

        % Messages
        \draw[->,thick] ($(Signer)+(0,-3.5)$) -- ($(Verifier)+(0,-3.5)$) node [pos=0.5,above,font=\footnotesize] {$\sigma_i\leftarrow (u_{ti},\alpha_{zi})$};
    \end{tikzpicture}

   When signing $m_0$, $u_{t0}=g^{\alpha_{t0}},\alpha_{z0}=\alpha_{t0}+\alpha c_0, c_0=H(m_0,u_{t0})$.
   
   When signing $m_1$, $u_{t1}=g^{a\alpha_{t0}+b}=u_{t0}^a\cdot g^b,\alpha_{z1}=a\cdot\alpha_{t0}+b+\alpha c_1,c1=H(m_1,u_{t1})$.

   If adversary obtain $(m_0,\sigma_0)$ and $(m_1,\sigma_1)$, he can get $\alpha_{z1}-a\alpha_{z0}=(c_1-ac_0)\alpha + b$ where $c_0,c_1,a,b$ is clear for adversary. 
   
   $$Adv_{sk}=P(c_1-ac_0\neq0)$$

   so the probability of obtaining secret key $\alpha$ is not negligible.
\end{homeworkProblem}

\begin{homeworkProblem}
    \begin{tikzpicture}
    
        % Public parameter:
        \node[draw=none,fill=none,align=center] (public) at (0,1) {Batch Schnorr signature scheme \\ $\alpha\xleftarrow{R}\mathbb{Z}_q$, $u\leftarrow g^\alpha$};
    
        % Signer
        \node[draw] (Signer) at (-2,0) {Signer};
        \draw[thick] (Signer) -- ++(0, -7);
    
        % Calculations of Signer
        \node[draw=none,fill=none,anchor=east] at ($(Signer) + (0,-1)$) {$sk\leftarrow \alpha$};
        \node[draw=none,fill=none,anchor=east] at ($(Signer) + (0,-2)$) {$\alpha_{ti}\xleftarrow{R} \mathbb{Z}_q , u_t\leftarrow g^{\alpha_t}$};
        \node[draw=none,fill=none,anchor=east] at ($(Signer) + (0,-3)$) {$c\leftarrow H(m,u_t),\alpha_z\leftarrow \alpha_t +\alpha c$};
    
        % Verifier
        \node[draw] (Verifier) at (2,0) {Verifier};
        \draw[thick] (Verifier) -- ++(0, -7);
    
        % Calculations of Verifier
        \node[draw=none,fill=none,anchor=west] at ($(Verifier) + (0,-1)$) {$pk\leftarrow u $};
        \node[draw=none,fill=none,anchor=west] at ($(Verifier) + (0,-4)$) {$\beta_1,\beta_2,...,\beta_n\xleftarrow{R}\mathcal{C}$};
        \node[draw=none,fill=none,anchor=west] at ($(Verifier) + (0,-5)$) {$c_i\leftarrow H(m_i,u_{ti})$,$\bar{\alpha}\leftarrow \sum_{i=1}^n{\beta_i \alpha_{zi}}\in\mathbb{Z}_q$ and $\bar{c}\leftarrow \sum_{i=1}^n\beta_ic_i \in \mathbb{Z}_q$};
        \node[draw=none,fill=none,anchor=west] at ($(Verifier) + (0,-6)$) {accept if $g^{\bar{\alpha}}=u^{\bar{c}}\cdot \prod_{i=1}^n u_{ti}^{\beta_i}$};

        % Messages
        \draw[->,thick] ($(Signer)+(0,-3.5)$) -- ($(Verifier)+(0,-3.5)$) node [pos=0.5,above,font=\footnotesize] {$\sigma_i\leftarrow (u_{ti},\alpha_{zi})$};
        \draw[->,thick] ($(Signer)+(0,-4.1)$) -- ($(Verifier)+(0,-4.1)$) node [pos=0.5,above,font=\footnotesize] {\vdots};

    \end{tikzpicture}
    We get 
    \begin{equation}
        \begin{split}
            g^{\bar{\alpha}}&=g^{\sum_{i=1}^n{\beta_i\alpha_{zi}}}\\
            &=\prod_{i=1}^n({g^{\alpha_{zi}}})^{\beta_i}\\
            &=g^{\sum_{i=1}^n \alpha_{zi}\beta_i}
        \end{split}
    \end{equation}
    \begin{equation}
        \begin{split}
           u^{\bar{c}}\cdot\prod_{i=1}^n{u_{ti}^{\beta_i}}
           &=g^{\alpha\cdot\sum_{i=1}^n{\beta_i c_i}} \cdot g^{\sum_{i=1}^n{\alpha_{ti}\beta_i}}\\
           &=g^{\sum_{i=1}^n{(\alpha_{ti}+\alpha c_i)\beta_i}}\\
           &=\prod_{i=1}^n (g^{\alpha_{ti}+\alpha c_i})^{\beta_i}\\
           &=g^{\sum_{i=1}^n(\alpha_{ti}+\alpha c_i)\beta_i}
        \end{split}
    \end{equation}
    Suppose that $\beta_i$ is not used, then the $g^{\bar{\alpha}}=g^{\sum_{i=1}^n \alpha_{zi}}$ and $u^{\bar{c}}\cdot\prod_{i=1}^n{u_{ti}^{\beta_i}}=g^{\sum_{i=1}^n(\alpha_{ti}+\alpha c_i)}$

    $$P(\exists{\alpha_{zi}\neq\alpha_{ti}+\alpha c_i},g^{\sum_{i=1}^n \alpha_{zi}}=g^{\sum_{i=1}^n(\alpha_{ti}+\alpha c_i)})$$ is not negligible.
    while $$P(\exists{\alpha_{zi}\neq\alpha_{ti}+\alpha c_i},g^{\sum_{i=1}^n \alpha_{zi}\beta_i}=g^{\sum_{i=1}^n(\alpha_{ti}+\alpha c_i)\beta_i})$$ is negligible.
    so advantage is $$Adv_{BSV}=P(\exists{\alpha_{zi}\neq\alpha_{ti}+\alpha c_i},g^{\sum_{i=1}^n \alpha_{zi}\beta_i}=g^{\sum_{i=1}^n(\alpha_{ti}+\alpha c_i)\beta_i})P$$

\end{homeworkProblem}
\end{document}
