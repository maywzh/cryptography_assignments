\documentclass[11pt,en]{elegantpaper}
\usepackage{float}


\title{Signal messaging service technical report}
\author{Wangzhihui Mei \\ 2019124044 6603385}
\institute{CCNU-UOW JI}

\version{}
\date{}


\begin{document}

\maketitle

\begin{abstract}
% Online messaging service has been playing an important role in people’s daily life for social activities and other related businesses. The most widely used messaging services include Wechat,Whatsapp, Facebook, Line and so on. On the other hand, the privacy issues related to the messaging services are being discussed and investigated more and more often recently. The concept of the end to end encryption (E2EE) is proposed to mainly address the security issues that the private information may be compromised at the messaging server. In other words,in the scenario that messaging servers are required, which is the case for most of the asynchronous messaging services, messages communicated between two or multiple parties need to go through the servers for various functionality purposes. As a result, precautions need to betaken on whether the messages and other related information can be learned only by the end parties or not.

%Signal is one of the most popular messaging services that claims to achieve the end to end encryption security level and what’s more importantly, it is an open source project applying the noise cryptographic protocol framework[1], which is also used by Whatsapp, wireguard, facebook Messenger, Skype, and Google Allo.



\end{abstract}

%TODO: You are required to provide a research report on the above issues containing the related background introduction and the corresponding solutions. Protocols should be concrete including all the mathematical details of the primitives you would apply. Please explain the reason behind the design to support the correctness and security properties, and show that the design can indeed achieve the goals.One technical report (12 pages in length excluding the references) should be submitted with the following format:


\section{Introduction}
%TODO: Including the Signal protocol and other related background
In the modern network environment, people have increasing demands for privacy protection. Now the world is worried that people ’s personal privacy will be violated, as people use instant messaging apps and services, where the service provider may be the vulnerable because of attacker can crack the server or perform Man-In-The-Middle attack in the case that only transmission encryption is adopted. In other scenario, the privacy of user is transparent to server, so service providers may acquire the content of communication as they want. To solve the natural weakness of transmission encryption, End-to-end encryption is introduced. 

End-to-end encryption (E2EE) is a communication system where only users participating in the communication can read the information. In general, it can prevent potential eavesdroppers-including telecommunications providers, Internet services. Such systems are designed to prevent potential surveillance or corrective attempts, because it is difficult for third parties without keys to decipher Data transmitted or stored. Generally speaking, communication providers that use end-to-end encryption will not be able to provide their customers' communication data to the specification. 

Signal is an excellent End-to-end encryption protocol.\cite{alwen2019double} It is very famous in both IT and security field and applied in Whatsapp,Facebook Messenger,Skype, etc. The core algorithm of Signal protocol is X3DH and Double Ratchet, referring to the key agreement protocol "Extended Triple Diffie-Hellman" and one secure key management algorithm respectively. 








\section{Solution}
%TODO: In this research report, you are required to first understand how Signal messaging service work by referring to the documents [2, 3, 4] along with the source code [5]. Then based on the Signal framework, provide your solutions to the following two issues.



\subsection{Issue 1}
%TODO: Group chatting is one of the important functions in the online messaging application. Users within the group should be able to communicate with each other securely; A users should be allowed to join or leave the group; messaging server should not be able to know the sensitive information regarding the group member identities, message content and so on.You are required to define the security goals in detail that you believe to be reasonable in the group chatting scenario, and provide the solution based on the Signal framework. [2,3,4]



\subsection{Issue 2}
%TODO: There are many popular messaging services which do not satisfy the E2EE security level, especially for the non-open source products. This is sometimes due to the auditing or other requirements by both the government and the company itself. Please provide the solution for this scenario so that the security requirements are satisfied as in the E2EE scenario except for the case that the messaging server is able to audit the corresponding communication session (message content, and user identity and so on). Please describe how you can limit the damage if the messaging server is compromised.

\section{Conclusion}



\bibliography{wpref}

\end{document}
