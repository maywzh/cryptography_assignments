\documentclass{article}
\usepackage{fancyhdr}
\usepackage{extramarks}
\usepackage{amsmath}
\usepackage{amsthm}
\usepackage{amsfonts}
\usepackage{tikz}
\usepackage[plain]{algorithm}
\usepackage{algpseudocode}
\usepackage{listings} 
\usepackage{neuralnetwork}
\usetikzlibrary{automata,positioning}

\usepackage{color}

\definecolor{dkgreen}{rgb}{0,0.6,0}
\definecolor{gray}{rgb}{0.5,0.5,0.5}
\definecolor{mauve}{rgb}{0.58,0,0.82}

\lstset{frame=tb,
  language=Python,
  aboveskip=3mm,
  belowskip=3mm,
  showstringspaces=false,
  columns=flexible,
  basicstyle={\small\ttfamily},
  numbers=none,
  numberstyle=\tiny\color{gray},
  keywordstyle=\color{blue},
  commentstyle=\color{dkgreen},
  stringstyle=\color{mauve},
  breaklines=true,
  breakatwhitespace=true,
  tabsize=3
}
%
% Basic Document Settings
%

\topmargin=-0.45in
\evensidemargin=0in
\oddsidemargin=0in
\textwidth=6.5in
\textheight=9.0in
\headsep=0.25in

\linespread{1.1}

\pagestyle{fancy}
\lhead{\hmwkAuthorName}
\chead{\hmwkClass\: \hmwkTitle}
\rhead{\firstxmark}
\lfoot{\lastxmark}
\cfoot{\thepage}

\renewcommand\headrulewidth{0.4pt}
\renewcommand\footrulewidth{0.4pt}

\setlength\parindent{0pt}

%
% Create Problem Sections
%

\newcommand{\enterProblemHeader}[1]{
    \nobreak\extramarks{}{Task \arabic{#1} continued on next page\ldots}\nobreak{}
    \nobreak\extramarks{Task \arabic{#1} (continued)}{Problem \arabic{#1} continued on next page\ldots}\nobreak{}
}

\newcommand{\exitProblemHeader}[1]{
    \nobreak\extramarks{Task \arabic{#1} (continued)}{Problem \arabic{#1} continued on next page\ldots}\nobreak{}
    \stepcounter{#1}
    \nobreak\extramarks{Task \arabic{#1}}{}\nobreak{}
}

\setcounter{secnumdepth}{0}
\newcounter{partCounter}
\newcounter{homeworkProblemCounter}
\setcounter{homeworkProblemCounter}{1}
\nobreak\extramarks{Task \arabic{homeworkProblemCounter}}{}\nobreak{}

%
% Homework Problem Environment
%
% This environment takes an optional argument. When given, it will adjust the
% problem counter. This is useful for when the problems given for your
% assignment aren't sequential. See the last 3 problems of this template for an
% example.
%
\newenvironment{homeworkProblem}[1][-1]{
    \ifnum#1>0
        \setcounter{homeworkProblemCounter}{#1}
    \fi
    \section{Task \arabic{homeworkProblemCounter}}
    \setcounter{partCounter}{1}
    \enterProblemHeader{homeworkProblemCounter}
}{
    \exitProblemHeader{homeworkProblemCounter}
}

%
% Homework Details
%   - Title
%   - Due date
%   - Class
%   - Section/Time
%   - Instructor
%   - Author
%

\newcommand{\hmwkTitle}{Lab\ \#4}
\newcommand{\hmwkDueDate}{September 25, 2019}
\newcommand{\hmwkClass}{CSCI964 Computational Intelligence}
\newcommand{\hmwkClassTime}{3.5}
\newcommand{\hmwkClassInstructor}{Zhifeng Wang}
\newcommand{\hmwkAuthorName}{\textbf{Mei Wangzhihui}}
\newcommand{\hmwkAuthorNum}{\textbf{2019124044}}
%
% Title Page
%

\title{
    \vspace{2in}
    \textmd{\textbf{\hmwkClass:\ \hmwkTitle}}\\
    % \normalsize\vspace{0.1in}\small{Due\ on\ \hmwkDueDate\ at 3:10pm}\\
    % \vspace{0.1in}\large{\textit{\hmwkClassInstructor\ \hmwkClassTime}}
    \vspace{3in}
}

\author{\hmwkAuthorName\ \hmwkAuthorNum}
\date{}

\renewcommand{\part}[1]{\textbf{\large Part \Alph{partCounter}}\stepcounter{partCounter}\\}

%
% Various Helper Commands
%

% Useful for algorithms
\newcommand{\alg}[1]{\textsc{\bfseries \footnotesize #1}}

% For derivatives
\newcommand{\deriv}[1]{\frac{\mathrm{d}}{\mathrm{d}x} (#1)}

% For partial derivatives
\newcommand{\pderiv}[2]{\frac{\partial}{\partial #1} (#2)}

% Integral dx
\newcommand{\dx}{\mathrm{d}x}

% Alias for the Solution section header
\newcommand{\solution}{\textbf{\large Solution}}

% Probability commands: Expectation, Variance, Covariance, Bias
\newcommand{\E}{\mathrm{E}}
\newcommand{\Var}{\mathrm{Var}}
\newcommand{\Cov}{\mathrm{Cov}}
\newcommand{\Bias}{\mathrm{Bias}}

\begin{document}

\maketitle

\pagebreak

\begin{homeworkProblem}
\lstset{language=C}
\begin{lstlisting}
    import libsvm.commonutil as commonutil
    import libsvm.svmutil as sut
    
    y, x = sut.svm_read_problem("../data/iris_libsvm_random.txt")
    model = sut.svm_train(y, x, '-t 0')
    
    p_label, p_acc, p_val = sut.svm_predict(y, x, model)
    
    
\end{lstlisting}
\textbf{linear kernel}

iter = 32
nu = 0.204900
obj = -15.759854, rho = -6.783209
nSV = 23, nBSV = 19

iter = 5
nu = 0.014961
obj = -0.748057, rho = 1.447464
nSV = 3, nBSV = 0

iter = 12
nu = 0.004074
obj = -0.203684, rho = 1.507573
nSV = 3, nBSV = 0
Total nSV = 27
Accuracy = 99.3333 (149/150) (classification)

\textbf{polynomial kernel}
iter = 29574
nu = 0.062675
obj = -5.885061, rho = -7.005220
nSV = 11, nBSV = 4

iter = 6
nu = 0.000188
obj = -0.009399, rho = 1.134350
nSV = 3, nBSV = 0

iter = 9
nu = 0.000036
obj = -0.001817, rho = 1.184761
nSV = 4, nBSV = 0
Total nSV = 16
Accuracy = 98\% (147/150) (classification)

\textbf{radial basis function}
iter = 47
nu = 0.293557
obj = -21.377492, rho = 0.144354
nSV = 33, nBSV = 26

iter = 49
nu = 0.046504
obj = -2.403415, rho = -0.040019
nSV = 11, nBSV = 2

iter = 22
nu = 0.038910
obj = -1.945147, rho = -0.167726
nSV = 10, nBSV = 0
Total nSV = 45
Accuracy = 98.6667\% (148/150) (classification)

\textbf{sigmoid: tanh}
iter = 50
nu = 1.000000
obj = -100.000000, rho = 0.000000
nSV = 100, nBSV = 100

iter = 50
nu = 1.000000
obj = -100.000043, rho = 0.000001
nSV = 100, nBSV = 100

iter = 50
nu = 1.000000
obj = -100.000047, rho = 0.000000
nSV = 100, nBSV = 100
Total nSV = 150
Accuracy = 4\% (6/150) (classification)
\end{homeworkProblem}

\end{document}
