\documentclass[11pt,en]{elegantpaper}

\title{Seminar Review}
\author{Wangzhihui Mei 2019124044}
\institute{CCNU-UOW JI}

\version{}
\date{}


\begin{document}

\maketitle
\section{Computational Intelligence Applications in Challenging Disciplines}

% With the boom of AI and machine learning technologies, huge development of new algorithms and methods in areas such as cybersecurity and computer vision has seen human’s life much easier. However, there are some disciplines, such as education, medicine and transport, still lacking appropriate approaches to tackling challenging problems. This talk will introduce some research work carried out at Centre for Digital Transformation in University of Wollongong, where evolutionary algorithms, classical machine learning and deep learning methods are helping with some novel solutions. The underlying problems include cold start recommendation in micro learning, pathogen-pathogen protein interaction predictions and primary delay analysis in train networks.

Huge development of Artificial Intelligence technologies has been made during the second decade of 21st  century. Computational Intelligence (CI) is a branch of computer science studying problems for which there are no effective computational algorithms. CI generally focus on problems that only humans and animals can solve, problems requiring intelligence. From this point of view AI is a part of CI focusing on problems that require higher cognition and at present are easier to solve using symbolic knowledge representation. It is possible that other CI methods will also find applications to these problems in future. A good part of CI research is concerned with low-level cognitive functions: perception, object recognition, signal analysis, discovery of structures in data, simple associations and control. Methods developed for this type of problems include supervised and unsupervised learning by adaptive systems, and they encompass not only neural, fuzzy and evolutionary approaches but also probabilistic and statistical approaches, such as Bayesian networks or kernel methods. 

While the application of CI runs smoothly in some scenarios some disciplines lacking appropriate approaches to tackling challenging problems exists, such as education, medicine and transport. There are some underlying problems in the disciplines:
\begin{itemize}
	\item cold start recommendation in micro learning
	\item pathogen-pathogen protein interaction predictions
	\item primary delay analysis in train networks
\end{itemize}

Micro learning refers to short-term learning activities on small learning units. It is usually difficult to provide reliable recommendations due to the insufficiency of initial data of ratings or preferences. This leads to the occurrence of cold start problem. The cold start problem becomes more severe in the open learning, especially in micro learning through open education resources. Pathogen-pathogen protein interaction, for example, impacts various virulence factors in human infection. The prediction model for train delay is useful not only for passengers wishing to plan their journeys more reliably, but also for railway operators developing more efficient train schedules and more reasonable pricing plans.
\section{The Development of Bag-of-features Model in Image Recognition}
Bag-of-features model is a vector of occurrence counts of a vocabulary of local image features, which can be applied to image classification by treating image features as words. The outline of bag-of-features is:
\begin{enumerate}
	\item Extract features
	\item Learn “visual vocabulary”
	\item Quantize features using visual vocabulary
	\item Represent images by frequencies of “visual words” 
\end{enumerate}

There are also 3 key components: codebook generation, coding scheme, and pooling method. The codebook generation is the process to transform the local descriptor obtained to a new form. The creating codebook use the K-means clustering, which is a vector quantization method and the potential most common way of constructing the visual vocabulary. The standard BoF method for encoding transforms the local descriptor into a more adapted form using the codebook. In the encoding step three different approaches can be applied, first a hard assignment encoding as the standard BoF vector quantization method. Then, in order to better apprehend the new soft-assignment technique, a distance-based soft quantization method and a reconstruction-based soft assignment method renowned are used for its good performance. After encoded descriptors are obtained, an image is still represented by too many code vectors (encoded descriptors), so pooling these vectors is needed in order to describe the image with one final descriptor. Given the coding coefficient $\gamma$ of each local descriptor in an image, a pooling operation is often used to obtain an image level representation $p$ where $p\in R^M$ with $M$ the total number of visual words.

Convolutional Neural Networks are powerful well-known models used for various computer vision tasks, ranging from image classification, to visual question answering. However, they are becoming increasingly larger, using millions of parameters, while they are restricted to handling images of fixed size. A quantization-based approach, inspired from the well-known Bag-of-Features model, is proposed to overcome these limitations. The proposed approach, called Convolutional BoF, uses
RBF neurons to quantize the information extracted from the convolutional layers and it is able to natively classify images of various sizes as well as to significantly reduce the number of parameters in the network. The ability of the proposed method to reduce the parameters of the network and increase the classification accuracy over other state-of-the-art techniques is demonstrated using three image datasets.

\end{document}
