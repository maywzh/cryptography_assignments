\documentclass{article}
\usepackage{fancyhdr}
\usepackage{extramarks}
\usepackage{amsmath}
\usepackage{amsthm}
\usepackage{amsfonts}
\usepackage{tikz}
\usepackage[plain]{algorithm}
\usepackage{algpseudocode}
\usepackage{listings} 
\usepackage{amssymb}
\usetikzlibrary{automata,positioning}

\usepackage{color}

\definecolor{dkgreen}{rgb}{0,0.6,0}
\definecolor{gray}{rgb}{0.5,0.5,0.5}
\definecolor{mauve}{rgb}{0.58,0,0.82}

\lstset{frame=tb,
  language=Python,
  aboveskip=3mm,
  belowskip=3mm,
  showstringspaces=false,
  columns=flexible,
  basicstyle={\small\ttfamily},
  numbers=none,
  numberstyle=\tiny\color{gray},
  keywordstyle=\color{blue},
  commentstyle=\color{dkgreen},
  stringstyle=\color{mauve},
  breaklines=true,
  breakatwhitespace=true,
  tabsize=3
}
%
% Basic Document Settings
%

\topmargin=-0.45in
\evensidemargin=0in
\oddsidemargin=0in
\textwidth=6.5in
\textheight=9.0in
\headsep=0.25in

\linespread{1.1}

\pagestyle{fancy}
\lhead{\hmwkAuthorName}
\chead{\hmwkClass\: \hmwkTitle}
\rhead{\firstxmark}
\lfoot{\lastxmark}
\cfoot{\thepage}

\renewcommand\headrulewidth{0.4pt}
\renewcommand\footrulewidth{0.4pt}

\setlength\parindent{0pt}

%
% Create Problem Sections
%

\newcommand{\enterProblemHeader}[1]{
    \nobreak\extramarks{}{Problem \arabic{#1} continued on next page\ldots}\nobreak{}
    \nobreak\extramarks{Problem \arabic{#1} (continued)}{Problem \arabic{#1} continued on next page\ldots}\nobreak{}
}

\newcommand{\exitProblemHeader}[1]{
    \nobreak\extramarks{Problem \arabic{#1} (continued)}{Problem \arabic{#1} continued on next page\ldots}\nobreak{}
    \stepcounter{#1}
    \nobreak\extramarks{Problem \arabic{#1}}{}\nobreak{}
}

\setcounter{secnumdepth}{0}
\newcounter{partCounter}
\newcounter{homeworkProblemCounter}
\setcounter{homeworkProblemCounter}{1}
\nobreak\extramarks{Problem \arabic{homeworkProblemCounter}}{}\nobreak{}

%
% Homework Problem Environment
%
% This environment takes an optional argument. When given, it will adjust the
% problem counter. This is useful for when the problems given for your
% assignment aren't sequential. See the last 3 problems of this template for an
% example.
%
\newenvironment{homeworkProblem}[1][-1]{
    \ifnum#1>0
        \setcounter{homeworkProblemCounter}{#1}
    \fi
    \section{Problem \arabic{homeworkProblemCounter}}
    \setcounter{partCounter}{1}
    \enterProblemHeader{homeworkProblemCounter}
}{
    \exitProblemHeader{homeworkProblemCounter}
}

%
% Homework Details
%   - Title
%   - Due date
%   - Class
%   - Section/Time
%   - Instructor
%   - Author
%
\newcommand{\hmwkNum}{6}
\newcommand{\hmwkTitle}{Homework\ \#\hmwkNum}
\newcommand{\hmwkDueDate}{October 31, 2019}
\newcommand{\hmwkClass}{CSCI971 Advance Computer Security}
\newcommand{\hmwkClassInstructor}{Chen Jiageng}
\newcommand{\hmwkAuthorName}{\textbf{Mei Wangzhihui}}
\newcommand{\hmwkAuthorNum}{\textbf{2019124044}}
%
% Title Page
%

\title{
    \vspace{2in}
    \textmd{\textbf{\hmwkClass:\\ \hmwkTitle}}\\
    % \normalsize\vspace{0.1in}\small{Due\ on\ \hmwkDueDate\ at 3:10pm}\\
    % \vspace{0.1in}\large{\textit{\hmwkClassInstructor\ \hmwkClassTime}}
    \vspace{3in}
}

\author{\hmwkAuthorName\ \\ \hmwkAuthorNum}
\date{}

\renewcommand{\part}[1]{\textbf{\large Part \Alph{partCounter}}\stepcounter{partCounter}\\}

%
% Various Helper Commands
%

% Useful for algorithms
\newcommand{\alg}[1]{\textsc{\bfseries \footnotesize #1}}

% For derivatives
\newcommand{\deriv}[1]{\frac{\mathrm{d}}{\mathrm{d}x} (#1)}

% For partial derivatives
\newcommand{\pderiv}[2]{\frac{\partial}{\partial #1} (#2)}

% Integral dx
\newcommand{\dx}{\mathrm{d}x}

% Alias for the Solution section header
\newcommand{\solution}{\textbf{\large Solution}}

% Probability commands: Expectation, Variance, Covariance, Bias
\newcommand{\E}{\mathrm{E}}
\newcommand{\Var}{\mathrm{Var}}
\newcommand{\Cov}{\mathrm{Cov}}
\newcommand{\Bias}{\mathrm{Bias}}

\begin{document}

\maketitle

\pagebreak

\begin{homeworkProblem}
\textbf{Solution:} \\
\textbf{a)\ How does a Merkle tree work?}\\
The original message are a sequence of l-bit blocks $x_1,x_2,...,x_n$. We just verify several blocks which are used in in the block set to minimize the computation size. Each block can be verified independently.

The Merkle tree applied collision resistant function $h$ to each block in $(x_1,x_2,...,x_n)\in \mathcal{X}^n$ get the accordant hash value set $(y_1,y_2,...,y_n)\in \mathcal{Y}^n$ by the algorithm:\\
for i = 1 to n, $y_i\leftarrow h(x_i)$

Then applied $h$ to $(y_1,y_2,...,y_n)$ to get the parental nodes $(y_{n+1},..., y_{2n+1})$ by the algorithm:

for i = 1 to n-1 , $y_{i+n}\leftarrow h(y_{2i-1},y_{2i})$\\
So we get an binary tree with $2n+1$ nodes.

When we want to verify the ith block ($\hat{x_i}=x_i?$), the algorithm need \textbf{Merkle proof} $\pi$, which is the intermediate hashes  of siblings of nodes on the path from i to root. For example, set $i=5, n=8$, then $\pi=(y_6,y_{12},y_{13})$. 

We then do hash computation for $\hat{x_i}$ from ith node position to root. In this example, calculate:\\
$\hat{y_5} \leftarrow h(\hat{x_5})\\ \hat{y_{11}}=h(\hat{y_5},y_6)\\\hat{y_{15}}=h(\hat{y_{11}}, y_{13})$

If $\hat{y_{15}}=y_{15}$, then $hat{x_5}$ is verified, else not. 
\\

\textbf{b)\ Why is it efficient when using Merkle tree to prove membership?}\\
As the computation complexity for binary tree from leaf to root is $O(log_2^n)$, so the there need $log_2^n$ times of hash computation in the verification process. 

Also, Merkle tree algorithm need not secret keys.
\\

\textbf{c)\ How to take advantage of a Merkle tree to prove non-membership?}\\
Suppose verifier want to verify that $x$ is not in the list $T$, The verifier sort all the elements in $T$, and then build Merkle tree, this is sorted Merkle tree. 

The verifier then find two adjancent leaves $x_{i},x_{i+1}$ and it satify that $x_{i}<x<x_{i+1}$. Next, verifier check Merkle proofs of $x_{i},x_{i+1}$, to make sure $x_{i},x_{i+1}$ is in the Merkle tree. If $x\neq x_{i}$ and $x\neq x_{i+1}$ then $x$ is not int the $T$, as $x_{i}$ and $x_{i+1}$ are adjancent leaf nodes. Else, $x$ may be either $x_{i}$ or $x_{i+1}$, so $x$ is in the $T$. 
\\

\textbf{d)\ How does blockchain use Merkle tree to verify transactions? Please describe by concrete example.}\\

\end{homeworkProblem}
\end{document}


