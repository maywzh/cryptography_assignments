\documentclass{article}
\usepackage{fancyhdr}
\usepackage{extramarks}
\usepackage{amsmath}
\usepackage{amsthm}
\usepackage{amsfonts}
\usepackage{tikz}
\usepackage[plain]{algorithm}
\usepackage{algpseudocode}
\usepackage{listings}
\usepackage{amssymb}
\usetikzlibrary{automata,positioning,graphs}

\usepackage{color}

\definecolor{dkgreen}{rgb}{0,0.6,0}
\definecolor{gray}{rgb}{0.5,0.5,0.5}
\definecolor{mauve}{rgb}{0.58,0,0.82}

\lstset{frame=tb,
  language=Python,
  aboveskip=3mm,
  belowskip=3mm,
  showstringspaces=false,
  columns=flexible,
  basicstyle={\small\ttfamily},
  numbers=none,
  numberstyle=\tiny\color{gray},
  keywordstyle=\color{blue},
  commentstyle=\color{dkgreen},
  stringstyle=\color{mauve},
  breaklines=true,
  breakatwhitespace=true,
  tabsize=3
}
%
% Basic Document Settings
%

\topmargin=-0.45in
\evensidemargin=0in
\oddsidemargin=0in
\textwidth=6.5in
\textheight=9.0in
\headsep=0.25in

\linespread{1.1}

\pagestyle{fancy}
\lhead{\hmwkAuthorName}
\chead{\hmwkClass\: \hmwkTitle}
\rhead{\firstxmark}
\lfoot{\lastxmark}
\cfoot{\thepage}

\renewcommand\headrulewidth{0.4pt}
\renewcommand\footrulewidth{0.4pt}

\setlength\parindent{0pt}

%
% Create Problem Sections
%

\newcommand{\enterProblemHeader}[1]{
    \nobreak\extramarks{}{Problem \arabic{#1} continued on next page\ldots}\nobreak{}
    \nobreak\extramarks{Problem \arabic{#1} (continued)}{Problem \arabic{#1} continued on next page\ldots}\nobreak{}
}

\newcommand{\exitProblemHeader}[1]{
    \nobreak\extramarks{Problem \arabic{#1} (continued)}{Problem \arabic{#1} continued on next page\ldots}\nobreak{}
    \stepcounter{#1}
    \nobreak\extramarks{Problem \arabic{#1}}{}\nobreak{}
}

\setcounter{secnumdepth}{0}
\newcounter{partCounter}
\newcounter{homeworkProblemCounter}
\setcounter{homeworkProblemCounter}{1}
\nobreak\extramarks{Problem \arabic{homeworkProblemCounter}}{}\nobreak{}

%
% Homework Problem Environment
%
% This environment takes an optional argument. When given, it will adjust the
% problem counter. This is useful for when the problems given for your
% assignment aren't sequential. See the last 3 problems of this template for an
% example.
%
\newenvironment{homeworkProblem}[1][-1]{
    \ifnum#1>0
        \setcounter{homeworkProblemCounter}{#1}
    \fi
    \section{Problem \arabic{homeworkProblemCounter}}
    \setcounter{partCounter}{1}
    \enterProblemHeader{homeworkProblemCounter}
}{
    \exitProblemHeader{homeworkProblemCounter}
}

%
% Homework Details
%   - Title
%   - Due date
%   - Class
%   - Section/Time
%   - Instructor
%   - Author
%
\newcommand{\hmwkNum}{7}
\newcommand{\hmwkTitle}{Homework\ \#\hmwkNum}
\newcommand{\hmwkDueDate}{November 20, 2019}
\newcommand{\hmwkClass}{CSCI971 Advance Computer Security}
\newcommand{\hmwkClassInstructor}{Chen Jiageng}
\newcommand{\hmwkAuthorName}{\textbf{Mei Wangzhihui}}
\newcommand{\hmwkAuthorNum}{\textbf{2019124044}}
%
% Title Page
%

\title{
    \vspace{2in}
    \textmd{\textbf{\hmwkClass:\\ \hmwkTitle}}\\
    % \normalsize\vspace{0.1in}\small{Due\ on\ \hmwkDueDate\ at 3:10pm}\\
    % \vspace{0.1in}\large{\textit{\hmwkClassInstructor\ \hmwkClassTime}}
    \vspace{3in}
}

\author{\hmwkAuthorName\ \\ \hmwkAuthorNum}
\date{}

\renewcommand{\part}[1]{\textbf{\large Part \Alph{partCounter}}\stepcounter{partCounter}\\}

%
% Various Helper Commands
%

% Useful for algorithms
\newcommand{\alg}[1]{\textsc{\bfseries \footnotesize #1}}

% For derivatives
\newcommand{\deriv}[1]{\frac{\mathrm{d}}{\mathrm{d}x} (#1)}

% For partial derivatives
\newcommand{\pderiv}[2]{\frac{\partial}{\partial #1} (#2)}

% Integral dx
\newcommand{\dx}{\mathrm{d}x}

% Alias for the Solution section header
\newcommand{\solution}{\textbf{\large Solution}}

% Probability commands: Expectation, Variance, Covariance, Bias
\newcommand{\E}{\mathrm{E}}
\newcommand{\Var}{\mathrm{Var}}
\newcommand{\Cov}{\mathrm{Cov}}
\newcommand{\Bias}{\mathrm{Bias}}

\begin{document}

\maketitle

\clearpage

\begin{homeworkProblem}

AE-secure $\Leftrightarrow$ semantically secure under CPA and CI.

For the first cipher, assume an attacker who can perform CPA. He intercept the ciphertext $c=E_1(k, m)=(E(k,m),H_1(m))$, He can perform as many as CPA. We assume in CPA attack game. Adversary $\mathcal{A}$ first send $m_{0},m_{0}$ to challenger $\mathcal{C}$, he get the ciphertext $c=(E(k_0,m_0),H_1(m_0))$. Then $\mathcal{A}$ send $m_{0},m_{1}$ to $\mathcal{C}$, as $E$ is CPA secure, so key has to be changed. $\mathcal{A}$ get the ciphertext  $c=(E(k_1,m_0),H_1(m_0))$ or  $c=(E(k_1,m_1),H_1(m_1))$ based on $b$. Then if $b=1$, $\mathcal{A}$ can easily differ the plaintext from the tag $H_1(m_b)$. So $Adv_{CPA}(\mathcal{A},\mathcal{E}) = 1/2$ is not negligible. Cipher1 is not CPA-secure, so it's not AE-secure.

For the second cipher, attacker can intercept the ciphertext $(c,H_2(c))$, so he can learn the mapping model of $H_2$ function. So in CI attack game, Adversary $\mathcal{A}$ can easily generate an valid ciphertext-tag pair $(c_{atk}, H_2(c_{atk}))$. Then Decryptor $D_2(k, (c_{atk},H_2(c_{atk}))) \neq \perp$. So $Adv_{CI}(\mathcal{A},\mathcal{E})$ is not negligible. Cipher2 does not safisfy CI, so it's not AE-secure.

\end{homeworkProblem}

\begin{homeworkProblem}
    
Addition $\mathcal{Z}_6^*$ is a cyclic group. 

$\mathcal{Z}_6^*=\{0,1,2,3,4,5,6\}$ 

1 generate \{0,1,2,3,4,5\}

2 generate \{0,2,4\}

3 generate \{0,3\}

4 generate \{0,2,4\}

5 generate \{0,1,2,3,4,5\}

So the generators of $\mathcal{Z}_6^*$ are 1,5, the subgroups are \{0,1,2,3,4,5\}, \{0,2,4\}, \{0,3\}



\end{homeworkProblem}

\begin{homeworkProblem}





Group under multiplication $\mathcal{Z}_{13}^*$ is a cyclic group.

$\mathcal{Z}_{13}^*=\{0,1,2,3,4,5,6,7,8,9,10,11,12\}$

    <1> = \{1\} \\
    <2> = \{1, 2, 3, 4, 5, 6, 7, 8, 9, 10, 11, 12\} \\ 
    <3> = \{1, 3, 9\} \\
    <4> = \{1, 3, 4, 9, 10, 12\} \\ 
    <5> = \{1, 5, 8, 12\} \\
    <6> = \{1, 2, 3, 4, 5, 6, 7, 8, 9, 10, 11, 12\} \\
    <7> = \{1, 2, 3, 4, 5, 6, 7, 8, 9, 10, 11, 12\} \\ 
    <8> = \{1, 5, 8, 12\} \\
    <9> = \{1, 3, 9 \} \\
    <10> = \{1, 3, 4, 9, 10, 12\} \\
    <11> = \{1, 2, 3, 4, 5, 6, 7, 8, 9, 10, 11, 12\} \\ 
    <12> = \{1, 12 \}

    So subgroups are \{1\}, \{1, 12 \}, \{1, 3, 9\}, \{1, 5, 8, 12\}, \{1, 3, 4, 9, 10, 12\}, \{1, 2, 3, 4, 5, 6, 7, 8, 9, 10, 11, 12\}
    
\clearpage
python codes:

\begin{lstlisting}
    for i in range(2,13):
    num=1
    cset = set()
    for j in range(1,20):
        num*=i
        cset.add(num % 13)
    print(i, cset)

\end{lstlisting}



\end{homeworkProblem}
\end{document}