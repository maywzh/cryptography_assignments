\documentclass[lang=cn,11pt,a4paper]{elegantpaper}
%\usepackage{showframe}
\usepackage{fancyhdr}
\title{工程伦理的未来研究方向:微观伦理学,宏观伦理学和专业协会的角色 \\ 翻译自:Future Directions in Engineering Ethics Research: Microethics, Macroethics and the Role of Professional Societies}
\author{Joseph R. Herkert}
\institute{North Carolina State University, USA}

\version{}
\date{}


\begin{document}

\maketitle
\thispagestyle{fancy}         %更改plain状态
\fancyhead{}                     %清除以前的命令
\lhead{\textbf{学号:2019124044}}          %左上角添加
\chead{\textbf{姓名:梅王智汇}}
\rhead{\textbf{学院:伍伦贡联合研究院}}
\lfoot{}
\cfoot{\thepage}  %current page number
\rfoot{}
\renewcommand{\headrulewidth}{0pt}      %把页眉线的宽度设为零,即去掉页眉线
\renewcommand{\footrulewidth}{0pt}

\begin{abstract}
本文讨论三个工程伦理中框架——个人、专业和协会——这可以进一步分为关注个人和工程行业的内部关系的“微观伦理”和关注工程领域的集体社会责任和社会技术决策的“宏观伦理”。很少有人尝试将微观伦理和宏观伦理方法整合到工程伦理中。本文建议的方法是着重于专业工程协会在联系个人和专业伦理以及在联系专业和社会伦理方面的作用。前者的例子是用伦理支持概述了一个研究项目,后者以产品责任立场声明的发布为例。
\keywords{伦理支持,宏观伦理,产品责任,专业协会,公共政策,工程伦理的研究} 
\end{abstract}



\section{工程中的微观伦理和宏观伦理\cite{1,2}}
许多作者认为工程伦理包含多个领域。伦理学家John Ladd\cite{3}将工程伦理学细分为“微观伦理学”或“宏观伦理学”,这取决于关注的焦点是工程师个人与客户、同事和雇主之间的关系,或者该职业的集体社会责任。在每一个案例中,Ladd似乎都关心所谓的“职业道德”,微观道德主要关注职业内部的问题,而宏观道德则关注更广泛的社会背景下的职业责。

作为一名工程师,McLean\cite{4}在讨论工程伦理学时运用了三个范畴:处理工程师技术决策的技术伦理;处理经理、工程师和雇主之间的关系的职业伦理和处理与技术相关的社会政治决策的社会伦理。McLean的职业道德概念比Ladd更狭隘,只包含了Ladd所描述的微观道德的那些维度。与此同时,McLean对与工程相关的道德领域有一个比Ladd更广泛的总体概念,因为他包括个人和社会两个方面。

另一位工程师Vanderburg\cite{5}虽然使用了与Ladd类似的术语,
在区分“个体技术或从业者”的“微观层面”分析和“整体技术”的“宏观层面”分析时,似乎完全忽视了职业道德类别。Vanderburg的分类与另一位工程师Devon相似,\cite{6}他提出了一种新的工程伦理范式,他称之为“社会伦理”,包括与技术相关的“专业知识的社会关系”管理和决策,而不是关注个人。

伦理学家De George\cite{7}对“工程伦理学”和“伦理学”进行了区分。前者关注的是个人的行为,而后者关注的是专业内部的关系和工程专业对社会的责任。De George的“工程伦理”概念因此融合了Ladd的微观和宏观维度。此外,“工程伦理”还具体包括专业工程学会。

如表1所示,在梳理工程伦理的这些不同方面时,出现了一个有趣的模式。三个显而易见的参考框架是:个人、专业和社会。结合Ladd和Vanderburg的术语,“微观伦理学”可以被视为包括对个人和工程专业内部关系的关注,而“宏观伦理学”既适用于工程专业的集体社会责任,也适用于关于技术的社会决策。
\section{研究的问题}
迄今为止,大多数工程伦理学的研究和教学都有一个微观的焦点,要么是Vanderburg使用的术语,要么是Ladd使用的术语。政治哲学家Langdon Winner\cite[p. 62]{8}对这种情况表示遗憾,他批评工程伦理学过分强调微观伦理困境的案例研究,而忽视了与技术发展相关的更大的问题:
\begin{quotation}
  伦理责任……不仅仅是过一种体面的、诚实的、真实的生活,这样的生活当然还是很重要的。当这些选择突然、意外地出现时,它所涉及的远不止是做出明智的选择。我们的道德义务必须……包括愿意让他人参与艰难的工作,以确定技术社会面临的关键选择是什么,以及如何明智地应对这些选择。
\end{quotation}

最近,学者们开始讨论与工程有关的宏观伦理问题\cite{9,10,11}。然而,整合微观伦理和宏观伦理方法的综合框架仍待开发。开发这样一个框架的一个方法是把重点放在专业协会在弥合微观伦理和宏观伦理问题扮演的角色,正如在De George的“工程伦理”的概念中所建议的(见表1)。除了颁布伦理准则,工程专业的专业工程协会的作用已经在很大程度上被忽视了。一些作者,包括Layton \cite{12}和Unger\cite{13} 已经认真对待专业协会的作用,但他们的工作大部分集中在专业协会如何连接职业道德的内部和社会责任维度(尽管如下所述,Unger也讨论了道德支持)。

除了De George的“工程伦理”概念所暗示的作用外,专业协会似乎有潜力作为贯穿表1所示的整个伦理框架统一体的管道;也就是说,专业协会在将个人道德和职业道德以及专业道德和社会道德联系起来方面也可以发挥重要作用。在前一种情况下,专业协会的重要作用是为表现出道德行为的个人提供支持。在后一种情况下,专业协会通过发表关于公共政策事项的立场声明,表面上提供了专业的社会责任和关于技术的社会决定之间的联系。工程学会一直积极参与的一个领域是产品责任改革。

那么,概括地说,这里提出的研究问题是:在把个人道德与职业道德联系起来,以及把职业道德与社会道德联系起来方面,职业社会的作用是什么?

在每种情况下,这个问题大体上可以分为三个组成部分的问题:

\begin{enumerate}
  \item 专业工程协会在连接个人和职业道德(以及连接职业道德和社会道德)方面应该扮演什么样的角色?
  \item 各专业协会在建立这种联系方面有多成功?
  \item 既然没有成功,为什么专业协会不能或不愿建立这种联系?
\end{enumerate}

在本文的其余部分,我将尝试回答这些问题或建议回答他们的研究项目,使用例子道德支持和产品责任改革的情况下,反映在社会最大的专业工程的经验,电气和电子工程师协会(IEEE)。

\begin{table}[htbp]

  \centering
  \caption{工程中的微观伦理和宏观伦理}
  %\resizebox{\textwidth}{!}{
 
  \begin{tabular}{|l|p{2.5cm}|p{3cm}|p{3cm}|p{2.5cm}|}
    \hline
    \multirow{2}{*}{来源} & \multicolumn{2}{l|}{微观伦理} & \multicolumn{2}{l|}{宏观伦理} \\ \cline{2-5} 
                        & 个体          & \multicolumn{2}{l|}{专业}     & 社会          \\ \hline
    Ladd(1980)          &             & 微观伦理——专业人员与客户、同事和雇主之间的专业关系         & 宏观伦理——一个职业的成员作为一个群体所面临的与社会的关系问题(即专业人士的社会责任)          &             \\ \hline
    McLean(1993)        & 技术伦理——工程师做出的技术决定和判断          & 专业伦理——工程师与其他群体(如经理、工程师、雇主)之间的职业道德互动          &             & 社会伦理——社会层面的技术政策决策          \\ \hline
    Vanderburg(1995)    & 微观分析——个体技术或从业者的微观层次分析          &             &             & 宏观分析——整体技术的宏观层面分析          \\ \hline
    Devon(1999)         & 个体伦理          &             &             & 社会伦理——社会伦理与技术管理和决策有关的“专业知识的社会关系”          \\ \hline
    De George(1993)     & 工程中的伦理——工程师个人工程行为的道德规范          & \multicolumn{2}{p{6cm}|}{工程中的伦理——工程道德工程师在工业和其他组织、专业工程协会中的作用,以及该职业的责任}   &             \\ \hline
  \end{tabular}
  %}
\end{table}

在伦理支持的情况下,要解决的问题的具体形式是:

\begin{enumerate}
  \item 专业工程协会在提供伦理支持方面应该扮演什么角色?
  \item 专业协会在提供伦理支持方面有多成功?
  \item 既然没有成功,为什么专业协会不能或不愿提供伦理支持?
\end{enumerate}

在关于产品责任改革的立场声明方面,情况更为复杂,因为专业协会长期以来一直积极发表这种声明。然而,这种立场声明是否确实是考虑到专业或社会道德而提出的,这是一个悬而未决的问题。所要处理的具体问题如下:
\begin{enumerate}
  \item 在就产品责任等公共政策问题发表立场声明时,专业工程协会应该扮演怎样的角色?
  \item 在就产品责任等公共政策问题发表立场声明时,专业协会在纳入道德考量方面有多成功?
  \item 就他们在产品责任等公共政策问题上未能成功发表立场声明的程度而言,为什么专业协会不能或不愿考虑道德问题?
\end{enumerate}

\section{专业协会应该扮演什么样的角色?\cite{1,9}}
第一个问题是规范性的,但要得到答案,我们只需看看专业工程协会颁布的道德规范即可。毕竟,道德准则是专业工程学会对道德的立场的标志。

然而,许多伦理学家如Ladd\cite{3}怀疑相关性和有用性的规范,他们认为这主要是为了创造一个积极的公众形象的专业,很大程度上自私自利,用来从宏观伦理问题转移注意力,是毫无意义的道德推理和伦理传统主义的一种形式。其他人,尤其是Davis,认为规范实际上是工程行业的道德“标准”。Davis\cite{14}给出了工程师应该支持他们的职业规范的几个理由,包括:促进一个支持道德行为的工作环境,并帮助使“他们的职业成为一种他们不需要感到道德上合理的尴尬、羞耻或内疚的实践”。“如果我们接受戴维斯的解释,那么我们应该能够从微观伦理和宏观伦理的规范中推断出规范的位置。

虽然准则因专业团体而异,但在规定工程师对公众、他们的雇主和客户,以及他们的同事的责任方面,它们通常有共同的特点。所有现代工程规范都规定,工程师最重要的责任是保护公共安全、健康和福利。法典还常常强调能力、诚信、诚实和公正等特征。\cite{13}

在考虑的这两种情况下,第一个问题的答案可以从保护公共安全、健康和福利的首要工程道德规范中推断出来。规范似乎暗示专业协会应该支持个人工程师采取行动保护公共安全,健康和福利,专业协会当在公共政策问题发出声明,如产品责任,应当考虑道德义务保护公众安全、健康和福利。否则,就会破坏保护公众安全、健康和福利至关重要的观念。

像大多数工程伦理准则,第一个提供IEEE规范,在1990年实现,拥有最高的公共安全,健康和福利,保证其成员“接受责任工程决策符合安全、公共健康和福利,并及时披露可能危及公众或环境的因素。IEEE代码还包含了针对这两种情况的更具体的语言。该准则的第五个条款承诺IEEE成员“提高对技术的理解、适当的应用和潜在的后果”。“改进对技术的了解似乎包括关于产品责任等公共政策问题的立场声明;技术的适当应用和潜在后果当然包括与公共安全、健康和福利有关的问题。守则的第十项也是最后一项规定直接支持道德支持的概念,承诺IEEE成员“协助同事和同事的专业发展,并支持他们遵守这一道德守则。”

\section{伦理支持\cite{9}}
揭发不道德行为或采取其他符合其道德准则的行动的工程师,往往不得不为他们的道德立场付出高昂的代价,包括降职、解雇、列入黑名单,甚至威胁生命和肢体。许多人认为,期望工程师以这种方式成为“道德英雄”是不合理的\cite{15}。事实上,大量的注意力都集中在为工程师提供道德支持上,即社会成员有培养道德行为的集体责任。\cite{16}通过企业道德办公室和政府监管来提供这种支持的努力,往好了说,也是不完整的。人们一直把职业工程协会看作是工作场所对工程师行为施加的压力的抗衡力量。\cite{13}

\subsection{专业协会在提供伦理支持方面有多成功?}
这是一个经验主义的问题;现有的证据表明,除了颁布道德准则外,专业协会似乎不愿或无法继续努力支持其成员的道德行为。IEEE最近的经历就说明了这一点。在IEEE,员工和志愿者领导者的强烈反对,使长期以来寻求道德支持的努力化为泡影。

IEEE在董事会层面有两个与伦理相关的委员会,成员行为委员会(MCC)和伦理委员会。管理咨询委员会的目的是双重的:建议对被控违反道德准则的成员采取纪律行动,并建议对遵守道德准则而受到报复的成员给予支持。道德委员会成立于1995年,是成员们努力提高IEEE中道德的重要性的结果,它向成员提供信息,并就与道德相关的政策和问题向委员会提供建议。IEEE道德委员会成立后,于1996年设立了一条道德热线,旨在为IEEE关注领域的专业人士提供有关道德问题的信息和建议。伦理热线关注的案例包括侵犯知识产权、伪造质量检测、设计和检测缺陷,这些缺陷可能会对公共安全造成威胁。在某些情况下,这种情况已由MCC\cite{17}提出并采取行动。

在运作了不到一年之后,IEEE伦理热线在1997年被董事会执行委员会暂停。1998年,执行委员会否决并压制了其工作小组建议恢复热线的报告。\cite{17}同样在1998年,IEEE实施细则条款变化,减少办公室的成员中的道德行为委员会成员,使他们更容易出于政治考虑而更换,而且,在IEEE漠视道德规范,这被道德委员会建议成员和其他个人禁止。

道德委员会的网页\cite{18}上现在贴着这样的通知:“IEEE道德委员会将不参与处理个人投诉或向个人提供建议。IEEE内部和外部的其他资源可根据要求提供给个人。“IEEE也有限的角色行为委员会成员提供伦理支持,这在MCC web页面上的免责声明中指明:\cite{19}”执行委员会成员的角色在争端涉及行业、学术界、或政府通常仅限于重述的道德戒律,指导IEEE成员的行为。具有讽刺意味的是,IEEE道德规范中要求伦理支持的条款一直没有改变。

\subsection{为什么专业协会不能或不愿提供伦理支持?}
当大多数反对IEEE内部伦理支持的人指出责任问题时,一个有说服力的论点被Unger\cite{14,17}反驳,一些人还指出,一个伦理热线把IEEE置于他们认为不受欢迎的位置,来调解成员和他们的雇主之间的争端。事实上,企业对专业协会的影响是传统上对社团缺乏道德支持的解释。

这种影响被认为是源于工程和商业之间的关系,莱顿对此有生动的描述,\cite{12}他把工程师描述成一半是科学家,一半是商人,但实际上两者都不是;也就是说,两者都是边际的。这种情况是工程作为一种职业和技术驱动型企业共同进化的结果,在工程师所追求的职业价值和雇主的商业价值之间产生了不可避免的冲突。超过四分之三的工程师在公司工作。这一统计数据与其他行业形成了对比,比如法律和医学,至少在历史上,这些行业的模式一直是让专业人士在私人诊所工作,为客户或病人服务,而不是雇主。莱顿指出,专业人士重视自主权、合谋控制和社会责任,而企业则重视忠诚、从众,并最终把追求利润作为主要目标。工程师的职业道路往往会把他们带入管理层,这一事实加剧了这种紧张;希望在公司层级中晋升的工程师应该在职业生涯的早期就接受商业价值。固有的工程师的员工地位和专业自主权之间的冲突被Davis反驳\cite{14},但是,正如莱顿指出的那样,\cite{12},许多专业工程社会高级成员的领导人已经从技术工程职责到业务管理角色在他们的公司。此外,许多公司资助和支持其雇员参加专业协会。

另一个可能的解释是缺乏来自专业协会的道德支持,与第一个相关,是一种高度重视经济效率而忽视工程的社会环境的工程/商业文化。许多作者在某种程度上将“工程观点”描述为主要关注问题的技术解决方案。工程观点的这一特点也许可以解释为什么有些工程师不愿涉足工程的社会和伦理方面的未知领域。\cite{9} .如我在其他地方所指出的:\cite{20}
\begin{quotation}
  主流的工程文化是很容易从内部和外部识别的。工程师是严肃的问题解决者,受科学理性和创新眼光的指引。效率和实用性是流行语。感情上的偏见和毫无根据的行为是令人厌恶的。给他们一个需要解决的问题,指定边界条件,让他们在没有外部影响(和责任)的情况下去做。如果问题不局限于工作岗位或工厂车间,那么这些问题最好留给管理层或(但愿如此)留给政治家。
\end{quotation}

其他因素可能导致专业工程协会不愿从事伦理支持,保证调查包括:错误维护公共形象的努力,这通过抑制或限制道德讨论和专业工程学会日益国际化所带来的复杂性带来。

\section{产品责任\cite{2,9}}
产品责任改革是专业工程学会参与有关技术开发和使用的公共政策问题的辩论的一个丰富的例子\cite{9}。对美国现行产品责任法持批评态度的人士呼吁对此法进行回滚,这往往接近于过去那种“买家小心”的政策。例如,1996年,美国国会通过了一项立法,对惩罚性损害赔偿设定了上限,并对追究制造商责任提出了更严格的要求,这将严重限制产品责任诉讼的效果。不出所料,克林顿总统否决了该法案\cite{21};然而,关于产品责任改革的争论仍在继续。

产品责任改革的支持者辩称,现行制度不公平地奖励了原告,抑制了技术创新,导致美国制造商缺乏竞争力,降低了产品安全性。现行制度的支持者反驳说,这一制度在劝阻生产缺陷产品和赔偿因缺陷而受伤的人方面,大体上是按其目的起作用的\cite{22}。对一些人来说,关于产品责任改革的争论是典型的商业和消费者冲突。例如,《纽约时报》的一篇社论称\cite{23},拟议中的立法是“1996年的反消费者法案”。

尽管双方都有争论,但关于产品责任奖励是否能改善产品安全方面的证据似乎有好有坏。\cite{22}从社会伦理的观点来看,对产品责任制度的评价及其改革的要求显然是非常令人关切的领域。鉴于工程师对上述公共安全、健康和福利的首要责任,产品责任问题也应该从职业道德的角度进行伦理审查。例如,产品责任诉讼在创造一种环境方面所起的作用是值得考虑的,在这种环境中,有安全顾虑的工程师会得到其经理的听证。正如Ladd\cite{16}和其他人所指出的那样,企业不是道德的代理人,它们唯一的目标是创造利润。为了影响企业的行为,做正确的事情必须符合企业的经济利益。从表面上看,产品责任诉讼似乎是实现这种影响的一种机制。因此在宣传产品责任改革的财务报告时,期待产品责任诉讼的威胁与设计师、质量控制工程师以及其他负责产品安全的人提出和提出安全问题的能力之间的联系得到认真考虑是不无道理的。    

\subsection{在就产品责任等公共政策问题发表立场声明时,专业协会在纳入道德考量方面有多成功?}
这是一个经验主义的问题;现有的证据似乎表明,虽然专业协会对产品责任采取了政策立场,但这些立场并没有受到伦理上的批评。

工程师和工程学会倾向于站在产品责任改革的支持者一边。例如,美国一家大型汽车公司的工程副总裁认为,产品责任限制了工程实践,因为它抑制了创新,阻碍了对安全特性的关键评估,阻碍了新设计或改进设计的实施。\cite{24} IEEE-USA的立场声明产品责任,一些IEEE单位关心的专业问题\cite{25}于1998年发布,要求严格的产品责任限制包括持有制造商的现有标准得到满足时,提供了足够的警告,或产品被误用或由用户改变。其他工程学会,如美国机械工程师学会\cite{26},也积极支持产品责任改革。

在IEEE内部,上面讨论的与伦理相关的委员会没有正式的互动,而且几乎没有与负责起草公共政策问题立场声明的委员会的非正式互动。事实上,IEEE-USA只代表了大约四分之三的居住在美国的IEEE成员,而它的立场声明常常在其母组织内部引起争议。

除了组织上的障碍之外,几乎没有任何证据表明,工程协会在促进产品责任体系的变化时,考虑过降低产品责任对工程道德的影响。毫不奇怪,专业协会并没有将要求产品责任改革的呼声置于道德审查之下,因为总的来说,工程界很少关注产品责任的道德含义。例如,1994年的一个主要在产品责任和创新的国家工程院等问题上的研究\cite{24},认为企业实践、保险、监管,以及科学技术信息的作用在法庭上,只是道德上有过短暂接触(需要解决公众风险感知)\cite{27}。甚至伦理文献在产品责任问题上也是模棱两可的。\cite{28}例如,在De George那篇著名的关于平托案中工程责任的文章中\cite{29},他主张对公司官员实施更严格的监管、罚款和监禁,以达到预期的安全水平,而对产品责任诉讼的作用只是敷衍了事。

\subsection{在就产品责任等公共政策问题发表立场声明时,为什么专业协会不能或不愿考虑伦理?}
专业工程协会对产品责任改革表面上不加批判地接受的可能解释,反映了缺乏伦理支持的潜在原因。特别是,减少责任(这次的工程专业人员),屈从于商业利益,一个工程文化经济效率在社会和伦理的影响,有时不愿意承认工程项目做的可预防的伤害都似乎是社会专业工程的动机的合理解释。

\section{结论}
虽然在工程伦理学的微观伦理问题上仍有重要的工作要做,但在宏观伦理领域的工作却很少,在制定处理工程中的微观伦理和宏观伦理的综合方法方面的工作更少。正如这篇论文所建议的,为此目的的富有成果的研究途径将起到衡量专业工程协会在微观伦理和宏观伦理领域的作用,也即他们在连接个人和职业道德以及连接职业道德和社会道德方面的作用。伦理支持和关于公共政策问题(如产品责任)的立场声明被建议分别作为微观伦理和宏观伦理领域的研究领域。

这里的讨论很大程度上是基于对IEEE伦理活动的学术和专业文献的参与观察和评论。可以而且应该采取类似的方法来调查其他主要工程学会的活动,包括那些作用和机构文化可能与美国有很大不同的非美国的学会。在处理关于公共政策的立场声明时,也必须适当考虑到国家的背景。在美国,产品责任的法律和规范因国家而异。

然而,要弄清这里提出的问题的核心,特别是每个案例中提出的第三个问题,就需要进行更有条理的研究,以揭示个人和机构的价值和动机。Davis\cite{14}和他的同事们在研究工程和管理方面的态度和行为时采用的问卷调查和开放式访谈技术是一种值得认真考虑的方法。在这里讨论的情况下,需要对工作的工程师、专业协会的普通成员、志愿领导人和协会的工作人员以及其他有关方面进行采访。虽然可能有部分领导不情愿的专业协会参与这样的采访中,Davis在十家公司面试的成功和指明在这个想法中重叠的宽广领域以及工程师和管理人员的作用,足以提供一些乐观的理由使得专业协会愿意参与其中。

这里讨论的研究项目的一个目标应该是确定策略,以消除专业工程协会在伦理支持方面的障碍,以及在制定公共政策问题的立场声明过程中引入伦理考虑方面的障碍。

\section{致谢}
本文的部分内容取自参考资料中提到的我之前的工作。我要感谢Michiel Brumsen和Ibo van de Poel邀请我参加2000年罗马举行的”工程伦理研究项目”国际技术和社会研讨会(ISTAS 2000),和许多对论文有用的评论和修订意见。

\nocite{*}
\bibliography{wpref}

\end{document}
