%%%%%%%%%%%%%%%%%%%%%%%%%%%%%%%%%%%%%%%%%%%%%%%
%%%  这是一份 beamer 文档. 本源文件仅供学习参考之用.                                         %%%
%%%  使用请注明出处. 本文作者拥有版权 (c)2017. 保留所有权利.                               %%%
%%%  若不能编译通过, 可能是您的 beamer 需要更新了.                                             %%%
%%%  更新的具体方法可参考: http://bbs.ctex.org/forums/index.php?showtopic=27695 %%%
%%%  温馨提示:文档基本对所都的宏包和命令都加了注释,一般情况下不要删除,会发生错误喔 %%%
%%%%%%%%%%%%%%%%%%%%%%%%%%%%%%%%%%%%%%%%%%%%%%%
\documentclass[CJK,notheorems,compress,mathserif,table,11pt]{beamer} %更改全文字体大小,设置xxpt(如9pt,10pt,.....)%一定要定义documentclass[cjk]{beamer},别忘了“cjk”,否则编译不通过
%\useoutertheme[height=0.1\textwidth,width=0.15\textwidth,hideothersubsections]{sidebar}%加上此命令会出现上部和左侧边框
\usetheme{Madrid}%主题AnnArbor Antibes Bergen Berkeley Berlin Boadilla boxes CambridgeUS Copenhagen Darmstadt
                  %Dresden Frankfurt Goettingen Hannover Ilmenau JuanLesPins Luebeck Madrid MalmoeMarburg Montpellier
                  %PaloAlto Pittsburgh Rochester Singapore Szeged Warsaw
\usecolortheme{whale}           % 外部主题 Outer color themes: whale, seahorse, dolphin
\usecolortheme{orchid}          % 内部主题 Inner color themes: lily, orchid
\useinnertheme[shadow]{rounded} %阴影环绕
\setbeamercolor{sidebar}{bg=blue!60}%sidebar的颜色,百分之六十的蓝色
\setbeamercolor{background canvas}{bg=blue!9}%背景色,百分之九的蓝色
\usefonttheme{serif}            %排版幻灯片中较大的sans或serif字体
%%----------------------------------------常用宏包-----------------------------------------------
\usepackage{amsmath,amssymb}%美国数学会的数学公式宏包(amsmath)、美国数学会的数学符号宏包(amssymb)
\usepackage{amsfonts}%美国数学会提供一套数学符号的字库
\usepackage{latexsym}%数学符号宏包(latexsym)
\usepackage{CJK}      %中文环境
\usepackage{times}    %使用 Times New Roman 字体
\usepackage{ulem}    %下划线
\usepackage{caption} %提供了多种命令来更方便的设计浮动图形和表格的标题式样。
\usepackage{epsfig}  %eps图像
\usepackage{epstopdf}%eps图片转PDF
\usepackage{graphicx}%图形宏包
\usepackage{color}     %支持彩色
\usepackage{subfigure}%子图片宏包
\usepackage{colortbl,dcolumn}
\usepackage{mathrsfs}%字母花写
\usepackage{float}      %浮动环境
\usepackage{pause}   %暂停宏包
\usepackage{bm}       %加粗
\usepackage{bbm}     %再加粗
\usepackage{dsfont}  %一种字体包
\usepackage{pgf,tikz}%TikZ的基本ideas是使用点和路径。pgf是一个在tex系统中的画图宏包,tex尤其是beamer中使用pgf作图,“作精确图还比较方便, 色彩锐丽”(huangzh73)。除了可以精确的作图外,对于某些不要求精确控制的图形绘制,如:流程图,树图,等等,也提供了简便易用的支持。
\usepackage[english]{babel}%%%%%
\usepackage{booktabs}%画三线表,线条精细可变
\usepackage{xmpmulti}%支持文中的 \multiinclude 等命令, 使mp文件逐帧出现. 具体讨论见beamer手册.
\usepackage{hyperref}%超链接命令宏包
\hypersetup{linkcolor=blue,colorlinks=true,CJKbookmarks=true}%超链接命令
 
\usepackage{multirow}%表中的跨行合并宏包(本beamer不可使用跨列合并宏包)
 
%使用 metapost 动画
%\usepackage{xmpmulti}
\DeclareGraphicsRule{*}{mps}{*}{}
 
%自定义二重和三重高斯积分符号
\def\ooint{{\bigcirc}\kern-13.2pt{\int}\kern-6.5pt{\int}}
\def\oooint{{\bigcirc}\kern-12.3pt{\int}\kern-7pt{\int}\kern-7pt{\int}}
%---------------------------------------计时器%-----------------------------------------------
\usepackage[font=HelvI,timeinterval=1,timeduration=20,colorwarningfirst=green,colorwarningsecond=red, timewarningfirst=25,timewarningsecond=75]{tdclock}
%timeinterval表示时间更新间隔
%timeduration表示设定的总时间 timewarningfirst和timewarningsecond分别表示第一次、第二次时间提醒百分值,同时可以更改显示颜色(colorwarningfirst、colorwarningsecond)
%然后在\data里加入
%\date[\initclock \cronominutes\timeseparator\cronoseconds]{\today}
%其中\initclock必须有
%==========================================================
\setbeamertemplate{navigation symbols}{}    %此命令取消导航条,即翻页命令,直达第一页以及最后一页的命令
%\setbeamertemplate{footline}[page number]{} %除掉页面下方的信息条
\setbeamertemplate{caption}[numbered]      %图和表格的标题显示标号
\setbeamertemplate{definitions}[numbered] %默认定义环境是不显示编号的,通过导言区使用开始使用定理编号
\setbeamertemplate{theorems}[numbered]   %定理的标题显示标号
\theoremstyle{plain}                                     %定理环境风格,plain是LaTeX的原始风格
%====================logo图标========================
%\pgfdeclareimage[height=1cm]{logo}{figures/LZU_logo.jpg}
%\logo{\pgfuseimage{logo}}
%\logo{\includegraphics[height=0.1\textwidth]{figures/LZU_logo.jpg}}
%===================自定义页脚=======================
%\hypersetup{dvipdfm,linkcolor=blue,pdfstartview=FitH,
%            CJKbookmarks=true,
%            bookmarksnumbered=false,
%            bookmarksopen=true,
%            colorlinks=true, %注释掉此项则交叉引用为彩色边框(将colorlinks和pdfborder同时注释掉)
%            pdfborder=000,   %注释掉此项则交叉引用为彩色边框
%            citecolor=magenta}
%\usefonttheme{professionalfonts}
%%%%%%%%%%%%%%%%%%%%%%%%%%%%%%%%%%%%%%%%%%%%%%%%%%
%\usefoottemplate{
%\hfill{\insertframenumber\,/ \inserttotalframenumber}
% \hfill{{\the\year}/{\the\month}/{\the\day}}
%}
%\setlength\tabcolsep{2pt}
%\renewcommand{\raggedright}{\leftskip=0pt \rightskip=0pt plus 0cm}
%\raggedright
%\setlength{\parindent}{22pt}
%\def\hilite<#1>{%
%\temporal<#1>{\color{blue!35}}{\color{magenta}}%
%{\color{blue!75}}}
%\newcolumntype{H}{>{\columncolor{blue!20}}c!{\vrule}}
%\newcolumntype{H}{>{\columncolor{blue!20}}c}、
%===================== 参考文献========================
\newcommand{\upcite}[1]{\textsuperscript{\cite{#1}}}  %自定义命令\upcite, 使参考文献引用以上标出现
\bibliographystyle{arabic}%参考文献的命令
%另一种定义方式
%将参考文献图标改成标准格式(不然使用\bibitem{}文献标号就是个小信封图标),\begin{document}之前添加如下:
\setbeamertemplate{bibliography item}[text]
%============自定义: 逐个 item 高亮(\hilite), 或"高黑"(\hidark)============
\def\hilite<#1>{%
\temporal<#1>{\color{blue!35}}{\color{magenta}}%
{\color{blue!75}}}
\def\hidark<#1>{%
\temporal<#1>{\color{black!35}}{\color{magenta}}%
{\color{black}}}
%%%%%%%%%%%%%%%%%%%%%自定义脚注%%%%%%%%%%%%%%%%%%%%%%%%
  \rnode[t]{bza}{\psframebox*[fillcolor=yellow,framesep=0pt,
    linestyle=none]{#1}}
%  \rnode[t]{bza}{\textcolor{blue}{$<$#1$>$}}
  \footnote{
     \rnode[c]{bzb}{\psframebox*[fillcolor=magenta,%bianzhucolor,
        linestyle=none,framearc=0.15]
        {\parbox{25mm}{\small #2}}}
     \ncangles[angleA=-90,angleB=180,armA=0.5ex,armB=3ex,linecolor=red,linewidth=1pt]{bza}{bzb}
  }
}
%%%%%%%%%%%% 重定义字体、字号命令 %%%%%%%%%%%%%
\newcommand{\songti}{\CJKfamily{song}}        % 宋体
\newcommand{\fangsong}{\CJKfamily{fs}}        % 仿宋体
\newcommand{\kaishu}{\CJKfamily{kai}}         % 楷体
\newcommand{\heiti}{\CJKfamily{hei}}          % 黑体
\newcommand{\lishu}{\CJKfamily{li}}           % 隶书
\newcommand{\youyuang}{\CJKfamily{you}}       % 幼圆
%CTeX只添加了GBK编码的六种中文字体(宋体、仿宋、楷体、黑体、隶书和幼圆)
\newcommand{\chaoda}{\fontsize{55pt}{\baselineskip}\selectfont}     % 字豪设置 超大号
\newcommand{\chuhao}{\fontsize{42pt}{\baselineskip}\selectfont}     % 字号设置 初号
\newcommand{\xiaochuhao}{\fontsize{36pt}{\baselineskip}\selectfont} % 字号设置 小初号
\newcommand{\yihao}{\fontsize{28pt}{\baselineskip}\selectfont}      % 字号设置 一号
\newcommand{\erhao}{\fontsize{30pt}{\baselineskip}\selectfont}      % 字号设置 二号
\newcommand{\xiaosan}{\fontsize{16pt}{\baselineskip}\selectfont}    % 字号设置 小三号
\newcommand{\sihao}{\fontsize{14pt}{\baselineskip}\selectfont}      % 字号设置 四号
\newcommand{\xiaosihao}{\fontsize{12pt}{\baselineskip}\selectfont}  % 字号设置 小四号
\newcommand{\wuhao}{\fontsize{10.5pt}{\baselineskip}\selectfont}    % 字号设置 五号
\newcommand{\xiaowuhao}{\fontsize{9pt}{\baselineskip}\selectfont}   % 字号设置 小五号
\newcommand{\liuhao}{\fontsize{7.875pt}{\baselineskip}\selectfont}  % 字号设置 六号
\newcommand{\qihao}{\fontsize{5.25pt}{\baselineskip}\selectfont}    % 字号设置 七号
\newcommand{\xiaoqihao}{\fontsize{5pt}{\baselineskip}\selectfont}   % 字号设置 小七号
\newcommand{\bahao}{\fontsize{2pt}{\baselineskip}\selectfont}       % 字号设置 八号
%===============行或者列的空格命令=======================
\newcommand{\vs}{\vspace{10 pt}}%行间空的距离为10pt大小,产生垂直方向的空白
\newcommand{\hs}{\hspace{21 pt}}%使用命令\hspace{长度}生成水平方向的空格,如\hspace{1cm}长度单位cm可为mm,em,in,pt
%%%%%%%%%%%%%%%%%%%%%%%%%%%%%%%%%%%%%%%%%%%%%%%%%%
%==================定义图,公式、表编号等格式=============
\numberwithin{figure}{section}%\numberwithin是amsmath宏包定义的命令,用在\usepackage{amsmath}后面
\numberwithin{table}{section}
\numberwithin{equation}{section}
%%%%%%%%%%%%%%%%%%%%%%%%%%%%%%%%%%%%%%%%%%%%%%%%%
\hypersetup{pdfpagemode=FullScreen}% 设置用acrobat打开就会全屏显示
%%%%%%%%%%%%%%%%%%%%%%%%%%%%%%%%%%%%%%%%%%%%%%%%%
\begin{document}
\begin{CJK*}{GBK}{kai}%楷体
    %=============在每一小节{section}之前都显示一下目录===================
    \AtBeginSection[xxxx大学]{
        \frametitle{目录}\small
        \tableofcontents[currentsection,currentsubsection,subsectionstyle=show/show/hide]
    }
    %===================================================
    %%%%%%%%%%%%%%%%%%%%%%%%%%%%%%%%%%%
    %======================页面首页======================
    \title[\songti\color{white}{第十次习题课ppt}]{\heiti{第十次习题课ppt}}
    \author[SJTU-Math]{本次ppt使用了XX同学的作业作为样板,特此说明。如觉不妥(侵权),请联系助教xxx}
    {\center\fangsong 答~辩~人:~~~\textcolor[rgb]{0.00,0.00,0.90}{~~x~x~x~~~~~~~~~~~~~~~~~~}
        \center\fangsong 导~~~~~~师:~~~\textcolor[rgb]{0.00,0.00,0.90}{x~x~x~教~授~~~~~~~~~~~~}
        \center\fangsong~~~~~~~~~~~~~~~~\textcolor[rgb]{0.00,0.00,0.90}{~~x~x~x~副~教~授~~~~~~~}
        \center\fangsong~~专~~~~~~业:~~\textcolor[rgb]{0.00,0.00,0.90}{~计~算~数~学~~~~~~~~~~~~~~~~~~~}\\
        \center\fangsong ~方~~~~~~向:~~\textcolor[rgb]{0.00,0.00,0.90}{~数值代数及其应用~~~~~}
    }

    \institute[SJTU-Math]{\fontsize{10pt}{0pt}\selectfont SJTU-Math}%{\textcolor[rgb]{0.85,0.42,0.00}{\fontsize{10pt}{0pt}\selectfont SJTU-Math}}
    \date[\initclock \cronominutes\timeseparator\cronoseconds]{}%此命令为计时器
    %\titlegraphic{\includegraphics[height=1.35cm]{figures/LZU_logo.jpg}}
    \frame{\titlepage }
    %%%%%%%%%%%%%%%%%%%%%%%%%%%%%%%%%%%%%%%%%%%%%%%%
    \newtheorem{theorem}{定理}[section]
    \newtheorem*{THEOREM}{定理(续)}%自定义不参与编号定理
    \newtheorem{definition}{定义}[section]
    \newtheorem{lemma}[theorem]{引理}
    \newtheorem{corollary}[theorem]{推论}
    \newtheorem*{COROLLARY}{推论(续)}%自定义不参与编号推论
    \newtheorem{proposition}{命题}
    \newtheorem{example}{例}[section]
    \newtheorem{remark}{注}
    \newtheorem{conjecture}{猜想}
    \newtheorem{method}{方法}
    \renewcommand\figurename{\rm 图}
    \renewcommand\tablename{\bf 表}
    %%%%%%%%%%%%%%%%%%%开始每个frame的编写就好%%%%%%%%%%%%%%%%

    \begin{frame}

        内容\cite{bai2010modified}

    \end{frame}

    %参考文献的样例

    \begin{frame}[allowframebreaks]{References}
        \bibliographystyle{plain}
        %\bibliographystyle{amsalpha}
        %\bibliography{mybeamer} also works
        \bibliography{./ref.bib}
    \end{frame}
\end{CJK*}
\end{document}